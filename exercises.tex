% Exercise 1.1.19: (iv), (vii), (v), (xii), (xiii);
% Exercise 1.2.33: (v), (xii), (xiii);
% Exercise 1.3.11: (iv);
% Exercise 1.5.19: (ii), (vii);
% Exercise 1.9: 3, 5, 7, 21, 28;
% Exercise 3.1.7: (iv);
% Exercise 3.5.23: (ii);
% Exercise 4.2.27: (ii), (iii), (iv), (v);
% Exercise 5.5: 5,6;
% Exercise 6.2.20: (i), (viii);
% Exercise 7.4.13: (iii);
% Exercise 8.1.21: (i);
% Exercise 8.2.19: (i), (v);

\documentclass[10pt, toc=sectionentrywithdots]{scrartcl}
\usepackage[italian]{babel}
\usepackage{amsmath, amsfonts, amssymb, amsthm}
\usepackage{braket}
\usepackage[utf8]{inputenc}
\usepackage{libertine}
\usepackage{tikz-cd}


\title{Esercizi di Topologia Algebrica}
\author{Gabriele Bozzola \\ Matricola: 882709}
\date{Gennaio 2017}

\newcounter{lemmac}
\newtheorem{lemma}[lemmac]{Lemma}
% \newenvironment{proof}{{\textbf{Dimostrazione}:}}{\hfill $\blacksquare$}

\newcommand{\Z}{\mathbb{Z}}
\newcommand{\Sph}[1][]{\mathcal{S}^#1}
\newcommand{\Disk}[1][]{\mathcal{D}^#1}

\let\latexcirc=\circ
\newcommand{\ccirc}{\mathbin{\mathchoice
  {\xcirc\scriptstyle}
  {\xcirc\scriptstyle}
  {\xcirc\scriptscriptstyle}
  {\xcirc\scriptscriptstyle}
}}
\newcommand{\xcirc}[1]{\vcenter{\hbox{$#1\latexcirc$}}}
\let\circ\ccirc

\begin{document}
\maketitle

\renewcommand\contentsname{Esercizi}

\tableofcontents{}
\addtocontents{toc}{\vspace*{-20pt}~\hfill\textbf{Pagina}\par}


\section[1.1.19 (iv)]{Esercizio 1.1.19 (iv)}

\subsubsection*{Testo}
% Show that homotopy equivalence amongst spaces is an equivalence relation.
Mostra che l'equivalenza omotopica tra spazi topologici è una relazione di
equivalenza.

\subsubsection*{Soluzione}
Siano $ X $ e $ Y $ due spazi topologici, dico che $ X $ è
omotopicamente equivalente a $ Y $ se esiste una funzione
continua $ f \colon X \to Y $ tale che esiste una funzione
continua $ g \colon Y \to X $ tale che $ f \circ g \sim 1_{Y} $ e
$ g \circ f \sim 1_{X} $, dove $ \sim $ indica la relazione di
omotopia tra due applicazioni continue.

Devo mostrare che la relazione di omotopia tra due spazi
topologici è una relazione di equivalenza,
cioè, indicando anche questa relazione con $ \sim $, soddisfa:
\begin{enumerate}
\item Riflessività: $ X \sim X $
\item Simmetria: se $ X \sim Y $ allora $ Y \sim X $
\item Transitività: se $ X \sim Y $ e $ Y \sim Z $ allora $ X \sim Z $
\end{enumerate}
Ma:
\begin{enumerate}
\item Devo trovare una funzione continua $ f \colon X \to X $
  tale che esiste una seconda funzione continua $ g \colon X \to X $
  con $ f \circ g \sim 1_X $ e $ g \circ f \sim 1_X $.
  Una possibile scelta per queste funzioni è $ f = g = 1_X $
  che è tale che $ f \circ g = g \circ f = 1_X \sim 1_X $ per la
  riflessività della relazione di omotopia tra funzioni.
\item Per ipotesi esiste una funzione continua $ f \colon X \to Y $
  tale che esiste una seconda funzione continua $ g \colon Y \to X $
  con $ f \circ g \sim 1_Y $ e $ g \circ f \sim 1_X $, devo trovare
  una funzione continua $ \phi \colon Y \to X $ tale che esiste una
  seconda funzione continua $ \gamma \colon X \to Y $
  con $ \phi \circ \gamma \sim 1_X $ e $ \gamma \circ \phi \sim 1_Y $
  Una possibile scelta per queste funzioni è $ \phi = f $
  e $ \gamma = g $, infatti queste sono funzioni continue
  con il giusto dominio e codominio e sono tali che
  $ \phi \circ \gamma = g \circ f \sim 1_{X} $ e $ \gamma \circ \phi = f \circ g \sim 1_{Y} $.
\end{enumerate}
Per dimostrare il terzo punto è conveniente utilizzare
un lemma:
\begin{lemma}
  \label{lemma:composizione_funzioni_omotopia}
  La relazione di omotopia tra funzioni si comporta bene rispetto alla
  composizione, cioè siano $ X, Y, W, Z $ spazi topologici,
  $ f, g \colon X \to Y $, $ h \colon W \to X $ e $ k \colon Y \to Z $ mappe
  continue con $ f \sim g $, allora $ f \circ h \sim g \circ h $ e
  $ k \circ f \sim h \circ g $.
\end{lemma}
\begin{proof}
  Siccome $ f \circ g $ significa che esiste una funzione
  continua $ F \colon X \times I \to Y $ (con $ I = [0,1] $) tale che:
  \begin{align*}
    F(x, 0) = f(x) \\
    F(x, 1) = g(x)
  \end{align*}
  Definisco $ \phi = f \circ h $ e $ \gamma = g \circ h $, vale che
  $ \phi, \gamma \colon W \to Y $ sono funzioni continue perché
  composizioni di funzioni continue. Devo mostrare che:
  \[
    \exists H \colon W \times I \to Y \text{ continua tale che }
    H(w, 0) = \phi(w) \text{ e } H(w, 1) = \gamma(w)
  \]
  Una possibile scelta per $ H $ è $ H = F \circ (h, 1_I) $,
  questa è continua perché composizione di funzioni
  continue, inoltre è tale che $ H(w, 0) = f \circ h (w) = \phi(w) $
  e $ H(w, 1) = g \circ h (w) = \gamma(w) $, e quindi è l'omotopia
  cercata.
\end{proof}
% \newline \newline
A questo punto:
\begin{enumerate}
  \setcounter{enumi}{2}
\item Per ipotesi so che $ X \sim Y $ e $ Y \sim Z $, cioè
  so che:
  \begin{gather*}
    \exists f_1 \colon X \to Y \text{ tale che } \exists g_1 \colon Y \to X \text{ tale che }
    f_1 \circ g_1 \sim 1_Y \text{ e } g_1 \circ f_1 \sim 1_X \\
    \exists f_2 \colon Y \to Z \text{ tale che } \exists g_2 \colon Z \to Y \text{ tale che }
    f_2 \circ g_2 \sim 1_Z \text{ e } g_2 \circ f_2 \sim 1_Y
  \end{gather*}
  Devo mostrare che:
  \[
    \exists f_3 \colon X \to Z \text{ tale che } \exists g_3 \colon Z \to X \text{ tale che }
    f_3 \circ g_3 \sim 1_Z \text{ e } g_3 \circ f_3 \sim 1_X
  \]
  Una possibile scelta per $ f_3 $ e $ g_3 $ è $ f_3 = f_2 \circ f_1 $
  e $ g_3 = g_1 \circ g_2 $. In questo modo ho
  $ f_3 \colon X \to Z $ e $ g_3 \colon Z \to X $, queste
  mappe sono continue perché sono composizione
  di funzioni continue.
  Perché questa sia una buona scelta deve essere che:
  \begin{enumerate}
  \item  $ f_2 \circ f_1 \circ g_1 \circ g_2 \sim 1_Z $
  \item  $  g_1 \circ g_2 \circ f_2 \circ f_1 \sim 1_X $
  \end{enumerate}
  Definisco $ h = f_1 \circ g_1 \circ g_2 $, mostrare la prima relazione equivale a
  mostrare che $ f_2 \circ h \sim 1_Z $. Per il lemma
  \ref{lemma:composizione_funzioni_omotopia} vale che
  $ h = f_1 \circ g_1 \circ g_2 \sim 1_Y \circ g_2 = g_2 $, in quanto
  $ f_1 \circ g_1 \sim 1_Y $ per ipotesi, ma siccome $ h \sim g_2 $ e
  $ f_2 \circ g_2 \sim 1_Z $ per il medesimo lemma $ f_2 \circ h \sim 1_Z $.

  La seconda relazione è analoga.
\hfill $ \square $
\end{enumerate}

\section[1.1.19 (v)]{Esercizio 1.1.19 (v)}

\subsubsection*{Testo}
Mostra che una mappa equivalente a una equivalenza omotopica è
una equivalenza omotopica.

\subsubsection*{Soluzione}

Se $ X, Y $ sono spazi topologici omotopicamente equivalenti esiste una
funzione continua $ f \colon X \to Y $, detta equivalenza omotopica tale che esista una
seconda funzione continua $ g \colon Y \to X $ con $ f \circ g \sim 1_Y $ e
$ g \circ f \sim 1_X $. Sia $ h \colon X \to Y $ una funzione continua con
$ h \sim f $, devo mostrare che $ h $ è una equivalenza omotopica, cioè che esiste una
funzione continua $ k \colon Y \to X $ tale che $ h \circ k \sim 1_Y $ e
$ k \circ h \sim 1_X $. Una possibile scelta per questa funzione $ k $ è la funzione
$ g $ stessa. Questa è continua e per il lemma
\ref{lemma:composizione_funzioni_omotopia} vale che $ h \circ g \sim 1_Y $ e
$ g \circ h \sim 1_X $ in quanto per ipotesi $ f \circ g \sim 1_Y $ e
$ g \circ f \sim 1_X $ e $ f \sim h $. \hfill $ \square $

\section[1.1.19 (vii)]{Esercizio 1.1.19 (vii)}

\subsubsection*{Testo}

Sia $ f \colon X \to Y $ e $ g \colon Y \to X $ mappe tali che $ f \circ g $ e
$ g \circ f $ siano relazioni omotopiche, mostra che $ f $ e $ g $ sono equivalenze
omotopiche.

\subsubsection*{Soluzione}

Siano $ X, Y $ spazi topologici e $ f \colon X \to X $,
$ g \colon Y \to Y $ funzioni continue tali che $ f \circ g $
e $ g \circ f $ siano equivalenze omotopiche, devo mostrare
che questo implica che $ f $ e $ g $ stesse siano
equivalenze omotopiche, cioè che:
\begin{gather*}
  \exists \phi \colon Y \to X \text{ continua tale che } f \circ \phi \sim 1_Y \text{ e } \phi \circ f \sim 1_X \\
  \exists \gamma \colon X \to Y \text{ continua tale che } g \circ \gamma \sim 1_X \text{ e } \gamma \circ g \sim 1_Y
\end{gather*}
Siccome $ f \circ g $ e $ g \circ f $ sono equivalenze omotopiche vale che:
\begin{gather*}
  \exists h \colon Y \to Y  \text{ continua tale che } f \circ g \circ h \sim 1_Y \text{ e } h \circ f \circ g \sim 1_Y \\
  \exists k \colon X \to X  \text{ continua tale che } g \circ f \circ k \sim 1_X \text{ e } k \circ g \circ f \sim 1_X
\end{gather*}
$ k \circ g \colon Y \to X $ è continua e vale che
$ f \circ g \circ h \sim 1_Y $, allora per il lemma
\ref{lemma:composizione_funzioni_omotopia}
$ k \circ g \circ f \circ g \circ h \sim k \circ g $. Per il medesimo lemma e per il fatto che
$ k \circ g \circ f \sim 1_X $ deriva similmente che $ g \circ h \sim k \circ g $.

Una possibile scelta per $ \phi $ è $ \phi = g \circ h $, infatti utilizzando
il fatto che $ f \circ g $ è equivalenza omotopica:
\[
  f \circ \phi = f \circ g \circ h \sim 1_Y
\]
Inoltre utilizzando l'osservazione appena fatta e il lemma
\ref{lemma:composizione_funzioni_omotopia}:
\[
  g \circ h \circ f \sim k \circ g \circ f \sim 1_X
\]
Anche in questo caso ho utilizzato il fatto che $ g \circ f $ è equivalenza
omotopica.
Si può utilizzare un ragionamento analogo per $ \gamma $.
\hfill $ \square $

\section[1.2.33 (v)]{Esercizio 1.2.33 (v)}

\subsubsection*{Testo}
Considera la mappa $ f \colon \Sph{1} \times \Sph{1} \to \Sph{1} \times \Sph{1} $
data da $ f(z_1, z_2) = (z_1 z_2, z_2) $, calcola l'omomorfismo
indotto $ f_\star $ sul gruppo fondamentale.

\subsubsection*{Soluzione}

È dato l'omomorfismo:
\begin{align*}
  f \colon \Sph{1} \times \Sph{1} & \to  \Sph{1} \times \Sph{1} \\
  (z_1, z_2) & \mapsto (z_1 z_2, z_2)
\end{align*}
Sullo spazio $ \Sph{1} \times \Sph{1} $ ogni laccio è
omotopo ad un laccio della forma:
\begin{align*}
  \sigma \colon \Delta_1 \times \Delta_1 & \to \Sph{1} \times \Sph{1} \\
  (s,t) & \mapsto (\mathrm{e}^{2 \pi i n s}, \mathrm{e}^{2 \pi i m t})
\end{align*}
Dove $ \Delta_1 \simeq [0,1] $ è l'$ 1 $-simplesso standard, e in
cui sostanzialmente $ n, m \in \Z $ contano il numero
di avvolgimenti del laccio attorno alle due circonferenze,
per questo il gruppo fondamentale è $ \pi_1(\Sph{1} \times \Sph{1})
\cong \Z \times \Z $. Sto cercando quindi la mappa:
\begin{align*}
  f_\star \colon \Z \times \Z & \to \Z \times \Z \\
  (n, m) & \mapsto \; ?
\end{align*}
L'azione di $ f_\star $ è definita a partire dai lacci
omotopicamente distinti del gruppo fondamentale:
$ f_\star[\sigma] = [f \circ \sigma] $ quindi:
\[
  \begin{tikzcd}
    (s,t) \rar{\sigma} &  (\mathrm{e}^{2 \pi i n s}, \mathrm{e}^{2 \pi i m t}) \rar &
     (\mathrm{e}^{2 \pi i n s} \mathrm{e}^{2 \pi i m t}, \mathrm{e}^{2 \pi i m t})
  \end{tikzcd}
\]
Le classi di equivalenza distinte sono quelle in cui il numero
di avvolgimenti del laccio intorno a $ \Sph{1} $ è diverso,
quindi:
\begin{align*}
  f_\star \colon \Z \times \Z & \to \Z \times \Z \\
  (n, m) & \mapsto (n+m, m)
\end{align*}
Intuitivamente quello che succede è che $ f $ fa ruotare il punto
sulla prima $ \Sph{1} $ di un angolo pari a quello del punto della seconda
$ \Sph{1} $ e quindi il numero di avvolgimenti si somma.
\hfill $ \square $

\section[1.2.33 (xii)]{Esercizio 1.2.33 (xii)}

\subsubsection*{Testo}

Mostra che $ \Sph{1} $ non è dello stesso tipo di omotopia
di $ \Sph{n} $ per $ n \geq 2 $.

\subsubsection*{Soluzione}

Due spazi topologici hanno lo stesso tipo di omotopia se sono omotopicamente
equivalenti. Per assurdo $ \Sph{1} $ e $ \Sph{n} $ con $ n \geq 2 $ sono
omotopicamente equivalenti, questo implica che i loro gruppi fondamentali sono
isomorfi, indipendentemente dal punto base in quanto gli spazi sono connessi per
archi. Ma $ \pi_1(\Sph{1}) \cong \Z $, mentre $ \pi_1(\Sph{n}) \cong 1 $ per
$ n \geq 2 $, quindi siccome i gruppi fondamentali non sono isomorfi gli spazi non
possono essere omotopicamente equivalenti. \hfill $ \square $

\section[1.2.33 (xiii)]{Esercizio 1.2.33 (xiii)}

\subsubsection*{Testo}
Mostra che $ \mathbb{R}^{2} $ non è omeomorfo a $ \mathbb{R}^{n} $ per ogni
$ n > 2 $.

\subsubsection*{Soluzione}

Per assurdo esiste una funzione continua
$ f \colon \mathbb{R}^2 \to \mathbb{R}^n $ con $ n > 2 $ che sia omeomorfismo. Tolgo un
punto $ p $ da $ \mathbb{R}^2 $, se $ f $ è un omeomorfismo anche la restrizione
$ \tilde{f} $ di $ f $ su $ \mathbb{R}^2 \setminus \set{p} $ è omeomorfismo. Ma
$ \mathbb{R}^{2} \setminus \set{p} \simeq \mathbb{R} \times \mathcal{S}^1 $, infatti una mappa che
realizza esplicitamente questo omeomorfismo è, dopo aver portato $ p $ in $ 0 $
(una traslazione è un omeomorfismo):
\begin{align*}
  \mathbb{R}^{2} \setminus \set{0} & \to \mathbb{R} \times \mathcal{S}^1 \\
  \vec{x} & \mapsto \left( || \vec{x} ||, \frac{\vec{x}}{|| \vec{x} ||} \right)
\end{align*}
Analogamente
$ \mathbb{R}^{n} \setminus \set{f(p)} \simeq \mathbb{R} \times \Sph{n-1} $. Quindi siccome per
ipotesi esiste un omeomorfismo tra $ \mathbb{R}^{2} $ e $ \mathbb{R}^{n} $
allora $ \mathbb{R} \times \Sph{1} \simeq \mathbb{R} \times \Sph{n-1} $, e questo implica che i
gruppi fondamentali sono isomorfi. Siccome il gruppo fondamentale di un prodotto
è il prodotto dei gruppi fondamentali e vale che:
\begin{gather*}
  \pi_1(\mathbb{R}) = 1 \\
  \pi_1(\Sph{1}) = \Z \\
  \pi_1(\Sph{n}) = 1
\end{gather*}
allora $ \pi_1 (\mathbb{R} \times \Sph{1}) = \Z $ e $ \pi_1(\mathbb{R} \times \Sph{n-1}) = 1 $, ma
questi gruppi non sono isomorfi, e quindi ho trovato l'assurdo.
\hfill $ \square $

% \section[1.3.11 (iv)]{Esercizio 1.3.11 (iv)}

\section[1.5.19 (vii)]{Esercizio 1.5.19 (vii)}

\subsubsection*{Testo}

Mostra che ogni mappa $ \Sph{1} \to X $ è omotopa a zero se e solo se $ \pi_1(X, x_0) $
è banale in ogni punto $ x_0 \in X $.

\subsubsection*{Soluzione}

Il gruppo fondamentale di uno spazio topologico $ X $ sul punto base $ x_0 \in X $
è:
\[
  \pi_1(X,x_0) = \set{ f \colon \Sph{1} \to X \text{ continua} | f(1) = x_0 } \slash \sim
\]
Dove $ \sim $ è la relazione di equivalenza omotopica. Una funzione si dice
omotopa a zero quando è omotopa ad una funzione costante.

Se il gruppo fondamentale è banale $ \forall x_0 \in X $ singifica che il suo unico elemento
è la classe di equivalenza del laccio costante $ [1] $, per cui considerata
la generica funzione $ g \colon \Sph{1} \to X $ continua, questa è necessariamente
nella stessa classe di equivalenza di $ [1] $ in $ \pi_1(X, g(1)) $ e ciò singifica
che è omotopa ad un cammino costante, quindi omotopa a zero.

Mostro il viceversa. Fisso $ x_0 \in X $ e considero tutte le funzioni continue
$ h \colon \Sph{1} \to X $ tali che $ h(1) = x_0 $, per ipotesi questi lacci sono
omotopi a zero, quindi sono tutti equivalenti al laccio costante $ C_{h(1)} $.
Per questo motivo il gruppo fondamentale $ \pi_1(X, h(1)) $ contiene la sola
classe di equivalenza del laccio costante, ed è quindi banale. \hfill $ \square $

\section[1.9 (7)]{Esercizio 1.9 (7)}
\label{esercizio:grado_sfere}

\subsubsection*{Testo}

Considera le mappe $ (x_1, \dots, x_{n+1}) \mapsto (\pm x_1, \dots, \pm x_{n+1}) $ da $ \Sph{n} $ a $ \Sph{n} $
e inseriscile in classi di equivalenza rispetto omotopia. Quante classi ci sono?

\subsubsection*{Soluzione}

Per risolvere velocemente questo esercizio sono utili alcuni lemmi:
\begin{lemma}
  Sia $ \rho_i $ la riflessione rispetto al piano $ x_i = 0 $:
  \begin{align*}
    \rho_i \colon \Sph{n} & \to \Sph{n} \\
    (x_1\dots, x_{n+1}) & \mapsto (x_1,\dots,-x_i, \dots, x_{n+1})
  \end{align*}
  allora il suo grado è $ -1 $, cioè $ \deg \rho_i = -1 \; \forall i \in \set{1, \dots, n+1} $.
\end{lemma}
\begin{proof}
  In dimensione $ n + 1 $ ci sono $ n + 1 $ diverse riflessioni, tuttavia
  ciascuna di queste può essere ricondotta a una riflessione di riferimento
  scambiando due opportunamente due coordinate, ma questa operazione è
  un omeomorfismo tra sfere in quanto è continua e ammette inverso continuo,
  % mediante una opportuna rotazione della sfera, ad esempio per $ \Sph{1} $
  % $ \rho_0 = \rho_1 \circ r(\frac{\pi}{2}) $, dove $ r $ è la rotazione di angolo
  % $ \frac{\pi}{2} $ e la sua azione è
  % $ r(\frac{\pi}{2}) \colon (x,y) \to (y,x) $. Le rotazioni sono omeomorfismi di
  % $ \Sph{n} $ in $ \Sph{n} $ e hanno grado $ 1 $ in quanto sostanzialmente
  % consistono in una riparametrizzazione dei generatori del gruppo di omologia.
  e quindi è sufficiente dimostrare che una riflessione in $\Sph{n} $
  ha grado $ - 1 $ per dimostrare che anche tutte le altre hanno grado $ - 1 $.

  Tale dimostrazione è per induzione, per $ n = 1 $:
  \begin{align*}
    \rho \colon \Sph{1} & \to \Sph{1} \\
    (x_0,x_1) & \mapsto (x_0, -x_1)
  \end{align*}
  Considero il generatore $ \sigma $:
  \begin{align*}
    \sigma \colon \Delta_1 & \to \Sph{1} \\
    t & \mapsto \left(\cos(2 \pi t), \sin(2 \pi t)\right)
  \end{align*}
  Quindi:
  \begin{align*}
    \rho \circ \sigma \colon \Delta_1 & \to \Sph{1} \\
    t & \mapsto \left(\cos(2 \pi t), -\sin(2 \pi t))\right)
  \end{align*}
  Ma:
  \begin{gather*}
    \left(\cos(2 \pi t), -\sin(2 \pi t))\right) = \left(\cos(-2 \pi t), \sin(-2 \pi t))\right) = \\
    = \left(\cos(2 \pi (1-t)), \sin(2 \pi (1-t)))\right)
  \end{gather*}
  Quindi $ \rho \circ \sigma = \bar{\sigma} = - \sigma $ e quindi il grado è $ - 1 $.

  Suppongo che il risultato sia vero per $ \Sph{n-1} $ mostro che è vero anche per $ \Sph{n} $.
  Indico con $ \rho^{(k)} $ la riflessione in generica dimensione $ k $.
  % In $ \Sph{n} $ ho dei sottoinsiemi naturali:
  % \begin{gather*}
  %   \Disk{n}_+ = \set{ (x_1, \dots, x_{n+1}) \in \Sph{n} | x_1 \geq 0 } \\
  %   \Disk{n}_- = \set{ (x_1, \dots, x_{n+1}) \in \Sph{n} | x_1 \leq 0 }
  % \end{gather*}
  % Vale che $ \Disk{n}_+ \cap \Disk{n}_- = \set{ (x_1, \dots, x_{n+1}) \in \Sph{n} | x_1 \leq 0 } = \Sph{n-1} $.
  \noindent
  L'omologia di dischi e sfere è nota, e si sa che:
  \[
    \tilde{H}_p(\Sph{n}) \cong H_p(\Disk{n}, \Sph{n-1}) \cong H_p(\Sph{n}, \Disk{n})
  \]
  Siccome $ \rho $ induce una mappa $ \rho_\star $ a livello di omologia:
  \[
    \begin{tikzcd}
      H_n(\Sph{n}) \rar{\rho^{(n)}_\star} \arrow[leftrightarrow]{d}{\cong} & H_n(\Sph{n})  \arrow[leftrightarrow]{d}{\cong} \\
      H_n(\Disk{n}, \Sph{n-1}) &  H_n(\Disk{n}, \Sph{n-1})
    \end{tikzcd}
  \]
  Ma vale anche che $ H_n(\Disk{n}, \Sph{n-1}) \cong H_{n-1}(\Sph{n-1}) $, quindi il diagramma diventa:
  \[
    \begin{tikzcd}
      H_n(\Sph{n}) \rar{\rho^{(n)}_\star} \arrow[leftrightarrow]{d}{\cong} & H_n(\Sph{n})  \arrow[leftrightarrow]{d}{\cong} \\
      H_{n-1}(\Sph{n-1}) \rar{\rho_\star^{(n-1)}} &  H_{n-1}(\Sph{n-1})
    \end{tikzcd}
  \]
  Quindi il grado di $ \rho^{(n)} $ è $ - 1 $ in quanto per ipotesi induttiva il grado di $ \rho^{(n-1)} $ è $ - 1 $.
  % Siccome per ipotesi induttiva per $ n - 1 $ il grado di $ \rho $ è $ - 1 $ allora anche per $ n $ il grado è $ \rho $ è $ - 1 $,
  % a causa degli isomorfismi del precedente diagramma.
  % Si nota che in tutto ciò non si è usato da nessuna parte il fatto che la riflessione è rispetto
  % al piano definito da $ \set{x_{n+1} = 0} $, quindi la presente dimostrazione vale per ogni riflessione.
\end{proof}
\begin{lemma}
  \label{lemma:composizione_grado}
  Siano $ f,g \colon \Sph{n} \to \Sph{n} $ due mappe continue allora
  $ \deg {(f \circ g)} = \deg{f} \, \deg{g} $.
\end{lemma}
\begin{proof}
  Il grado è definito dall'azione delle mappe indotte da $ f $ e $ g $
  su $ H_n(\Sph{n}) $:
  \begin{align*}
    f_\star \colon H_n(\Sph{n}) & \to H_n(\Sph{n}) \\
    \alpha & \mapsto \deg f \, \alpha
  \end{align*}
  Analogamente:
  \begin{align*}
    g_\star \colon H_n(\Sph{n}) & \to H_n(\Sph{n}) \\
    \alpha & \mapsto \deg g \, \alpha
  \end{align*}
  Per la funtorialità:
  \[
    (f \circ g)_\star (\alpha) = (f_\star \circ g_\star)(\alpha) = f_\star(g_\star(\alpha)) = f_\star (\deg g \, \alpha)
  \]
  Siccome $ f_\star $ è omomorfismo:
  \[
    f_\star (\deg g \, \alpha) = \deg g f_\star (\alpha) = \deg g \, \deg f
  \]
  Ma per definizione:
  \[
    (f \circ g)_\star (\alpha) = \deg{(f \circ g)} \alpha
  \]
  Da cui $ \deg{(f \circ g)} = \deg f \, \deg g $.
\end{proof}
\begin{lemma}
  \label{lemma:hopf}
  Siano $ f,g \colon \Sph{n} \to \Sph{n} $ due mappe continue, se
  $ \deg f = \deg g $ allora $ f \sim g $, cioe se due applicazioni
  hanno le stesso grado allora sono omotope.
\end{lemma}
\noindent
% \begin{proof}
%   La dimostrazione non è banale, si assume questo lemma, enunciato
%   a lezione, vero.
% \end{proof}
Utilizzando questi lemmi diventa %dimostrati (o enunciati) a lezione diventa
semplice classificare tutte le mappe della forma:
\[
  (x_1, \dots, x_{n+1}) \mapsto (\pm x_1, \dots, \pm x_{n+1})
\]
Per rappresentare tutto questo insieme si può scrivere:
\[
   r^{i_1\dots i_{n+1}} \colon (x_1, \dots, x_{n+1}) \mapsto ( (-)^{i_1} x_1, \dots, (-)^{i_{n+1}} x_{n+1}) \text{ con } i_j \in \set{0,1} \; \forall j \in \set{1,\dots, n}
\]
L'insieme di queste applicazioni forma un gruppo discreto finito
generato dalle riflessioni $ \rho_i $, e la generica mappa si può
scrivere come:
\[
  r^{i_1\dots i_{n+1}} = \rho_{n+1}^{i_{n+1}} \circ \dots \circ \rho_{1}^{i_{1}} \text{ con } i_j \in \set{0,1} \; \forall j \in \set{1,\dots, n}
\]
Utilizzando il lemma \ref{lemma:composizione_grado} si ha che:
\[
  \deg r^{i_1\dots i_{n+1}} = \prod_j^{n+1} i_{j} = \pm 1
\]
Quindi per il lemma \ref{lemma:hopf} tutte le funzioni $ r^{i_1\dots i_{n+1}} $ si
ripartiscono in due classi di equivalenza, una con rappresentante l'identità,
l'altra con rappresentante la mappa antipodale. \hfill $ \square $

\section[1.9 (21)]{Esercizio 1.9 (21)}

\subsubsection*{Testo}

Sia $ q_1 (z) = z^n $ e $ q_2(z) = \bar{z} $. Calcola il grado di $ q_1, q_2 \colon \Sph{1} \to \Sph{1} $.

\subsubsection*{Soluzione}

Sia:
\begin{align*}
  q_1 \colon  \Sph{1} &\to \Sph{1} \\
  z & \mapsto z^n
\end{align*}
Questa induce una mappa sul gruppo $ H_1(\Sph{1}) $,
il quale è noto essere gruppo libero generato di rango 1.
Un suo generatore è dato dalla classe del simplesso
singolare:
\begin{align*}
  \sigma \colon \Delta_1 & \to \Sph{1} \\
  t & \mapsto \mathrm{e}^{2 \pi i t}
\end{align*}
Ma quindi:
\begin{align*}
  q_1 \circ \sigma \colon \Delta_1 & \to \Sph{1} \\
  t & \mapsto \mathrm{e}^{2 \pi i n t}
\end{align*}
E quindi $ q_1(\sigma) = \sigma \star \sigma \star \sigma \dots = \sigma^n $, cioè sui gruppi
di omologia:
\begin{align*}
  (q_1)_\star \colon H_1(\Sph{1}) & \to H_1(\Sph{1}) \\
  1 & \mapsto n
\end{align*}
Per questo il grado della mappa è $ n $.
Sia:
\begin{align*}
  q_2 \colon  \Sph{1} & \to \Sph{1} \\
  z & \mapsto \bar{z}
\end{align*}
Dove $ \bar{z} $ indica il cammino inverso. Allora
considerando lo stesso generatore:
\begin{align*}
  q_2 \circ \sigma \colon  \Delta_1 & \to \Sph{1} \\
  t & \mapsto \mathrm{e}^{2 \pi i (1 - t)} = \mathrm{e}^{-2 \pi i t}
\end{align*}
Quindi a livello di gruppi di omologia:
\begin{align*}
  (q_2)_\star \colon H_1(\Sph{1}) & \to H_1(\Sph{1}) \\
  1 & \mapsto -1
\end{align*}
E quindi il grado è $ -1 $.
\hfill $ \square $

\section[4.2.27 (v)]{Esercizio 4.2.27 (v)}

\subsubsection*{Testo}
Calcola il grado di ogni riflessione in un piano, cioè $ r \colon \Sph{n} \to \Sph{n} $
definita da:
\[
  \rho_0 \colon (x_0, \dots, x_{n+1}) \mapsto (-x_0, \dots, x_{n+1})
\]
Più in generale calcola il grado di una mappa  $ f_k \colon \Sph{n} \to \Sph{n} $ del tipo:
\[
  f_k \colon (x_0, \dots, x_{n+1}) \mapsto (-x_0, \dots, -x_k, \dots, x_{n+1})
\]

\subsubsection*{Soluzione}

Il grado delle riflessioni è già stato calcolato nell'esercizio
\ref{esercizio:grado_sfere}. Utilizzando i risultati di tale esercizio e le
medesima notazione, la funzione $ f_k $ si può scrivere come
$ f_k = \rho_k \circ \dots \circ \rho_1 $, quindi usando il lemma \ref{lemma:composizione_grado}:
\[
  \deg {f_k} =
  \begin{cases}
    + 1 & \text{se $ k $ è dispari} \\
    - 1 & \text{se $ k $ è pari}
  \end{cases}
\]
\hfill $ \square $


\end{document}

%%% Local Variables:
%%% mode: latex
%%% TeX-master: t
%%% End:
