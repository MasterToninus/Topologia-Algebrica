% Exercise 1.1.19: (iv), (vii), (v), (xii), (xiii);
% Exercise 1.2.33: (v), (xii), (xiii);
% Exercise 1.3.11: (iv);
% Exercise 1.5.19: (ii), (vii);
% Exercise 1.9: 3, 5, 7, 21, 28;
% Exercise 3.1.7: (iv);
% Exercise 3.5.23: (ii);
% Exercise 4.2.27: (ii), (iii), (iv), (v);
% Exercise 5.5: 5,6;
% Exercise 6.2.20: (i), (viii);
% Exercise 7.4.13: (iii);
% Exercise 8.1.21: (i);
% Exercise 8.2.19: (i), (v);

\documentclass[10pt]{scrartcl}
\usepackage[italian]{babel}
\usepackage{amsmath, amsfonts, amssymb, amsthm}
\usepackage[utf8]{inputenc}
\usepackage{libertine}

\title{Esercizi di Topologia Algebrica}
\author{Gabriele Bozzola \\ Matricola: 882709}
\date{Gennaio 2017}

\newtheorem{lemma}[section]{Lemma}
\newenvironment{proof}{{\textbf{Dimostrazione}:}}{\hfill $\blacksquare$}

\begin{document}
\maketitle

\section*{Esercizio 1.1.19 (iv)}
Siano $ X $ e $ Y $ due spazi topologici, dico che $ X $ è
omotopicamente equivalente a $ Y $ se esiste una funzione
continua $ f \colon X \to Y $ tale che esiste una funzione
continua $ g \colon Y \to X $ tale che $ f \circ g \sim 1_{Y} $ e
$ g \circ f \sim 1_{X} $, dove $ \sim $ indica la relazione di
omotopia tra due applicazioni continue.

Devo mostrare che la relazione di omotopia tra due spazi
topologici è una relazione di equivalenza,
cioè, indicando anche questa relazione con $ \sim $, soddisfa:
\begin{enumerate}
\item Riflessività: $ X \sim X $
\item Simmetria: se $ X \sim Y $ allora $ Y \sim X $
\item Transitività: se $ X \sim Y $ e $ Y \sim Z $ allora $ X \sim Z $
\end{enumerate}
Ma:
\begin{enumerate}
\item Devo trovare una funzione continua $ f \colon X \to X $
  tale che esiste una seconda funzione continua $ g \colon X \to X $
  con $ f \circ g \sim 1_X $ e $ g \circ f \sim 1_X $.
  Una possibile scelta per queste funzioni è $ f = g = 1_X $
  che è tale che $ f \circ g = g \circ f = 1_X \sim 1_X $ per la
  riflessività della relazione di omotopia tra funzioni.
\item Per ipotesi esiste una funzione continua $ f \colon X \to Y $
  tale che esiste una seconda funzione continua $ g \colon Y \to X $
  con $ f \circ g \sim 1_Y $ e $ g \circ f \sim 1_X $, devo trovare
  una funzione continua $ \phi \colon Y \to X $ tale che esiste una
  seconda funzione continua $ \gamma \colon X \to Y $
  con $ \phi \circ \gamma \sim 1_X $ e $ \gamma \circ \phi \sim 1_Y $
  Una possibile scelta per queste funzioni è $ \phi = f $
  e $ \gamma = g $, infatti queste sono funzioni continue
  con il giusto dominio e codominio e sono tali che
  $ \phi \circ \gamma = g \circ f \sim 1_{X} $ e $ \gamma \circ \phi = f \circ g \sim 1_{Y} $.
\end{enumerate}
Per dimostrare il terzo punto è conveniente utilizzare
un lemma:
\begin{lemma}
  \label{lemma:composizione_funzioni_omotopia}
  La relazione di omotopia tra funzioni si comporta bene
  rispetto alla composizione, cioè siano $ X, Y, W, Z $
  spazi topologici, $ f, g \colon X \to Y $,
  $ h \colon W \to X $ e $ k \colon Y \to Z $ mappe continue,
  allora $ f \circ h \sim g \circ h $ e $ k \circ f \sim h \circ g $.
\end{lemma}
\begin{proof}
  Siccome $ f \circ g $ significa che esiste una funzione
  continua $ F \colon X \times I \to Y $ tale che:
  \begin{align*}
    F(x, 0) = f(x) \\
    F(x, 1) = g(x)
  \end{align*}
  Definisco $ \phi = f \circ h $ e $ \gamma = g \circ h $, vale che
  $ \phi, \gamma \colon W \to Y $ sono funzioni continue perché
  composizioni di funzioni continue. Devo mostrare che:
  \[
    \exists H \colon W \times I \to Y \text{ continua tale che }
    H(w, 0) = \phi(w) \text{ e } H(w, 1) = \gamma(w)
  \]
  Una possibile scelta per $ H $ è $ H = F \circ (h, 1_I) $,
  questa è continua perché composizione di funzioni
  continue, inoltre è tale che $ H(w, 0) = f \circ h (w) = \phi(w) $
  e $ H(w, 1) = g \circ h (w) = \gamma(w) $, e quindi è l'omotopia
  cercata.
\end{proof}
\newline\newline
A questo punto:
\begin{enumerate}
  \setcounter{enumi}{2}
\item Per ipotesi so che $ X \sim Y $ e $ Y \sim Z $, cioè
  so che:
  \begin{gather*}
    \exists f_1 \colon X \to Y \text{ tale che } \exists g_1 \colon Y \to X \text{ tale che }
    f_1 \circ g_1 \sim 1_Y \text{ e } g_1 \circ f_1 \sim 1_X \\
    \exists f_2 \colon Y \to Z \text{ tale che } \exists g_2 \colon Z \to Y \text{ tale che }
    f_2 \circ g_2 \sim 1_Z \text{ e } g_2 \circ f_2 \sim 1_Y
  \end{gather*}
  Devo mostrare che:
  \[
    \exists f_3 \colon X \to Z \text{ tale che } \exists g_2 \colon Z \to X \text{ tale che }
    f_3 \circ g_3 \sim 1_Z \text{ e } g_3 \circ f_3 \sim 1_X
  \]
  Una possibile scelta per $ f_3 $ e $ g_3 $ è $ f_3 = f_2 \circ f_1 $
  e $ g_3 = g_1 \circ g_2 $. In questo modo ho
  $ f_3 \colon X \to Z $ e $ g_3 \colon Z \to X $, queste
  mappe sono continue perché sono composizione
  di funzioni continue.
  Perché questa sia una buona scelta deve essere
  $ f_2 \circ f_1 \circ g_1 \circ g_2 \sim 1_Z $
  e  $  g_1 \circ g_2 \circ f_2 \circ f_1 \sim 1_X $.

  Nel primo caso devo mostrare che $ f_2 \circ h \sim 1_Z $ con
  $ h = f_1 \circ g_1 \circ g_2 \sim 1_Y \circ g_2 = g_2 $ per il lemma
  \ref{lemma:composizione_funzioni_omotopia}, in quanto $ f_1 \circ g_1 \sim 1_Y $
  per ipotesi. Siccome $ h \sim g_2 $ e $ f_2 \circ g_2 \sim 1_Z $ per il medesimo
  lemma $ f_2 \circ h \sim 1_Z $. La seconda relazione è analoga.
\hfill $ \square $
\end{enumerate}

\section*{Esercizio 1.1.19 (v)}
Siano $ X, Y $ spazi topologici omotopicamente equivalenti quindi
esiste una funzione continua $ f \colon X \to Y $, detta
\emph{relazione di omotopia} tale che esista una seconda
funzione continua $ g \colon Y \to X $ con $ f \circ g \sim 1_Y $
e $ g \circ f \sim 1_X $. Sia $ h \colon X \to Y $ una funzione
continua con $ h \sim f $, devo mostrare che $ h $ è una
relazione di omotopia, cioè esiste una funzione continua
$ k \colon Y \to X $ tale che $ h \circ k \sim 1_Y $ e $ k \circ h \sim 1_X $.
Una possibile scelta per questa funzione $ k $ è la funzione
$ g $ stessa. Questa è continua e per il lemma \ref{lemma:composizione_funzioni_omotopia}
vale che $ h \circ g \sim 1_Y $ e $ g \circ h \sim 1_X $ in quanto per
ipotesi  $ f \circ g \sim 1_Y $ e $ g \circ f \sim 1_X $ e $ f \sim h $.
\hfill $ \square $

\section*{Esercizio 1.1.19 (vii)}
Siano $ X, Y $ spazi topologici e $ f \colon X \to X $,
$ g \colon Y \to Y $ funzioni continue tali che $ f \circ g $
e $ g \circ f $ siano equivalenze omotopiche, devo mostrare
che questo implica che $ f $ e $ g $ stesse siano
equivalenze omotopiche, cioè che:
\begin{gather*}
  \exists \phi \colon Y \to X \text{ continua tale che } f \circ \phi \sim 1_Y \text{ e } \phi \circ f \sim 1_X \\
  \exists \gamma \colon X \to Y \text{ continua tale che } g \circ \gamma \sim 1_X \text{ e } \gamma \circ g \sim 1_Y
\end{gather*}

\end{document}

%%% Local Variables:
%%% mode: latex
%%% TeX-master: t
%%% End:
