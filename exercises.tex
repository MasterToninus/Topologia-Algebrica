% Exercise 1.1.19: (iv), (vii), (v), (xii), (xiii);
% Exercise 1.2.33: (v), (xii), (xiii);
% Exercise 1.3.11: (iv);
% Exercise 1.5.19: (ii), (vii);
% Exercise 1.9: 3, 5, 7, 21, 28;
% Exercise 3.1.7: (iv);
% Exercise 3.5.23: (ii);
% Exercise 4.2.27: (ii), (iii), (iv), (v);
% Exercise 5.5: 5,6;
% Exercise 6.2.20: (i), (viii);
% Exercise 7.4.13: (iii);
% Exercise 8.1.21: (i);
% Exercise 8.2.19: (i), (v);

\documentclass[10pt]{scrartcl}
\usepackage[italian]{babel}
\usepackage{amsmath, amsfonts, amssymb, amsthm}
\usepackage{braket}
\usepackage[utf8]{inputenc}
\usepackage{libertine}

\title{Esercizi di Topologia Algebrica}
\author{Gabriele Bozzola \\ Matricola: 882709}
\date{Gennaio 2017}

\newtheorem{lemma}[section]{Lemma}
% \newenvironment{proof}{{\textbf{Dimostrazione}:}}{\hfill $\blacksquare$}

\newcommand{\Z}{\mathbb{Z}}
\newcommand{\Sph}[1][]{\mathcal{S}^#1}


\begin{document}
\maketitle

\section*{Esercizio 1.1.19 (iv)}
Siano $ X $ e $ Y $ due spazi topologici, dico che $ X $ è
omotopicamente equivalente a $ Y $ se esiste una funzione
continua $ f \colon X \to Y $ tale che esiste una funzione
continua $ g \colon Y \to X $ tale che $ f \circ g \sim 1_{Y} $ e
$ g \circ f \sim 1_{X} $, dove $ \sim $ indica la relazione di
omotopia tra due applicazioni continue.

Devo mostrare che la relazione di omotopia tra due spazi
topologici è una relazione di equivalenza,
cioè, indicando anche questa relazione con $ \sim $, soddisfa:
\begin{enumerate}
\item Riflessività: $ X \sim X $
\item Simmetria: se $ X \sim Y $ allora $ Y \sim X $
\item Transitività: se $ X \sim Y $ e $ Y \sim Z $ allora $ X \sim Z $
\end{enumerate}
Ma:
\begin{enumerate}
\item Devo trovare una funzione continua $ f \colon X \to X $
  tale che esiste una seconda funzione continua $ g \colon X \to X $
  con $ f \circ g \sim 1_X $ e $ g \circ f \sim 1_X $.
  Una possibile scelta per queste funzioni è $ f = g = 1_X $
  che è tale che $ f \circ g = g \circ f = 1_X \sim 1_X $ per la
  riflessività della relazione di omotopia tra funzioni.
\item Per ipotesi esiste una funzione continua $ f \colon X \to Y $
  tale che esiste una seconda funzione continua $ g \colon Y \to X $
  con $ f \circ g \sim 1_Y $ e $ g \circ f \sim 1_X $, devo trovare
  una funzione continua $ \phi \colon Y \to X $ tale che esiste una
  seconda funzione continua $ \gamma \colon X \to Y $
  con $ \phi \circ \gamma \sim 1_X $ e $ \gamma \circ \phi \sim 1_Y $
  Una possibile scelta per queste funzioni è $ \phi = f $
  e $ \gamma = g $, infatti queste sono funzioni continue
  con il giusto dominio e codominio e sono tali che
  $ \phi \circ \gamma = g \circ f \sim 1_{X} $ e $ \gamma \circ \phi = f \circ g \sim 1_{Y} $.
\end{enumerate}
Per dimostrare il terzo punto è conveniente utilizzare
un lemma:
\begin{lemma}
  \label{lemma:composizione_funzioni_omotopia}
  La relazione di omotopia tra funzioni si comporta bene
  rispetto alla composizione, cioè siano $ X, Y, W, Z $
  spazi topologici, $ f, g \colon X \to Y $,
  $ h \colon W \to X $ e $ k \colon Y \to Z $ mappe continue,
  allora $ f \circ h \sim g \circ h $ e $ k \circ f \sim h \circ g $.
\end{lemma}
\begin{proof}
  Siccome $ f \circ g $ significa che esiste una funzione
  continua $ F \colon X \times I \to Y $ tale che:
  \begin{align*}
    F(x, 0) = f(x) \\
    F(x, 1) = g(x)
  \end{align*}
  Definisco $ \phi = f \circ h $ e $ \gamma = g \circ h $, vale che
  $ \phi, \gamma \colon W \to Y $ sono funzioni continue perché
  composizioni di funzioni continue. Devo mostrare che:
  \[
    \exists H \colon W \times I \to Y \text{ continua tale che }
    H(w, 0) = \phi(w) \text{ e } H(w, 1) = \gamma(w)
  \]
  Una possibile scelta per $ H $ è $ H = F \circ (h, 1_I) $,
  questa è continua perché composizione di funzioni
  continue, inoltre è tale che $ H(w, 0) = f \circ h (w) = \phi(w) $
  e $ H(w, 1) = g \circ h (w) = \gamma(w) $, e quindi è l'omotopia
  cercata.
\end{proof}
% \newline \newline
A questo punto:
\begin{enumerate}
  \setcounter{enumi}{2}
\item Per ipotesi so che $ X \sim Y $ e $ Y \sim Z $, cioè
  so che:
  \begin{gather*}
    \exists f_1 \colon X \to Y \text{ tale che } \exists g_1 \colon Y \to X \text{ tale che }
    f_1 \circ g_1 \sim 1_Y \text{ e } g_1 \circ f_1 \sim 1_X \\
    \exists f_2 \colon Y \to Z \text{ tale che } \exists g_2 \colon Z \to Y \text{ tale che }
    f_2 \circ g_2 \sim 1_Z \text{ e } g_2 \circ f_2 \sim 1_Y
  \end{gather*}
  Devo mostrare che:
  \[
    \exists f_3 \colon X \to Z \text{ tale che } \exists g_2 \colon Z \to X \text{ tale che }
    f_3 \circ g_3 \sim 1_Z \text{ e } g_3 \circ f_3 \sim 1_X
  \]
  Una possibile scelta per $ f_3 $ e $ g_3 $ è $ f_3 = f_2 \circ f_1 $
  e $ g_3 = g_1 \circ g_2 $. In questo modo ho
  $ f_3 \colon X \to Z $ e $ g_3 \colon Z \to X $, queste
  mappe sono continue perché sono composizione
  di funzioni continue.
  Perché questa sia una buona scelta deve essere
  $ f_2 \circ f_1 \circ g_1 \circ g_2 \sim 1_Z $
  e  $  g_1 \circ g_2 \circ f_2 \circ f_1 \sim 1_X $.

  Nel primo caso devo mostrare che $ f_2 \circ h \sim 1_Z $ con
  $ h = f_1 \circ g_1 \circ g_2 \sim 1_Y \circ g_2 = g_2 $ per il lemma
  \ref{lemma:composizione_funzioni_omotopia}, in quanto $ f_1 \circ g_1 \sim 1_Y $
  per ipotesi. Siccome $ h \sim g_2 $ e $ f_2 \circ g_2 \sim 1_Z $ per il medesimo
  lemma $ f_2 \circ h \sim 1_Z $. La seconda relazione è analoga.
\hfill $ \square $
\end{enumerate}

\section*{Esercizio 1.1.19 (v)}
Siano $ X, Y $ spazi topologici omotopicamente equivalenti quindi
esiste una funzione continua $ f \colon X \to Y $, detta
\emph{relazione di omotopia} tale che esista una seconda
funzione continua $ g \colon Y \to X $ con $ f \circ g \sim 1_Y $
e $ g \circ f \sim 1_X $. Sia $ h \colon X \to Y $ una funzione
continua con $ h \sim f $, devo mostrare che $ h $ è una
relazione di omotopia, cioè esiste una funzione continua
$ k \colon Y \to X $ tale che $ h \circ k \sim 1_Y $ e $ k \circ h \sim 1_X $.
Una possibile scelta per questa funzione $ k $ è la funzione
$ g $ stessa. Questa è continua e per il lemma \ref{lemma:composizione_funzioni_omotopia}
vale che $ h \circ g \sim 1_Y $ e $ g \circ h \sim 1_X $ in quanto per
ipotesi  $ f \circ g \sim 1_Y $ e $ g \circ f \sim 1_X $ e $ f \sim h $.
\hfill $ \square $

\section*{Esercizio 1.1.19 (vii)}
Siano $ X, Y $ spazi topologici e $ f \colon X \to X $,
$ g \colon Y \to Y $ funzioni continue tali che $ f \circ g $
e $ g \circ f $ siano equivalenze omotopiche, devo mostrare
che questo implica che $ f $ e $ g $ stesse siano
equivalenze omotopiche, cioè che:
\begin{gather*}
  \exists \phi \colon Y \to X \text{ continua tale che } f \circ \phi \sim 1_Y \text{ e } \phi \circ f \sim 1_X \\
  \exists \gamma \colon X \to Y \text{ continua tale che } g \circ \gamma \sim 1_X \text{ e } \gamma \circ g \sim 1_Y
\end{gather*}
Siccome $ f \circ g $ e $ g \circ f $ sono equivalenze omotopiche vale che:
\begin{gather*}
  \exists h \colon Y \to Y  \text{ continua tale che } f \circ g \circ h \sim 1_Y \text{ e } h \circ f \circ g \sim 1_Y \\
  \exists k \colon X \to X  \text{ continua tale che } g \circ f \circ k \sim 1_X \text{ e } k \circ g \circ f \sim 1_X
\end{gather*}
Affermo che $ k \circ g \circ f \circ g \circ h \sim k \circ g $ questo è vero siccome $ k \circ g \colon Y \to X $ è continua
e siccome $ f \circ g \circ h \sim 1_Y $ posso utilizzare il lemma
\ref{lemma:composizione_funzioni_omotopia}. Ma per il medesimo lemma e
per il fatto che $  k \circ g \circ f \sim 1_X $ deriva che $ g \circ h \sim k \circ g $.

Una possibile scelta per $ \phi $ è $ \phi = g \circ h $, infatti utilizzando
il fatto che $ f \circ g $ è equivalenza omotopica:
\[
  f \circ \phi = f \circ g \circ h \sim 1_Y
\]
Inoltre utilizzando l'osservazione appena fatta e il lemma
\ref{lemma:composizione_funzioni_omotopia}:
\[
  g \circ h \circ f \sim k \circ g \circ f \sim 1_X
\]
Anche in questo caso ho utilizzato il fatto che $ g \circ f $ è equivalenza
omotopica.

Si può utilizzare un ragionamento analogo per $ \gamma $.
\hfill $ \square $

\section*{Esercizio 1.2.33 (v)}
È dato l'omomorfismo:
\begin{align*}
  f \colon \Sph{1} \times \Sph{1} & \to  \Sph{1} \times \Sph{1} \\
  (z_1, z_2) & \mapsto (z_1 z_2, z_2)
\end{align*}
Sullo spazio $ \Sph{1} \times \Sph{1} $ ogni laccio è
omotopo ad un laccio della forma:
\begin{align*}
  \sigma \colon \Delta_1 \times \Delta_1 & \to \Sph{1} \times \Sph{1} \\
  (s,t) & \mapsto (\mathrm{e}^{2 \pi i n s}, \mathrm{e}^{2 \pi i m t})
\end{align*}
Dove $ \Delta_1 \simeq [0,1] $ è l'$ 1 $-simplesso standard, e in
cui sostanzialmente $ n, m \in \Z $ contano il numero
di avvolgimenti del laccio attorno alle due circonferenze,
per questo il gruppo fondamentale è $ \pi_1(\Sph{1} \times \Sph{1})
\cong \Z \times \Z $.

\section*{Esercizio 1.2.33 (xii)}
Per assurdo $ \Sph{1} $ e $ \Sph{n} $ con $ n \geq 2 $ sono omotopicamente
equivalenti, questo implica che i loro gruppi fondamentali sono isomorfi,
indipendentemente dal punto base in quanto gli spazi sono connessi per archi.
Ma $ \pi_1(\Sph{1}) \cong \Z $, mentre $ \pi_1(\Sph{n}) \cong 0 $ per $ n \geq 2 $, quindi
siccome i gruppi fondamentali non sono isomorfi gli spazi non possono
essere omotopicamente equivalenti.
\hfill $ \square $

\section*{Esercizio 1.2.33 (xiii)}
Per assurdo esiste una funzione continua $ f \colon \mathbb{R}^2 \to \mathbb{R}^n $
con $ n > 2 $ che sia omeomorfismo. Tolgo un punto $ p $ da $ \mathbb{R}^2 $,
se $ f $ omeomorfismo anche la restrizione $ \tilde{f} $ di $ f $ su
$ \mathbb{R}^2 \setminus \set{p} $ è omeomorfismo.
Ma  $ \mathbb{R}^{2} \setminus  \set{p} \simeq \mathbb{R} \times \mathcal{S}^1 $, infatti una mappa
che realizza esplicitamente questo omeomorfismo è, dopo aver portato $ p $ in $ 0 $
(una traslazione è un omeomorfismo):
\begin{align*}
  \mathbb{R}^{2} \setminus \set{p} & \to \mathbb{R} \times \mathcal{S}^1 \\
  \vec{x} & \mapsto \left( || \vec{x} ||, \frac{\vec{x}}{|| \vec{x} ||} \right)
\end{align*}
Analogamente $ \mathbb{R}^{n} \setminus \set{f(p)}  \simeq \mathbb{R} \times \Sph{n-1} $.
Quindi siccome per ipotesi esiste un omeomorfismo tra $ \mathbb{R}^{2} $ e
$ \mathbb{R}^{n} $ allora $ \mathbb{R} \times \Sph{1} \simeq \mathbb{R} \times \Sph{n-1} $,
e questo implica che i gruppi fondamentali sono isomorfi.
Siccome il gruppo fondamentale di un prodotto è il prodotto dei gruppi fondamentali
e vale che:
\begin{gather*}
  \pi_1(\mathbb{R}) = 0 \\
  \pi_1(\Sph{1}) = \Z \\
  \pi_1(\Sph{n}) = 0
\end{gather*}
allora $ \pi_1 (\mathbb{R} \times \Sph{1}) = \Z $ e $ \pi_1(\mathbb{R} \times \Sph{n-1}) = 0 $, ma
questi gruppi non sono isomorfi, e quindi ho trovato l'assurdo.
\hfill $ \square $

% \section*{Esercizio 1.3.11 (iv)}


\section*{Esercizio 1.5.19 (vii)}
Il gruppo fondamentale di uno spazio topologico $ X $ sul punto base $ x_0 \in X $
è:
\[
  \pi_1(X,x_0) = \set{ f \colon \Sph{1} \to X \text{ continua} | f(1) = x_0 } \slash \sim
\]
Dove $ \sim $ è la relazione di equivalenza omotopica. Una funzione si dice
omotopa a zero quando è omotopa ad una funzione costante.

Se il gruppo fondamentale è banale $ \forall x_0 \in X $ singifica che il suo unico elemento
è la classe di equivalenza del laccio costante $ [1] $, per cui considerata
la generica funzione $ g \colon \Sph{1} \to X $ continua, questa è necessariamente
nella stessa classe di equivalenza di $ [1] $ in $ \pi_1(X, g(1)) $ e ciò singifica
che è omotopa ad un cammino costante, quindi omotopa a zero.

Mostro il viceversa. Se tutte le funzioni $ h \colon \Sph{1} \to X $ sono omotope a zero
allora sono tutte equivalenti al laccio costante $ C_{h(1)} $ e quindi il gruppo
fondamentale $ \pi_1(X, h(1)) $ contiene questa sola classe di equivalenza, ed è
quindi banale. Questo vale $ \forall x_0 \in X $, basta considerare una funzione tale
che $ h(1) = x_0 $.

\section*{Esercizio 1.9 (21)}
Sia:
\begin{align*}
  q_1 \colon & \Sph{1} \to \Sph{1} \\
  z & \mapsto z^n
\end{align*}
Questa induce una mappa sul gruppo $ H_1(\Sph{1}) $,
il quale è noto essere gruppo libero generato di rango 1.
Un suo generatore è dato dalla classe del simplesso
singolare:
\begin{align*}
  \sigma \colon \Delta_1 & \to \Sph{1} \\
  t & \mapsto \mathrm{e}^{2 \pi i t}
\end{align*}
Ma quindi:
\begin{align*}
  q_1 \circ \sigma \colon \Delta_1 & \to \Sph{1} \\
  t & \mapsto \mathrm{e}^{2 \pi i n t}
\end{align*}
E quindi $ q_1(\sigma) = \sigma \star \sigma \star \sigma \dots = \sigma^n $, cioè sui gruppi
di omologia:
\begin{align*}
  (q_1)_\star \colon H_1(\Sph{1}) & \to H_1(\Sph{1}) \\
  1 & \mapsto n
\end{align*}
Per questo il grado della mappa è $ n $.
Sia:
\begin{align*}
  q_2 \colon & \Sph{1} \to \Sph{1} \\
  z & \mapsto \bar{z}
\end{align*}
Dove $ \bar{z} $ indica il cammino inverso. Allora
considerando lo stesso generatore:
\begin{align*}
  q_2 \circ \sigma \colon & \Delta_1 \to \Sph{1} \\
  t & \mapsto \mathrm{e}^{2 \pi i (1 - t)} = \mathrm{e}^{-2 \pi i t}
\end{align*}
Quindi a livello di gruppi di omologia:
\begin{align*}
  (q_2)_\star \colon H_1(\Sph{1}) & \to H_1(\Sph{1}) \\
  1 & \mapsto -1
\end{align*}
E quindi il grado è $ -1 $.
\hfill $ \square $

\end{document}

%%% Local Variables:
%%% mode: latex
%%% TeX-master: t
%%% End:
