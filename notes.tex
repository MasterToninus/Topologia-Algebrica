\documentclass{article}

\usepackage{amsmath,amsfonts,amssymb}
\usepackage{graphicx}
\usepackage{lmodern}
\usepackage{tikz}
\usepackage{epigraph}
\usepackage{lipsum}
\usepackage[utf8]{inputenc}
\usepackage{braket}
\usepackage[italian]{babel}
\usepackage{tikz-cd}

\renewcommand\epigraphflush{flushright}
\renewcommand\epigraphsize{\normalsize}
\setlength\epigraphwidth{0.7\textwidth}

\definecolor{titlepagecolor}{cmyk}{1,.60,0,.40}

\DeclareFixedFont{\titlefont}{T1}{ppl}{b}{it}{0.5in}

\makeatletter
\def\printauthor{%
    {\large \@author}}
\makeatother
\author{%
  Professore: \\
  \emph{Gilberto Bini} \vspace{10pt}\\
  Scriba: \\
  \emph{Gabriele Bozzola}
    }

\newcommand\titlepagedecoration{%
\begin{tikzpicture}[remember picture,overlay,shorten >= -10pt]

\coordinate (aux1) at ([yshift=-15pt]current page.north east);
\coordinate (aux2) at ([yshift=-410pt]current page.north east);
\coordinate (aux3) at ([xshift=-4.5cm]current page.north east);
\coordinate (aux4) at ([yshift=-150pt]current page.north east);

\begin{scope}[titlepagecolor!40,line width=12pt,rounded corners=12pt]
\draw
  (aux1) -- coordinate (a)
  ++(225:5) --
  ++(-45:5.1) coordinate (b);
\draw[shorten <= -10pt]
  (aux3) --
  (a) --
  (aux1);
\draw[opacity=0.6,titlepagecolor,shorten <= -10pt]
  (b) --
  ++(225:2.2) --
  ++(-45:2.2);
\end{scope}
\draw[titlepagecolor,line width=8pt,rounded corners=8pt,shorten <= -10pt]
  (aux4) --
  ++(225:0.8) --
  ++(-45:0.8);
\begin{scope}[titlepagecolor!70,line width=6pt,rounded corners=8pt]
\draw[shorten <= -10pt]
  (aux2) --
  ++(225:3) coordinate[pos=0.45] (c) --
  ++(-45:3.1);
\draw
  (aux2) --
  (c) --
  ++(135:2.5) --
  ++(45:2.5) --
  ++(-45:2.5) coordinate[pos=0.3] (d);
\draw
  (d) -- +(45:1);
\end{scope}
\end{tikzpicture}%
}

\newcounter{lecnum}
\renewcommand{\thepage}{\thelecnum-\arabic{page}}
\renewcommand{\thesection}{\thelecnum.\arabic{section}}
\renewcommand{\theequation}{\thelecnum.\arabic{equation}}
\renewcommand{\thefigure}{\thelecnum.\arabic{figure}}
\renewcommand{\thetable}{\thelecnum.\arabic{table}}

\newcommand{\lecture}[3]{
   \pagestyle{myheadings}
   \thispagestyle{plain}
   \newpage
   \setcounter{lecnum}{#1}
   \setcounter{page}{1}
   \noindent
   \begin{center}
   \framebox{
      \vbox{\vspace{2mm}
    \hbox to \linewidth { {\bf Topologia Algebrica
	\hfill 2016/2017} }
       \vspace{4mm}
       \hbox to \linewidth  { {\Large \hfill Lezione #1: #2  \hfill} }
       \vspace{2mm}
       \hbox { {Agomenti: #3  \hfill} }
      \vspace{2mm}}
   }
   \end{center}
   \markboth{Lezione #1: #2}{Lezione #1: #2}
}


\newtheorem{theorem}{Teorema}[lecnum]
\newtheorem{lemma}[theorem]{Lemma}
\newtheorem{proposition}[theorem]{Propositione}
\newtheorem{osservation}[theorem]{Osservazione}
\newtheorem{corollary}[theorem]{Corollario}
\newtheorem{definition}[theorem]{Definitione}
\newcounter{exercises}
\newtheorem{exercise}[exercises]{Esercizio}
\newenvironment{proof}{{\bf Dimostrazione:}}{\hfill\rule{2mm}{2mm}}

\newcommand{\R}{\mathcal{R}}
\newcommand{\M}{\mathcal{M}}
\newcommand{\N}{\mathcal{N}}
\newcommand{\Z}{\mathbb{Z}}
\newcommand{\im}{\mathrm{Im}}
\renewcommand{\ker}{\mathrm{Ker}}
\renewcommand{\phi}{\varphi}
\newcommand{\RN}[1][]{\mathbb{R}^#1}
\newcommand{\Sph}[1][]{\mathcal{S}^#1}

\newcommand*\quot[2]{{^{\textstyle #1}\big/_{\textstyle #2}}}

\begin{document}

\begin{titlepage}

\noindent
\titlefont Topologia Algebrica
\epigraph{Zitto e studia.}%
{\textit{Parigi 1905}\\ \textsc{H.\ Poincarè}}
\null\vfill
\vspace*{1cm}
\noindent
\hfill
\begin{minipage}{0.35\linewidth}
    \begin{flushright}
        \printauthor
    \end{flushright}
\end{minipage}
%
\begin{minipage}{0.02\linewidth}
    \rule{1pt}{125pt}
\end{minipage}
\titlepagedecoration
\end{titlepage}

\lecture{1}{29 Settembre}{General introduction. Homology of a complex. Singular homology.}

\section{Introduzione}

\subsection{Richiami di geometria}

\begin{definition}
  Un \textbf{anello} è un insieme $ \R $ dotato di due operazioni $ + $ e $ \cdot $ tali che
  $ \R $ sia un gruppo abeliano con l'addizione, sia un monoide con la moltiplicazione
  (ovvero la moltiplicazione è associativa e possiede un elemento neutro) e goda della
  proprietà distributiva rispetto all'addizione.
\end{definition}

\begin{definition}
  Sia $ \R $ un anello commutativo\footnotemark[1] si definisce l' \textbf{$ \R $-modulo} il gruppo abeliano $ \M $
  equipaggiato con un'operazione di somma tale che $ \forall v,w \in \M $ e $ \forall a,b \in \R $ vale che:
  \begin{itemize}
  \item $ a(v + w) = av + aw $
  \item $ (a + b)v = av + bv $
  \item $ (ab)v = a(bv) $
  \end{itemize}
\end{definition}

\footnotetext[1]{Si può rilassare questa ipotesi e costruire moduli destri e sinistri.}

\begin{osservation}
  Se $ \R $ è un campo allora l'$ \R $-modulo è uno spazio vettoriale.
\end{osservation}

\begin{osservation}
  Ogni gruppo abeliano è uno $ \Z $-modulo in modo univoco.
\end{osservation}

\begin{definition}
  Siano $ (X, \cdot) $ e $ (Y, \star) $ due gruppi, un \textbf{omomorfismo} è un'applicazione $ f $
  tra $ X $ e $ Y $ che preserva la struttura di gruppo, cioè:
  \[
    \forall u,v \in X \quad f(u \cdot v) = f(u) \star f(v)
  \]
\end{definition}

\begin{osservation}
  Da questa definizione si trova immediatamente che gli omomorfismi si comportano bene nei
  confronti dell'inverso, cioè $ \forall v \in X $ vale che $ f(v^{-1}) = {f(v)}^{-1} $.
\end{osservation}

Voglio studiare gli omomorfismi tra $ \Z $-moduli.

\begin{definition}
  Sia $ \phi: \M \to \N $ un omomorfismo tra gli $ \R $-moduli $ \M $ e $ \N $,
  allora si definisce il \textbf{nucleo} e l'\textbf{immagine}:
  \[
    \ker (\phi) := \set{ m \in \M | \phi(m) = 0}  \qquad  \im (\phi) := \set{ m \in \M | \phi(m) = 0}
  \]
\end{definition}

\begin{osservation}
  $ \ker (\phi) $ e $ \im (\phi) $ sono $ \R $-sottomoduli.
\end{osservation}

Posso fare composizioni di omomorfismi, come:
\[
  \begin{tikzcd}
    \M_1 \arrow{r}{\phi_1} & \M_2 \arrow{r}{\phi_2} & \M_3
  \end{tikzcd}
\]
Se vale $ \phi_2 \circ \phi_1 = 0 $ allora $ \im (\phi_1) \subseteq \ker (\phi_2) $.

\begin{proof}
  Se $ u \in \im {\phi_2} $ allora $ \exists v \in \M_2 $ tale che $ \phi_1(v) = u $,
  ma $ \phi_2(u) = \phi_2(\phi_1(v)) = (\phi_2 \circ \phi_1)(v) = 0 $ quindi $ u \in \ker(\phi_2) $.
\end{proof}

\begin{definition}
  Siano $ \M $ un $ \R $-modulo e $ \N $ un suo sottomodulo, allora il \textbf{modulo
  quozionete} di $ \M $ con $ \N $ e definito da:
  \[
    \quot{\M}{\N} := \quot{\M}{\sim} \quad \text{dove} \sim \text{ è definita da: } x \sim y \Leftrightarrow x - y \in \N
  \]
\end{definition}

Siccome $ \im (\phi) $ è sottomodulo di $ \ker (\phi) $ allora posso prendere
il quoziente:
\[
  \quot{\ker (\phi_2)}{\im (\phi_1)}
\]
Questo è un sottomodulo.

A questo punto ci sono due possibilità:
\begin{enumerate}
\item $ {\ker (\phi_2)} \slash {\im (\phi_1)} = 0 $, che significa che $ \ker (\phi_2) = \im (\phi_1) $ in quanto non ci sono elementi di $ \ker (\phi_2) $ fuori da $ \im (\phi_1) $.
\item $ {\ker (\phi_2)} \slash {\im (\phi_1)} \not= 0 $, cioè $ \exists v \in \ker (\phi_2) $
  tale che $ v \not \in \im (\phi_1) $ e quindi $ \im (\phi_1) \subsetneq \ker (\phi_2) $.
\end{enumerate}
Nel primo caso si dice che la successione dei moduli $ \M $ e delle
applicazioni $ \phi $ è \textbf{esatta} in $ \M_2$, nel secondo caso la
successione è detta \textbf{complesso di moduli}.

Sostanzialmente il modulo quoziente quantifica la non esattazza nel punto $ \M_2 $
della successione.

\begin{definition}
  $ H(\M_\bullet) = {\ker (\phi_2)} \slash {\im (\phi_1)} $ è detto \textbf{modulo di omologia}
  del complesso $ M_\bullet = M_1 \longrightarrow M_2 \longrightarrow M_3 $ con le applicazioni $ \phi_1 $ e $ \phi_2 $.
\end{definition}
Per questo  $ H(\M_\bullet) $ quantifica quanto il complesso $ \M_\cdot $ non è esatto.

Questo deriva da un problema topologico concreto.

\begin{definition}
  La coppia $ (X, \mathcal{T}) $ è detta spazio topologico (generalmente si omette la $ \mathcal{T} $)
  se $ \mathcal{T} $ è una topologia, cioè se è una collezione di insiemi di $ X $ tali che:
  \begin{enumerate}
  \item $ \emptyset, X \in \mathcal{T} $
  \item $ \bigcup_{n \in \mathbb{N}} A_n \in \mathcal{T} $ se $ A_n \in \mathcal{T} \; \forall n \in \mathbb{N} $
  \item $ A \cap B \in \mathcal{T} $ se $ A,B \in \mathcal{T} $
  \end{enumerate}
\end{definition}

\subsection{Omomorfismo tra $ \RN{} $ e $ \RN{N} $}
È noto che $ \RN{} \not \simeq \RN{N} $ per $ n \geq 2 $, infatti basta
che tolgo un punto a $ \RN{} $ che diventa sconnesso mentre $ \RN{N} $
rimane connesso anche togliendogli un punto.

Tuttavia vale anche che $ \RN{2} \not \simeq \RN{N} $ per $ n \geq 3 $, infatti:

\begin{proof}
  Per assurdo $ f : \RN{2} \overset{\to}{\sim} \RN{N} $ è un omomorfismo con
  $ n \geq 3 $, tolgo un punto da $ \RN{2} $:
  \[
    \forall p \in \RN{2} \quad f:\RN{2} \setminus \set{p} \overset{\to}{\sim} \RN{N} \setminus \set{f(p)}
  \]
  Ma: $ \RN{2} \setminus \set{p} \simeq \RN{} \times \mathcal{S}^1 $ con la mappa
  che manda $ \underline{x} \mapsto \left( || \underline{x} ||, \frac{\underline{x}}{|| \underline{x} ||} \right) $. Ma quindi il gruppo fondamentale deve essere isomorfo:
  $ \pi_1 (\RN{} \times \mathcal{S}^{1}) \simeq \pi_1(\RN{}\times \mathcal{S}^{n-1}) $ ma $ \pi_1 (\RN{}\times \mathcal{S}^{1}) = \Z $ e $ \pi_1(\RN{}\times \mathcal{S}^{n-1}) = \set{1} $ quindi non possono essere isomorfi.
\end{proof}
Ho quindi dedotto proprietà geometriche a partire da considerazioni algebriche.

\begin{definition}
  Si definisce il \textbf{gruppo fondamentale} di uno spazio topologico $ X $
  connesso per archi attorno al punto $ x_0 \in X $
  \[
    \pi_1 (X, x_0) = \quot{\set{ g: \mathcal{S}^1 \to X | g \text{ continua}, g(1) = x_0}}{\sim}
  \]
  e $ \sim $ è la relazione di omotopia: $ g_1 \sim g_2 $ se $ \exists G: \mathcal{S}^1 \times I \to X  $ tale che
  $ G(z,0) = g_1(z), G(z,1) = g_2(z), G(1,t) = x_o $.
\end{definition}

Ora voglio mostrare per assurdo che non esiste omomorfismo tra $ \RN{3} $ e $ \RN{N} $.

\begin{proof}
  Come nel caso precedente tolgo $ q $ da $ \RN{3} $ e $ f(q) $ da $ \RN{3} $, quindi ottengo
  l'omomorfismo tra $ \RN{} \times \Sph{2} \simeq \RN{} \times \Sph{n-1} $, ma i gruppi fondamentali
  associati sono banali, quindi sono isomorfi, e non è posisbile replicare il ragionamento utilizzato sopra.
\end{proof}

Poincaré introdusse i gruppi di omotopia superiore.

\begin{definition}
  Si definiscono i \textbf{gruppi di omotopia superiore} di uno spazio topologico $ X $
  attorno al punto $ x_0 $ per $ k \geq 2 $:
  \[
    \pi_k(X) (X, x_0) = \quot{\set{ g: \Sph{k} \to X | g(p_0) = x_0, p_o \in \Sph{k}}}{\sim}
  \]
\end{definition}
Studiare i gruppi di omotopia superiore è un problema aperto della topologia moderna.
Tuttavia si sa che:
\begin{enumerate}
\item $ \pi_k(\Sph{m}) = 1 \quad \text{per} \quad 1 \leq k < m $
\item $ \pi_m(\Sph{m}) \simeq \Z \quad \text{per} \quad k = m $
\item $ \pi_1(\Sph{2}) = 1 $
\item $ \pi_2(\Sph{2}) \simeq 1 $
\end{enumerate}

Anche se non so calcolare i gruppi di omotopia superiore non vorrei buttarli via\dots

\subsection{Omologia}

Uso la teoria dell'omologia che mi permette di semplificare i problemi.
Ci sono varie possibilità:
\begin{itemize}
  \item Omologia simpliciale
  \item Omologia cellulare
  \item Omologia singolare
  \item Omologia persistente\footnote{Questa ha numerose applicazioni pratiche.}
\end{itemize}
Ma cosa è l'omologia?

\begin{definition}
  In $ \RN{k+1} $ si definisce il \textbf{simplesso standard} $ \Delta_k $ l'insieme:
  \[
    \Delta_k = \set{(x_1,x_2,\dots) \in \RN{k+1} | \forall i \quad 0 \leq x_i \leq 1 \text{ e } \sum_{i=1}^{k+1}x_i = 1}
  \]
\end{definition}

\begin{osservation} Alcuni esempi sono:
  \begin{itemize}
  \item $ \Delta_0 $ è un punto.
  \item $ \Delta_1 $ è un segmento omeomorfo a $ [0,1] $.
    [FIGURA]
  \end{itemize}
\end{osservation}

\begin{definition}
  Dato uno spazio topologico $ X $ si definisce il \textbf{$ k $-simplesso singolare}
  in $ X $ come un'applicazione continua $ g: \Delta_k \to X $.
\end{definition}
Spesso conviene identificare il $ k $-simplesso con la sua immagine in $ X $.
In quesot modo uno $ 0 $-simplesso è un punto in $ X $, mentre un $ 1 $-simplesso singolare potrebbe essere sia un segmento che un punto (se la mappa è costante). Siccome
il simplesso deforma è detto singolare.

Voglio costruire un complesso di gruppi abeliani e definire l'omologia singolare come l'omologia di tale complesso.

$ S_\cdot $ è il compesso, cioè: $ \dots \to S_{k+1}(X) \to S_k() \to S_{k-1} \to \dots \to S_0(X) $, dove
\begin{align*}
  S_k(X) ={}& \{\text{combinazioni lineari finite a coefficienti interi: } \\
         & \sum_g n_g g \;|\; n_g \in \Z, g \; k-\text{simplessi singolari di } X \}
\end{align*}
$ S_k(X) $ è un gruppo abeliano con l'operazione somma definita naturalmente:
\[
  \sum_g n_g g + \sum_h n_h =   \sum_g n_g g + \sum_g n_g^\star = \sum_g (n_g + n_g^\star)g
\]
Ad esempio:
\[
  (n_1 g_1 + n_2 g_2 + 2 n_3 g_3) + (m_1 g_1 + m_4 g_4) = (n_1 + m_1)g_1 + n_2 g_2 + 2 n_3 g_3 + m_4 g_4
\]
Questa è una somma con tutte le giuste proprietà. Lo zero è la catena con tutti i coefficienti nulli,
mentre l'inverso è la catena con i coefficienti opposti.
Queste catene sono chiamate $ k $-\textbf{catene singolari}.

Ad esempio:
Se $ k = 0 $ $ S_0(X) $ sono catene di punti ($ g_0 : \Delta_0 \to X $)
\[
  S_0(X) = \set { \sum n_i p_i | n_i \in \Z, \; p_i \in X}
\]
Ora devo introdurre le applicazioni tra i vari $ S_k $, queste applicazioni saranno il bordo.

Definisco $ h: \Delta_1 \to X $ in modo tale che $ h(\Delta) = \alpha $ dove $ \alpha $ è un arco, ovvero una funzione
da un intervallo $ I = [0,1] $ a $ X $ tale che $ \alpha(0) = x_0 $ e $ y_0 $.

[FIGURA]

Posso ottenere una $ 0 $-catena prendendo i punti estremi dell'arco.

\begin{definition}
  Sia $ \Delta_k $ un $ k $-simplesso standard con $ k \geq 0 $ si definisce l'operatore \textbf{faccia} come
  la mappa $ F_i^{\;k} $ da $ \Delta_{k-1} $ a $ \Delta_k $ tale che $ F_i^{\;k}(\Delta_{k-1}) $ è una faccia di $ \Delta_k $.
\end{definition}

Ad esempio per $ k = 2 $ $ \Delta_2 = \set{ (x_1,x_2,x_3) \in \RN{3} | x_1 + x_2 + x_3 = 1, \; 0 \leq x_i \leq 1 \; \forall i} $,
si definisce la base $ e_0 = (1,0,0) \; e_1 = (0,1,0) \; e_2 = (0,0,1) $, voglio vedere il bordo del triangolo
come facce.
[FIGURA, CONSIDERAZIONI]

\begin{exercise}
  Dimostrare che se $ [\cdot, \cdot] $ indica l'inviluppo convesso allora:
  \begin{enumerate}
  \item Per $ j > i $ vale che $ F_i^{\; k+1} \circ F_i^{\; k} = [e_0, \dots, e_i, \dots, e_j, \dots, e_k ] $.
  \item Per $ j \leq i $ vale che $ F_i^{\; k+1} \circ F_i^{\; k} = [e_0, \dots, e_j, \dots, e_{i+1}, \dots, e_k ] $.
  \end{enumerate}
\end{exercise}

Dato un $ k $-simplesso singolare $ \sigma: \Delta_k \to X $ si definisce la mappa $ \sigma^{(i)} = \sigma \circ F_i^{\; k} $.

[FIGURA]

\begin{definition}
  Si definisce il \textbf{bordo} di un $ k $-simplesso singolare come $ \partial_k \sigma = \sum_{i=0}^{k}(-)^i \sigma^{(i)} $.
\end{definition}

Per $ k = 1 $ $ \partial_1 \sigma = p_1 - p_0 $ infatti $ \sigma^{0} = \sigma \circ F_0^{\; 1} = \sigma(1) = p_1 $ e $ \sigma^{0} = \sigma \circ F_1^{\; 1} = \sigma(0) = p_0 $.\footnote{Tecnicamente si intende $ p_0 = \partial_1 \sigma^{(0)}(1) $ e $ p_1 = \partial_0 \sigma^{(1)}(1) $.}

Allora definisco $ \partial_k: S_k(X) \to S_{k-1}(X) $ infatti per linearità $ \partial_k \left( \sum_g n_g g\right) = \sum_g n_g \partial_k g $.

Devo mostrare che $ \partial_k $ è un omomorfismo.

\begin{proof}
  \begin{gather*}
  \partial_k \left( \sum_g n_g g + \sum_g m_g g\right) = \partial_k \left( \sum_g(m_g + n_g)g \right) = \sum_g (m_g + n_g) \partial_k g = \\
  = \sum_g n_g \partial_k g + \sum_g m_g \partial_k g = \partial_k \left( \sum_g n_g g\right) + \partial_k \left( \sum_g m_g g \right)
\end{gather*}
\end{proof}

Quindi il complesso è costituito da:
\[
  \begin{tikzcd}
   \dots \arrow{r}{\partial_{k+1}} & S_k(X) \arrow{r}{\partial_k} & S_{k-1}(X) \arrow{r}{\partial_{k-1}} & \dots
  \end{tikzcd}
\]

Devo verificare che $ \partial_k \circ \partial_{k+1} = $. Spesso come notazione si pone $ \partial^2 = 0 $.

\begin{proof}
  Se $ \sigma $ è un $ k $-complesso singolare $ \sigma : \Delta_k \to X $:
  \begin{align*}
    & \partial_k \circ \partial_{k+1} \sigma = \partial_k \left( \sum_{j=0}^{k+1}(-)^j (\sigma \circ F_j^{\; k+1}) \right) =  \sum_{j=0}^{k+1}(-)^j \partial_k (\sigma \circ F_j^{\; k+1}) &= \\
                    & = \sum_{j=0}^{k+1} (-)^j \sum_{i=0}^k (-)^i (\sigma \circ F_j^{\; k+1}) \circ F_i^{\; k} = \sum_{j = 0}^{k+1} \sum_{i = 0}^{k} (-)^{j+i} \sigma \circ F_j^{\; k+1} \circ F_{j}^{\; k} &= \\
                    & = \sum_{0 \leq i < j \leq k + 1} (-)^{i+j} \sigma \circ F_j^{\; k+1} \circ F_i^{\; k} + \sum_{0 \leq j < i \leq k} (-)^{i+j} \sigma \circ F_j^{\; k+1} \circ F_i^k &= \\
    & = \sum_{0 \leq i < j \leq k + 1} (-)^{i+j} \sigma \circ F_j^{\; k+1} \circ F_i^{\; k} + \sum_{0 \leq j < i \leq k} (-)^{i+j} \sigma \circ F_{i+1}^{\; k+1} \circ F_j^{\; k} & =  \\
    & = 0
  \end{align*}
\end{proof}

\lecture{2}{4 Ottobre}{Banane}
\lecture{3}{6 Ottobre}{Banane}

\subsection{Richiami sul gruppo fondamentale}
\begin{definition}
  Sia $ X $ uno spazio topologico e $ x_0 \in X $ un suo punto, allora la coppia $ (X, x_0) $ è detta spazio topologico puntato.
\end{definition}

\begin{definition}
  Sia $ (X, x_0) $ uno spazio topologico puntato e $ f: I \to X $ una mappa continua tale che $ f(0) = f(1) = x_0 \; \forall t \in I $,
  si dice che una funzione continua $ g $ è \textbf{omotopicamente equivalente} a $ f $ ($ g \sim_H f $) se esiste una funzione
  continua $ F: I \times I \to X $ tale che;
  \begin{itemize}
  \item $ F(0,x) = f(x) \; \forall x \in I $
  \item $ F(1,x) = g(x) \; \forall x \in I$
  \item $ F(s,0) = x_0 \; \forall s \in I $
  \item $ F(s,1) = x_0 \; \forall s \in I $
  \end{itemize}
  La relazione $ \sim_H $ è detta relazione di omotopia e si dimostra essere una relazione di equivalenza.
\end{definition}
[FIGURA]

Si definisce l'insieme;
\[
  \pi_1(X,x_0) = \quot{\set{f:I \to X | f \text{ continua}, f(0) = f(1) = x_0}}{\sim_H}
\]

\end{document}


%%% Local Variables:
%%% mode: latex
%%% TeX-master: t
%%% End:
