\documentclass[10pt, twoside=false, x11names]{scrbook}

\usepackage{amsmath}
\usepackage{amssymb}
\usepackage{amsfonts}
\usepackage{graphicx}
% \usepackage{lmodern}
\usepackage{libertine}
\usepackage{tikz}
\usepackage{epigraph}
\usepackage{lipsum}
\usepackage[utf8]{inputenc}
\usepackage{braket}
\usepackage[italian]{babel}
\usepackage{tikz-cd}
\usepackage{makeidx}
\usepackage{tikz}
\usepackage{mathtools}
\usepackage{supertabular}
\usepackage{array}
\usepackage{textcomp}
\usepackage[T1]{fontenc}
\usepackage{stmaryrd}
\usepackage{subcaption}
\usepackage[hidelinks]{hyperref}

\renewcommand\epigraphflush{flushright}
\renewcommand\epigraphsize{\normalsize}
\setlength\epigraphwidth{0.7\textwidth}

\definecolor{titlepagecolor}{cmyk}{1,.60,0,.40}

\DeclareFixedFont{\titlefont}{T1}{ppl}{b}{it}{0.5in}

\makeatletter
\def\printauthor{%
    {\large \@author}}
\makeatother
\author{%
  Professore: \\
  \emph{Gilberto Bini} \vspace{10pt}\\
  Umile scriba: \\
  \emph{Gabriele Bozzola}
    }

\newcommand\titlepagedecoration{%
\begin{tikzpicture}[remember picture,overlay,shorten >= -10pt]

\coordinate (aux1) at ([yshift=-15pt]current page.north east);
\coordinate (aux2) at ([yshift=-410pt]current page.north east);
\coordinate (aux3) at ([xshift=-4.5cm]current page.north east);
\coordinate (aux4) at ([yshift=-150pt]current page.north east);

\begin{scope}[titlepagecolor!40,line width=12pt,rounded corners=12pt]
\draw
  (aux1) -- coordinate (a)
  ++(225:5) --
  ++(-45:5.1) coordinate (b);
\draw[shorten <= -10pt]
  (aux3) --
  (a) --
  (aux1);
\draw[opacity=0.6,titlepagecolor,shorten <= -10pt]
  (b) --
  ++(225:2.2) --
  ++(-45:2.2);
\end{scope}
\draw[titlepagecolor,line width=8pt,rounded corners=8pt,shorten <= -10pt]
  (aux4) --
  ++(225:0.8) --
  ++(-45:0.8);
\begin{scope}[titlepagecolor!70,line width=6pt,rounded corners=8pt]
\draw[shorten <= -10pt]
  (aux2) --
  ++(225:3) coordinate[pos=0.45] (c) --
  ++(-45:3.1);
\draw
  (aux2) --
  (c) --
  ++(135:2.5) --
  ++(45:2.5) --
  ++(-45:2.5) coordinate[pos=0.3] (d);
\draw
  (d) -- +(45:1);
\end{scope}
\end{tikzpicture}%
}

\newcounter{lecnum}

\newcommand{\lecture}[3]{
   \pagestyle{myheadings}
   \thispagestyle{plain}
   \newpage
   \setcounter{lecnum}{#1}
   \noindent
   \begin{center}
   \framebox{
      \vbox{\vspace{2mm}
    \hbox to \linewidth { {\bf Topologia Algebrica
        \hfill 2016/2017} }
       \vspace{4mm}
       \hbox to \linewidth  { {\Large \hfill Lezione #1: #2  \hfill} }
       \vspace{2mm}
       \hbox { {Agomenti: #3  \hfill} }
      \vspace{2mm}}
   }
   \end{center}
   \markboth{Lezione #1: #2}{Lezione #1: #2}
}
%%% Local Variables:
%%% mode: latex
%%% TeX-master: "notes"
%%% End:


\newtheorem{theorem}{Teorema}[section]
\newtheorem{lemma}[theorem]{Lemma}
\newtheorem{proposition}[theorem]{Proposizione}
\newtheorem{osservation}[theorem]{Osservazione}
\newtheorem{corollary}[theorem]{Corollario}
\newtheorem{definition}[theorem]{Definizione}
\newtheorem{example}[theorem]{Esempio}
\newcounter{exercises}
\newtheorem{exercise}[exercises]{Esercizio}
\newenvironment{proof}{{\textbf{Dimostrazione}:}}{\hfill $\square$}

\newcommand{\R}{\mathcal{R}}
\newcommand{\M}{\mathcal{M}}
\newcommand{\N}{\mathcal{N}}
\newcommand{\Z}{\mathbb{Z}}
\newcommand{\me}{\mathrm{e}}
\newcommand{\im}[1]{\mathrm{Im}( #1 )}
\renewcommand{\ker}[1]{\mathrm{Ker}( #1)}
\renewcommand{\phi}{\varphi}
\newcommand{\RN}[1][]{\mathbb{R}^#1}
\newcommand{\Id}[1][]{\mathbb{I}_#1}
\newcommand{\Sph}[1][]{\mathcal{S}^#1}
\newcommand{\Disk}[1][]{\mathcal{D}^#1}
\newcommand{\homoto}{\xrightarrow{\,\smash{\raisebox{-0.65ex}{\ensuremath{\scriptstyle\sim}}}\,}}
\newcommand{\Ab}[1]{\mathrm{Ab}\left( #1 \right)}
\newcommand{\tr}{\mathrm{tr}}
\newcommand{\incl}{\xhookrightarrow{}}
\newcommand{\invamalg}{\mathbin{\text{\rotatebox[origin=c]{180}{$\amalg$}}}}

\newcommand{\vedi}[1]{\emph{vedi} #1}

\newcommand*\quot[2]{{^{\textstyle #1}\big/_{\textstyle #2}}}

\renewcommand\labelitemi{\tiny$\bullet$}

\makeindex

\makeatletter
\usepackage{auxhook}
\AddLineBeginAux{%
  \string\providecommand\string\new@aux@symb[3]{}%
}
\newcommand*{\symb@list}{}
\newlength{\symb@maxwidth}
\newlength{\symb@meaning}
\newcommand*{\new@aux@symb}[3]{%
  \g@addto@macro{\symb@list}{\symb@do{#1}{#2}{#3}}%
  \@newl@bel{SYMB}{#1}{#2}%
  \begingroup
    \settowidth{\dimen@}{#2}%
    \ifdim\dimen@>\symb@maxwidth
      \global\symb@maxwidth=\dimen@
    \fi
  \endgroup
}
\newcommand*{\new@symb}[3]{%
  \@bsphack
  \if@filesw
    \protected@write\@auxout{}{%
      \string\new@aux@symb{#1}{%
        \detokenize\expandafter{\unexpanded{#2}}%
      }{%
        \detokenize\expandafter{\unexpanded{#3}}%
      }%
    }%
  \fi
  \label{symb:#1}%
%  \@ifundefined{SYMB@#1}{%
%    \expandafter\gdef\csname SYMB@#1\endcsname{#2}%
%  }{}%
  \@esphack
}
\newcommand*{\newmathsymb}[3]{%
  \new@symb{#1}{\ensuremath{#2}}{#3}%
}
\newcommand*{\newtextsymb}{%
  \@dblarg\symb@newtext
}
\def\symb@newtext[#1]#2#3{%
  \new@symb{#1}{#2}{#3}%
}
\newcommand*{\symb}[1]{%
  \@ifundefined{SYMB@#1}{%
    \@latex@warning{Symbol `#1' is undefined}%
    \nfss@text{\textbf{??}}%
  }{%
    \csname SYMB@#1\endcsname
  }%
}
% List of symbols
\newcommand*{\symb@head}[1]{\textbf{\large#1}}
\newcommand*{\printsymblist}{%
  \twocolumn[%
    \section*{%
      \centering
      Lista dei simboli e abbreviazioni %
    }%
  ]%
  \thispagestyle{empty}% optional
  \renewcommand*{\arraystretch}{1.1}%
  \settowidth{\dimen@}{\symb@head{Symbol}}%
  \ifdim\dimen@>\symb@maxwidth
    \global\symb@maxwidth\dimen@
  \fi
  \setlength{\symb@meaning}{\linewidth}%
  \addtolength{\symb@meaning}{-\symb@maxwidth}%
  \settowidth{\dimen@}{\symb@head{Page}}%
  \addtolength{\symb@meaning}{-\dimen@}%
  \addtolength{\symb@meaning}{-4\tabcolsep}%
  \tablehead{%
    \symb@head{Simbolo} & \symb@head{Significato} & \symb@head{Pag.}\\[.5ex]%
  }%
  \begin{supertabular}{@{}p{\symb@maxwidth}p{\symb@meaning}c@{}}%
    \symb@list
  \end{supertabular}%
  \clearpage
  \onecolumn
}
\newcommand*{\symb@do}[3]{%
  #2&%
  \sbox0{%
    \renewcommand*{\arraystretch}{1}%
    \begin{tabular}[t]{@{}p{\symb@meaning}@{}}%
      \raggedright
      #3%
    \end{tabular}%
  }%
  \sbox2{%
    \renewcommand*{\arraystretch}{1}%
    \begin{tabular}[b]{@{}b{\symb@meaning}@{}}%
      \raggedright
      #3%
    \end{tabular}%
  }%
  \usebox0 %
  \xdef\symb@raise{\the\dimexpr\ht0-\ht2}%
  &\raisebox{\symb@raise}{\pageref{symb:#1}}\tabularnewline
}
\makeatother


%%% Local Variables:
%%% mode: latex
%%% TeX-master: "notes"
%%% End:


\graphicspath{{./images/}}

\begin{document}

\begin{titlepage}

\noindent
\titlefont Topologia Algebrica
\epigraph{Zitto e studia.}%
{\textit{Parigi 1905}\\ \textsc{H.\ Poincarè}}
\null\vfill
\vspace*{1cm}
\noindent
\hfill
\begin{minipage}{0.35\linewidth}
    \begin{flushright}
        \printauthor
    \end{flushright}
\end{minipage}
%
\begin{minipage}{0.02\linewidth}
    \rule{1pt}{125pt}
\end{minipage}
\titlepagedecoration
\end{titlepage}

\tableofcontents
\newmathsymb{N}{\mathbb{N}}{Numeri naturali}
\newmathsymb{Z}{\mathbb{Z}}{Numeri interi}
\printsymblist

% lezione 1
%  _     _____ ________ ___  _   _ _____   _
% | |   | ____|__  /_ _/ _ \| \ | | ____| / |
% | |   |  _|   / / | | | | |  \| |  _|   | |
% | |___| |___ / /_ | | |_| | |\  | |___  | |
% |_____|_____/____|___\___/|_| \_|_____| |_|

\chapter{Richimi di algebra e geometria}
\section{Richiami di algebra}

\newmathsymb{R}{\R}{Anello}
\begin{definition}
  Un \textbf{anello} \index{Anello} è un insieme $ \R $ dotato di due operazioni $ + $ e $ \cdot $ tali che
  $ \R $ sia un gruppo abeliano con l'addizione, sia un monoide con la moltiplicazione
  (ovvero la moltiplicazione è associativa e possiede un elemento neutro\footnote{La richiesta
    di esistenza dell'elemento neutro, cioè dell'unità non è comune a tutti gli autori,
    chi non la richiede chiama anello unitario \index{Anello unitario} la presente
    definizione di anello.}) e goda della proprietà distributiva rispetto all'addizione.
\end{definition}

\begin{definition}
  Un anello si dice \textbf{anello commutativo} \index{Anello commutativo} se l'operazione di moltiplicazione
  è commutativa.
\end{definition}

\begin{definition}
  Un \textbf{campo} \index{Campo} è un anello commutativo in cui ogni elemento non nullo ammette
  un inverso moltiplicativo.
\end{definition}

\begin{definition}
  Sia $ \R $ un anello commutativo si definisce l' \textbf{$ \R $-modulo} \index{$ \R $-modulo}
  un gruppo abeliano $ \M $ equipaggiato con un'operazione di moltiplicazione per uno scalare in $ \R $
  tale che $ \forall v,w \in \M $ e $ \forall a,b \in \R $ vale che:
  \begin{itemize}
  \item $ a(v + w) = av + aw $
  \item $ (a + b)v = av + bv $
  \item $ (ab)v = a(bv) $
  \end{itemize}
\end{definition}

\begin{osservation}
  Se $ \R $ è un campo allora l'$ \R $-modulo è uno spazio vettoriale.
\end{osservation}
Sostanzialmente la nozione di $ \R $-modulo generalizza agli anelli il concetto di spazio vettoriale sui campi.

\begin{osservation}
  Ogni gruppo abeliano $ \mathcal{G} $ è uno $ \Z $-modulo in modo univoco, cioè $ \mathcal{G} $ è un
  gruppo abeliano se e solo e è uno $ \Z $-modulo.
\end{osservation}
\begin{proof}
  Sia $ x \in \mathcal{G} $ si definisce l'applicazione di moltiplicazione per un elemento $ n \in \Z $ come
  \[
    nx =
    \begin{cases}
      \underbrace{ x + x + x + \dots}_{n \text{ volte}} & \text{se } n > 0 \\
      0 & \text{se } n = 0 \\
       \underbrace{ - x - x - x - \dots}_{|n| \text{ volte}} & \text{se } n < 0 \\
    \end{cases}
  \]
  Si verifica banalmente che questa operazione è ben definita e soddisfa
  le giuste proprietà perché la coppia $ (\mathcal{G}, \Z) $ sia uno $ \Z $-modulo.
  A questo punto non è possibile costruire applicazioni diverse che soddisfino le
  proprietà richieste infatti utilizzando la struttura di anello di $ \Z $:
  $ n x = (1 + 1 + 1 + 1 + \dots) x = x + x + x \dots $, quindi quella definita
  è l'unica possibile.
\end{proof}

\newmathsymb{groupgen}{<\dots>}{Gruppo generato}
\begin{definition}
  Un gruppo $ \mathcal{G} $ si dice \textbf{generato}\index{Gruppo generato}
  dai suoi elementi $ x_1, x_2, \dots \in \mathcal{G} $ se
  ogni suo elemento si può scrivere come combinazione lineare a elementi interi di $ x_1, x_2, \dots $.
  In questo caso si indica $ \mathcal{G} = \langle \{ x_1, x_2, \dots \} \rangle $.
\end{definition}

\begin{definition}
  Un gruppo abeliano si dice \textbf{libero}\index{$ \Z $-modulo libero} se è generato
  da un numero finito di elementi linearmente indipendenti, il numero di tali elementi
  definisce il \textbf{rango}\index{Rango di gruppo abeliano} del gruppo.
\end{definition}

\begin{definition}
  Siano $ (X, \cdot) $ e $ (Y, \star) $ due gruppi, un \textbf{omomorfismo} \index{Omomorfismo} è un'applicazione $ f $
  tra $ X $ e $ Y $ che preserva la struttura di gruppo, cioè tale che:
  \[
    \forall u,v \in X \quad f(u \cdot v) = f(u) \star f(v)
  \]
\end{definition}

\begin{osservation}
  Da questa definizione si trova immediatamente che gli omomorfismi si comportano bene nei
  confronti dell'inverso, cioè $ \forall v \in X $ vale che $ f(v^{-1}) = {f(v)}^{-1} $.
\end{osservation}

Voglio studiare gli omomorfismi tra $ \Z $-moduli.

\newmathsymb{ker}{\ker{f}}{Nucleo di $f$}
\newmathsymb{im}{\im{f}}{Immagine $ f$}
\begin{definition}
  Sia $ \phi: \M \to \N $ un omomorfismo tra gli $ \R $-moduli $ \M $ e $ \N $,
  allora si definisce il \textbf{nucleo} \index{Nucleo} e l'\textbf{immagine} \index{Immagine}:
  \[
    \ker {\phi} := \set{ m \in \M | \phi(m) = 0}  \qquad  \im{\phi} := \set{ m \in \N | \exists k \in M \text{ con } m = \phi(k)}
  \]
\end{definition}

\begin{osservation}
  $ \ker{\phi} $ e $ \im{\phi} $ sono $ \R $-sottomoduli, cioè sono sottoinsiemi di $ \M $ e $ \N $
  che posseggono la struttura di $ \R $-modulo.
\end{osservation}

Se $ M_i $ sono $ \R $-moduli posso fare composizioni di omomorfismi, come:
\[
  \begin{tikzcd}
    \M_1 \arrow{r}{\phi_1} & \M_2 \arrow{r}{\phi_2} & \M_3
  \end{tikzcd}
  \text{o equivalentemente}
  \begin{tikzcd}
    \M_1 \arrow{r}{\phi_2 \circ \phi_1} & \M_3
  \end{tikzcd}
\]
\begin{proposition}
Se vale $ \phi_2 \circ \phi_1 = 0 $ allora $ \im{\phi_1} \subseteq \ker{\phi_2} $.
\end{proposition}
\begin{proof}
  Se $ u \in \im {\phi_1} $ allora $ \exists v \in \M_2 $ tale che $ \phi_1(v) = u $,
  ma $ \phi_2(u) = \phi_2(\phi_1(v)) = (\phi_2 \circ \phi_1)(v) = 0 $ per ipotesi, quindi $ u \in \ker{\phi_2} $.
\end{proof}

Mi interessano questi morfismi perché hanno un preciso significato geometrico che
sarà chiaro successivamente.

\begin{definition}
  Siano $ \M $ un $ \R $-modulo e $ \N $ un suo sottomodulo, allora il \textbf{modulo
  quoziente} \index{Modulo quoziente} di $ \M $ con $ \N $ e definito da:
  \[
    \quot{\M}{\N} := \quot{\M}{\sim} \quad \text{dove } \sim \text{ è definita da: } x \sim y \Leftrightarrow x - y \in \N
  \]
  Dove $ \quot{\M}{\sim} $ è l'insieme delle classi di equivalenza di $ \sim $ equipaggiate
  con operazioni indotte dall'$ \R $-modulo, cioè se $ [u], [w] \in \quot{\M}{\sim} $ e $ a \in \R $:
  \begin{itemize}
  \item $ [u] + [w] = [u + w] $
  \item $ a [u] = [au] $
  \end{itemize}
  In questo caso gli elementi di $ \quot{\M}{\N} $ sono le classi di equivalenza
  $ [m] = \set{ m + n | n \in \N } $.
\end{definition}

Siccome $ \im{\phi} $ è sottomodulo di $ \ker{\phi} $ allora posso prendere
il quoziente:
\[
  \quot{\ker{\phi_2}}{\im{\phi_1}}
\]
Questo è un sottomodulo. Si nota che questo è sensato solo se si impone la condizione
$ \phi_2 \circ \phi_1 = 0 $, altrimenti non c'è l'inclusione e quindi non è possibile fare l'operazione
di quoziente.

A questo punto ci sono due possibilità:
\begin{enumerate}
\item $ {\ker {\phi_2}} \slash {\im{\phi_1}} = 0 $, che significa che $ \ker {\phi_2} = \im{\phi_1} $
  in quanto non ci sono elementi di $ \ker {\phi_2} $ fuori da $ \im{\phi_1} $, dato che l'unica
  classe di equivalenza presente è $ [0] $ significa che $ \forall m \in \ker{\phi_1} \; \exists n \in \im{\phi_2} $
  tale che $ m - n = 0 $, cioè $ m $ e $ n $ coincidono e quindi $ \ker {\phi_2} = \im{\phi_1} $.
\item $ {\ker {\phi_2}} \slash {\im{\phi_1}} \not= 0 $, cioè $ \exists v \in \ker {\phi_2} $
  tale che $ v \not \in \im {\phi_1} $ e quindi $ \im {\phi_1} \subsetneq \ker {\phi_2}$.
\end{enumerate}
Nel primo caso si dice che la successione dei moduli $ \M $ e delle
applicazioni $ \phi $ è \textbf{esatta}\index{Complesso di moduli esatto} in $ \M_2$, nel secondo caso la
successione è detta \textbf{complesso di moduli}\index{Complesso di moduli}.

Sostanzialmente il modulo quoziente quantifica la non esattezza nel punto $ \M_2 $
della successione.

\begin{definition}
  $ H(\M_\bullet) = {\ker {\phi_2}} \slash {\im {\phi_1}} $ è detto \textbf{modulo di omologia} \index{Modulo di omologia}
  del complesso $ M_\bullet = M_1 \longrightarrow M_2 \longrightarrow M_3 $ con le applicazioni $ \phi_1 $ e $ \phi_2 $.
\end{definition}
Per questo  $ H(\M_\bullet) $ quantifica quanto il complesso $ \M_\bullet $ non è esatto.

Questo deriva da un problema topologico concreto.

\newmathsymb{topsp}{X}{Spazio topologico}
\begin{definition}
  La coppia $ (X, \mathcal{T}) $ è detta \textbf{spazio topologico}\index{Spazio topologico}
  (generalmente si omette la $ \mathcal{T} $)
  se $ \mathcal{T} $ è una \textbf{topologia}\index{Topologia}, cioè se è una collezione di insiemi di $ X $ tali che:
  \begin{enumerate}
  \item $ \emptyset, X \in \mathcal{T} $
  \item $ \bigcup_{n \in \mathbb{N}} A_n \in \mathcal{T} $ se $ A_n \in \mathcal{T} \; \forall n \in \mathbb{N} $
  \item $ \bigcap_{n \in \set{0,1,\dots,N} } A_n \in \mathcal{T} $ se $ A_n \in \mathcal{T} \; \forall n \in \set{0,1,\dots,N} $
  \end{enumerate}
  Gli elementi di $ \mathcal{T} $ sono detti \textbf{aperti}\index{Insiemi aperti}.
\end{definition}
\begin{osservation}
  Se $ \tau $ è la collezione di tutti i sottoinsiemi di $ X $ allora le proprietà sono automaticamente
  verificate e questa è la \textbf{topologia discreta}\index{Topologia discreta}, invece
  $ \tau = \set{\emptyset, X} $ è una topologia ed è la \textbf{topologia triviale}.
  Infine in $ \RN{n} $ si definisce la \textbf{topologia usuale} che è la topologia in cui gli aperti
  sono iperintervalli aperti del tipo $ (a_1,b_1) \times (a_2, b_2) \times (a_3, b_3) \dots \times (a_n, b_n) $.
  Si dimostra che se si ammettono intersezioni infinite allora la topologia usuale coincide con la topologia
  triviale in $ \RN{n} $.
\end{osservation}

\begin{osservation}
  Uno spazio metrico si può rendere topologico definendo gli insiemi aperti come gli intorni sferici aperti.
\end{osservation}

\begin{osservation}
  Sia $ A \subseteq X $ spazio topologico, si può rendere anche $ A $ uno spazio topologico equipaggiandolo con la
  \textbf{topologia indotta}\index{Topologia indotta} in cui gli aperti sono gli aperti di $ X $ intersecati
  con $ A $.
\end{osservation}

\begin{osservation}
  Uno spazio topologico è \textbf{connesso}\index{Spazio connesso} se si può scrivere come
  unione disgiunta di due suoi aperti.
\end{osservation}

\begin{definition}
  Sia $ X $ uno spazio topologico l'insieme $ \set{A_i | A_i \in X \; \forall i} $ è un \textbf{ricoprimento}\index{Ricoprimento}
  di $ X $ se:
  \[
    \bigcup_{i} A_i = X
  \]
  Se in particolare gli insiemi $ A_i $ sono aperti il ricoprimento è detto \textbf{ricoprimento aperto}.
\end{definition}

\begin{definition}
  Un insieme $ U $ è detto \textbf{compatto}\index{Insieme compatto} se per ogni suo possibile ricoprimento
  aperto ne esiste un sottoinsieme che è un ricoprimento \emph{finito} di $ U $.
\end{definition}

\newmathsymb{homo}{\simeq}{Spazi omeomorfi}
\begin{definition}
  Una mappa tra spazi topologici è detta \textbf{omeomorfismo}\index{Omeomorfismo} se è continua
  e ammette inverso continuo, cioè se è una mappa uno a uno. Se due spazi sono omeomorfi si utilizza
  il simbolo $ \simeq $.
\end{definition}
Siccome gli omeomorfismi sono mappe uno a uno due spazi omeomorfi sono essenzialmente identici. La
relazione di omeomorfismo costituisce una relazione di equivalenza. Molti
degli strumenti sviluppati in questo corso servono a capire se due spazi sono omeomorfi o meno.

% lezione 3
%  _     _____ ________ ___  _   _ _____   _____
% | |   | ____|__  /_ _/ _ \| \ | | ____| |___ /
% | |   |  _|   / / | | | | |  \| |  _|     |_ \
% | |___| |___ / /_ | | |_| | |\  | |___   ___) |
% |_____|_____/____|___\___/|_| \_|_____| |____/


\section{Richiami sul gruppo fondamentale}

\begin{definition}
  Sia $ X $ uno spazio topologico e $ x_0 $ un suo punto, allora un \textbf{laccio}\index{Laccio} è un arco in $ X $
  avente come punto di partenza e punto di arrivo il punto $ x_0 $. Un laccio $ c_{x_0} $ si dice \textbf{costante} se $ \forall t \in I $
  $ c_{x_0}(t) = x_0 $ con $ x_0 \in X $.
\end{definition}

Vorrei strutturare l'insieme dei lacci in uno spazio $ X $ come un gruppo con l'operazione di giunzione
e avente come unità il laccio costante. Questo non si riesce a fare perché non sempre
la giunzione di un laccio con il suo inverso è il laccio costante. Per questo si passa al quoziente
rispetto la relazione di omotopia.

\begin{definition}
  Sia $ X $ uno spazio topologico e $ x_0 \in X $ un suo punto, allora la coppia $ (X, x_0) $ è detta \textbf{spazio topologico puntato}.
  \index{Spazio topologico puntato}
\end{definition}

\newmathsymb{homotop}{\sim_H}{Relazione di omotopia}
\begin{definition}
  Sia $ (X, x_0) $ uno spazio topologico puntato e $ f: I \to X $ una mappa continua tale che $ f(0) = f(1) = x_0 \; \forall t \in I $,
  cioè un laccio di base $ x_0 $,
  si dice che una funzione continua $ g $ è \textbf{omotopicamente equivalente} a $ f $ ($ g \sim_H f $) se esiste una funzione
  continua $ F \colon I \times I \to X $ tale che:
  \begin{itemize}
  \item $ F(0,x) = f(x) \; \forall x \in I $
  \item $ F(1,x) = g(x) \; \forall x \in I $
  \item $ F(t,0) = x_0 \; \forall s \in I $
  \item $ F(t,1) = x_0 \; \forall s \in I $
  \end{itemize}
  La relazione $ \sim_H $ è detta \textbf{relazione di omotopia tra lacci}\index{Relazione di omotopia} \index{Omotopia! \vedi{Relazione di omotopia}}
  e si dimostra essere una relazione di equivalenza.
\end{definition}

\begin{figure}[htbp]
  \centering
  \begin{tikzpicture}
    \draw (0, 0) rectangle (3,3);
    \draw[-Latex] (-0.5, 1) -- (-0.5, 2);
    \draw[-Latex] (1, -0.5) -- (2, -0.5);
    \node[left] () at (-0.5, 1.5) {$ t $};
    \node[below] () at (1.5, -0.5) {$ x $};
    \node[right] () at (0, 1.5) {$ x_0 $};
    \node[left] () at (3, 1.5) {$ x_0 $};
    \node[above] () at (1.5, 0) {$ f(x) $};
    \node[below] () at (1.5, 3) {$ g(x) $};
    \draw (0,2) -- (3,2);
    \draw[-Latex] (0, 2) -- (1.5, 2);
    \node[below] () at (1.5, 2) {$ F(t,x) $};
    \draw (6,1.5) circle [x radius=2, y radius=1];
    \node[above] () at (6, 2.5) {$ f $};
    \draw (6, 1.5) circle (0.5);
    \node[above] () at (6, 1.5) {$ g $};
    \draw[-Latex] (4.5, 1.5) -- (5, 1.5);
    \draw[-Latex] (7.5, 1.5) -- (7, 1.5);
  \end{tikzpicture}
  \caption{Omotopia: deforma $ f $ in $ g $ in modo continuo.}
  \label{fig:lez3:homotopy}
\end{figure}

Si definisce l'insieme;
\[
  \pi_1(X,x_0) = \quot{\set{f \colon I \to X | f \text{ continua}, f(0) = f(1) = x_0}}{\sim_H}
\]

Questo insieme può essere equipaggiato con un'operazione di somma facendolo diventare un gruppo,
questo è noto come \textbf{gruppo fondamentale}\index{Gruppo fondamentale}, e i suoi elementi
si indicano con la usuale notazione di classe di equivalenza.
Si vogliono definire le operazioni di gruppo:
siano $ [f], [g] \in \pi_i(X,x_0) $, si definisce $ [f][g] := [f \star g] $, dove l'operazione $ \star $ è
il \textbf{cammino composto}, o \textbf{giunzione}, \index{Cammino composto}
\index{Giunzione!\vedi{Cammino composto}} definita da:
\[
  (f \star g)(t) =
  \begin{cases}
    f(2t) & \text{se } 0 \leq t \leq \frac{1}{2} \\
    g(2t -1) & \text{se } \frac{1}{2} \leq t \leq 1
  \end{cases}
\]
Cioè è un cammino di base $ x_0 $ percorso a velocità doppia, metà del tempo
percorso su $ f $ l'altra metà su $ g $.
L'elemento neutro di questa operazione è il cammino costante $ 1_{\pi_1(X,x_0)} = [C_{x_0}] $ con
$ C_{x_0}(t) = x_0 \; \forall t $.
L'inverso di un elemento invece è $ [f]^{-1} = [\bar{f}] $ dove $ \bar{f} $ è il
cammino percorso in verso opposto, cioè definito da $ \bar{f}(t) = f(1-t) $, in
questo modo $ \bar{f}(0) = f(1) $ e $ \bar{f}(1) = f(0) $.

Alcune proprietà del gruppo fondamentale:
\begin{itemize}
\item $ \pi_1(X,x_0) $ è invariante omotopico, cioè se $ X \sim_H Y $, cioè se
  \[
    \exists f: X \to Y, g: Y \to X \; | \; f \circ g \sim_H 1_Y \text{ e }  g \circ f \sim_H 1_X
  \]
  allora $ \pi_1(X,x_o) \cong \pi_1(Y,f(x_0)) $. Questo in particolare porta alla seguente
  utile osservazione:
  \begin{osservation}
    Se due spazi topologici puntati hanno gruppi fondamentali non isomorfi allora
    non possono essere omotopicamente equivalenti.
  \end{osservation}
\item Se $ X $ è \textbf{contraibile}\index{Spazio contraibile} (cioè è
  omotopo ad un punto) allora vale che $ \pi_1(X,x_0) \cong 1 $, cioè il gruppo
  fondamentale è banale.
\item Si dimostra che:
  \begin{proposition}
    Se uno spazio tologico $ X $ è connesso per archi allora tutti i gruppi fondamentali
    degli spazi puntati $ (X,x_0) $ sono isomorfi, cioè si può omettere la dipendenza da $ x_0 $.
  \end{proposition}
  Questo intuitivamente è vero perché se gli spazi sono connessi per archi allora esistono cammini
  che collegano qualunque coppia di punti.
\end{itemize}

\begin{definition}
  Uno spazio topologico connesso per archi si dice \textbf{semplicemente connesso}\index{Semplicemente connesso}
  se il suo gruppo fondamentale è banale.
\end{definition}

\begin{osservation}
  Tutti gli spazi contraibili sono semplicemente connessi, ma
  non tutti gli spazi semplicemente connessi sono contraibili, come ad esempio $ \Sph{2} $.
\end{osservation}

\begin{itemize}
\item $ \pi_1(\Sph{1}) \cong \Z $, infatti si può costruire la mappa:
  \begin{align*}
    \sigma \colon I & \to \Sph{1} \\
    t & \mapsto  \me^{2 \pi i t}
  \end{align*}
  Questa è tale che $ \sigma(0) = \sigma(1) = 1 $ quindi $ [\sigma] \in \pi_1(\Sph{1}) $ e:
  \begin{align*}
     \pi_1(\Sph{1}) & \to \Z \\
    [\sigma] & \mapsto  1
  \end{align*}
  Ogni elemento è multiplo di $ \sigma $ e il fattore di proporzionalità conta il numero di avvolgimenti
  con segno del cammino su sé stesso.
\item $ \pi_1(X \times Y) \cong \pi_1(X) \times \pi_1(Y) $
\item Il gruppo fondamentale si calcola o partendo da gruppi omotopi oppure utilizzando il \textbf{teorema di Seifert–van Kampen}, \index{Teorema di Seifert–van Kampen}
  il quale fornisce un metodo algoritmico per il calcolo.
\end{itemize}

\begin{example}
  Si definisce:
  \[
    V_g =
    \begin{cases}
      \Sph{2} & \text{ se } g = 0 \\
      P_{\frac{4g}{N}} & \text{ se } g \geq 1 \in \mathbb{N}
    \end{cases}
  \]
  con $ P_{\frac{k}{N}} $ poligono con $ k $ lati e con identificazioni a coppie, come
  ad esempio nel caso $ g = 1 $ si ottiene un toro piatto identificando lati opposti
  di un quadrato.
  \begin{figure}[htbp]
    \centering
    \begin{subfigure}[htbp]{.45\linewidth}
      \centering{}
      \begin{tikzpicture}
        \draw (0,0) rectangle (3,3);
        \draw[-Latex] (0,0) -- (0,1.5);
        \draw[-Latex] (0,3) -- (1.5,3);
        \draw[-Latex] (3,3) -- (3,1.5);
        \draw[-Latex] (3,0) -- (1.5,0);
        \node[left] () at (0,1.5) {$ a $};
        \node[above] () at (1.5,3) {$ b $};
        \node[right] () at (3,1.5) {$ a $};
        \node[below] () at (1.5,0) {$ b $};
      \end{tikzpicture}
      \caption{Toro piatto, o anche toro di Clifford}
      \label{fig:lez3:clifford_torus}
    \end{subfigure}
    \begin{subfigure}[htbp]{.45\linewidth}
      \centering
      \def\svgwidth{0.9\textwidth}
      \input{images/torus_generators.pdf_tex}
      \caption{Generatori di un toro}
      \label{fig:lez3:torus_generators}
    \end{subfigure}
    \label{fig:lez3:torus}
    \caption{Toro}
  \end{figure}
  Si usano simboli combinatori per descrivere l'identificazione: si definisce un verso di percorrenza, si assegnano delle lettere a ciascun
  lato e si scrivono in ordine tali lettere, aggiungendo un esponente $ -1 $ quando il verso di percorrenza è opposto. In questo caso quindi
  si ha $ aba^{-1}b^{-1} $. Questo si estende a poligoni con $ 4g $ lati e si usa l'identificazione
  $ a_1 b_1 a_1^{-1} b_1^{-1} \dots a_g b_g a_g^{-1} b_g^{-1} $.

  Si dimostra che queste sono varietà topologiche, in particolare per $ g = 1 $ si ha un toro, per $ g = 2 $ un bitoro, \dots. $ g $ è
  detto \textbf{genere} \index{Genere}.

  \begin{definition}
    Una \textbf{varietà topologica}\index{Varietà topologica} $ \M $ è uno spazio topologico
    che localmente sembra uno spazio reale $ n $-dimensionale, cioè tale che esiste un
    interno $ n $ detto \textbf{dimensione}\index{Dimensione di una varietà topologica}
    tale ogni punto in $ \M $ possiede un intorno che è omeomorfo a $ \RN{n} $.
  \end{definition}

  Si trova con il teorema di Seifert-Van Kampen che:
  \[
    \pi_1(V_g) \cong
    \begin{cases}
      1 & \text{se } g = 0 \\
      \Z \oplus \Z & \text{se } g = 1 \\
      \langle a_1 b_1 \dots \Pi_{i=1}^g [a_i,b_i] = 1 \rangle & \text{se } g > 1
    \end{cases}
  \]
  Dove $ [,] $ è il commutatore, cioè esattamente $ a_1 b_1 a_1^{-1} b_1^{-1} \dots a_g b_g a_g^{-1} b_g^{-1} $.
  Solo per $ g = 0 $ o $ g = 1 $ si ottengono dei gruppi abeliani, ma io vorrei averlo sempre abeliano, quindi lo abelianizzo.
  \[
    \Ab{\pi_1(X)} = \quot{\pi_1(X)}{[\pi_1(X), \pi_1(X)]} = \quot{\pi_1(X)}{\pi_1'(X)}
  \]
  Chiaramente questo gruppo è abeliano e si calcola facilmente che $ \Ab{\pi_1(V_g)} \cong \Z^{2g} $ per $ g \geq 2 $,
  infatti il gruppo è generato su $ 2 g $ generatori $ a_1, b_1, a_2, b_2, \dots, a_g, b_g $ e poi
  si impone la relazione di identificazione e i commutatori diventano tutti banali.
  Si vedono facilmente anche i generatori, ad esempio per un toro sono riportati in figura.
  % L'abelianizzato è uno $ \Z $-modulo essendo un gruppo abeliano.
\end{example}


\subsection{Omomorfismo tra $ \RN{} $ e $ \RN{N} $}

\begin{definition}
  Un \textbf{arco}\index{Arco} in uno spazio topologico $ X $ tra i punti $ x_0 \in X $ e $ y_0 \in X $
  è una funzione continua da $ I = [0,1] $ a $ X $ tale che $ \alpha(0) = x_0 $ e $ \alpha(1) = y_0 $.
  Si dice che l'arco parte da $ x_0 $ e finisce in $ y_0 $.
\end{definition}

\begin{definition}
  Uno spazio topologico $ X $ è \textbf{connesso per archi}\index{Spazio connesso per archi} se per
  ogni coppia di punti $ x, y \in X $ esiste un arco che parte da $ x $ e termina in $ y $.
\end{definition}

\begin{definition}
  Uno spazio topologico $ X $ si dice \textbf{connesso per archi} \index{Spazio connesso per archi} se $ \forall x, y \in X $ esiste
  un arco con punto iniziale $ x $ e punto finale $ y $.
\end{definition}

\begin{proposition}
  Se $ f:X \to Y $ è una mappa continua suriettiva tra spazi topologici e se $ X $ è connesso per archi
  allora $ Y $ è connesso per archi. Questo vale in particolare se $ f $ è un omeomorfismo, cioè la
  connessione per archi è una proprietà invariante per omeomorfismi.
\end{proposition}

\begin{proof}
  Siano $ y_0, y_1 $ due punti di $ Y $. La funzione $ f $ è suriettiva, e dunque esistono $ x_0 $ e $ x_1 $ in $ X $
  tali che $ f(x_0)=y_0 $ e $ f(x_1)=y_1 $. Dato che $ X $ è connesso, esiste un cammino $ \alpha:[0,1] \to X $ tale che $ \alpha(0)=x_0 $
  e $ \alpha(1)=x_1 $. Ma la composizione di funzioni continue è continua, e quindi il cammino ottenuto componendo $ \alpha $ con $ f $:
  $ f \circ \alpha : [0,1] \to X \to Y $ è un cammino continuo che parte da $ y_0 $ e arriva a $ y_1 $.
\end{proof}

Si sa inoltre che:
\begin{proposition}
  $ \RN{n} $ è connesso per archi $ \forall n \in \mathbb{N} $.
\end{proposition}

È noto che $ \RN{} \not \simeq \RN{N} $ per $ n \geq 2 $, infatti basta togliere un punto a $ \RN{} $ che diventa sconnesso per archi
mentre $ \RN{N} $ rimane connesso per archi anche togliendogli un punto. In questa dimostrazione ho utilizzato
il seguente risultato fondamentale:
\begin{proposition}
  Se $ f: X \to Y $ è omeomorfismo tra spazi topologici allora $ f \rvert_U : U \to f(U) $ è omeomorfismo per ogni $ U \subseteq X $.
\end{proposition}
Nel caso considerato $ U = {x_0} $, siccome ho trovato un $ U $ per cui la funzione ristretta non è omeomorfismo $ f $
non può essere omeomorfismo. Infatti l'immagine di un punto rimane un punto.

Tuttavia vale anche che $ \RN{2} \not \simeq \RN{N} $ per $ n \geq 3 $, infatti:

\newmathsymb{homoto}{\homoto}{Omeomorfismo}
\begin{proof}
  Per assurdo $ f : \RN{2} \homoto \RN{N} $ è un omeomorfismo con
  $ n \geq 3 $, tolgo un punto da $ \RN{2} $, se $ f $ omeomorfismo anche la restrizione deve essere omeomorfismo, cioè
  $ \forall p \in \RN{2} \quad f:\RN{2} \setminus \set{p} \homoto \RN{N} \setminus \set{f(p)} $.
  Ma $ \RN{2} \setminus \set{p} \simeq \RN{} \times \mathcal{S}^1 $ con la mappa
  $ \vec{x} \mapsto \left( || \vec{x} ||, \frac{\vec{x}}{|| \vec{x} ||} \right) $ (dopo aver fatto
  una traslazione di $ p $ nell'origine, operazione che è certamente un omeomorfismo). In pratica
  sto dicendo che il piano senza un punto è omeomorfo ad un cilindro infinito.
  Analogamente $ \RN{n} \setminus \set{f(p)} \simeq \RN{} \times \Sph{n-1} $. Quindi se esiste un omeomorfismo tra $ \RN{2} $ e
  $ \RN{n} $ significherebbe che $ \RN{} \times \Sph{1} \simeq \RN{} \times \Sph{n-1} $, ma quindi i gruppi fondamentali
  dovrebbero essere isomorfi:
  $ \pi_1 (\RN{} \times \Sph{1}) \simeq \pi_1(\RN{}\times \Sph{n-1}) $ ma
  $ \pi_1 (\RN{} \times \Sph{1}) = \Z $ infatti il gruppo fondamentale di un prodotto è il prodotto dei gruppi
  fondamentali e $ \pi_1(\RN{}) = 1 $, $ \pi_1(\Sph{1}) = \Z $ dato che i lacci omotopicamente distinti
  sono quelli che avvolgono il buco un numero differente di volte. Analogamente $ \pi_1(\RN{}\times \Sph{n-1}) = 1 $
  perché le sfere sono contraibili. Trovo quindi che dovrebbero essere isomorfi $ \pi_1 (\RN{} \times \Sph{1}) = \Z $
  e $ \pi_1(\RN{}\times \Sph{n-1}) = 1 $ che è assurdo.
\end{proof}

Ho quindi dedotto proprietà topologiche a partire da considerazioni algebriche (con il gruppo fondamentale).
Il gruppo fondamentale è un invariante algebrico per problemi topologici.

\newmathsymb{fondgroup}{\pi_1}{Gruppo fondamentale}
\begin{definition}
  Si definisce il \textbf{gruppo fondamentale}\index{Gruppo fondamentale} di uno spazio topologico $ X $
  connesso per archi attorno al punto $ x_0 \in X $
  \[
    \pi_1 (X, x_0) = \quot{\set{ g: \Sph{1} \to X | g \text{ continua}, g(1) = x_0}}{\sim}
  \]
  e $ \sim $ è la relazione di omotopia: $ g_1 \sim g_2 $ se $ \exists G: \mathcal{S}^1 \times I \to X  $ tale che
  $ G(z,0) = g_1(z), G(z,1) = g_2(z), G(1,t) = x_o $ con $ G $ continua. In questo vedo $ \Sph{1} $ come sottospazio
  di $ \RN{2} $ con la topologia indotta (il punto $ 1 $ è un punto della circonferenza vedendola come
  insieme nello spazio complesso $ \Sph{1} = \set{ z \in \mathbb{C} | |z| = 1} $).
\end{definition}
Sostanzialmente il gruppo fondamentale è l'insieme dei lacci quozientato rispetto alla relazione di omotopia.
Infatti $ g $ è un laccio dato che è un arco e il punto di partenza e il punto di arrivo necessariamente
coincidono dato che $ g $ è definito su $ \Sph{1} $.
Questo perché l'insieme dei lacci non è strutturabile come gruppo in quanto il laccio costante non è
l'unità.

Ora voglio mostrare per assurdo che non esiste omomorfismo tra $ \RN{3} $ e $ \RN{N} $.

\begin{proof}
  Come nel caso precedente suppongo esiste $ f $ omeomorfismo tra $ \RN{3} $ a $ \RN{n} $,
  tolgo $ q $ da $ \RN{3} $ e $ f(q) $ da $ \RN{n} $, quindi ottengo
  l'omomorfismo tra $ \RN{} \times \Sph{2} \simeq \RN{} \times \Sph{n-1} $, ma i gruppi fondamentali
  associati sono banali, quindi sono isomorfi, e non è possibile replicare il ragionamento utilizzato sopra.
\end{proof}

Poincaré introdusse i gruppi di omotopia superiore.

\begin{definition}
  Si definiscono i \textbf{gruppi di omotopia superiore}\index{Gruppi di omotopia superiore} di uno spazio topologico $ X $
  attorno al punto $ x_0 $ per $ k \geq 2 $:
  \[
    \pi_k(X) (X, x_0) = \quot{\set{ g: \Sph{k} \to X | g \text{ continua}, \; g(p_0) = x_0}}{\sim}
  \]
  Con $ p_0 \in \Sph{k} $ e $ \sim $ relazione di omotopia.
\end{definition}
Studiare i gruppi di omotopia superiore è un problema aperto della topologia moderna.
Tuttavia si sa che:
\begin{enumerate}
\item $ \pi_k(\Sph{m}) = 1 \quad \text{per} \quad 1 \leq k < m \quad (m > 2)$
\item $ \pi_m(\Sph{m}) \simeq \Z \quad \text{per} \quad k = m $
\item $ \pi_1(\Sph{2}) = 1 $
\item $ \pi_2(\Sph{2}) \simeq \Z $
\item $ \pi_3(\Sph{2}) \simeq \Z $\footnote{Questo da origine alla fibrazione di Hopf che ha molte applicazioni in fisica.}
\end{enumerate}

\begin{definition}
  Sia $ A \subseteq X $ con $ X $ spazio topologico $ i: A \to X $ si definisce mappa di \textbf{inclusione}\index{Inclusione}
  e si scrive $ i: A \incl X $ se $ \forall a \in A $ vale che $ i(a) = a $.
\end{definition}


Anche se non so calcolare i gruppi di omotopia superiore non vorrei buttarli via\dots
Vorrei degli invarianti algebrici per problemi topologici, come i gruppi di omotopia.

\chapter{Omologia Singolare}

\section{Introduzione}

Inizio definendo l'omologia singolare, che è la più generale.

\section{Simplessi singolari}

Uso la teoria dell'omologia che mi permette di semplificare i problemi. La teoria
dell'omologia serve ad associare agli spazi topologici degli oggetti algebrici
meno complicati dei gruppi di omotopia.
Ci sono varie possibilità:
\begin{itemize}
  \item Omologia singolare
  \item Omologia cellulare
  \item Omologia persistente\footnote{Questa ha numerose applicazioni pratiche, come la ricostruzione di immagini.}
  \item Omologia simpliciale
\end{itemize}
Ma cosa è l'omologia? Assocerò ad ogni spazio topologico (anche patologico) gruppi abeliani e omomorfismi a partire
da applicazioni continue tra due spazi topologici. In tutto questo lavoro sempre con anello di base $ \Z $, che
quindi rimane sottinteso a meno di scriverlo esplicitamente.

\newmathsymb{simplexstd}{\Delta_k}{Simplesso standard}
\begin{definition}
  In $ \RN{k+1} $ si definisce il \textbf{simplesso standard}\index{Simplesso standard} $ \Delta_k $ l'insieme:
  \[
    \Delta_k = \set{(x_1,x_2,\dots) \in \RN{k+1} | \forall i \; 0 \leq x_i \leq 1 \text{ e } \sum_{i=1}^{k+1}x_i = 1}
  \]
  Le coordinate $ x_i $ sono dette \textbf{coordinate baricentrali}\index{Coordinate baricentrali}.
\end{definition}

\begin{osservation} Alcuni esempi sono:
  \begin{itemize}
  \item $ \Delta_0 $ è un punto.
  \item $ \Delta_1 $ è un segmento, che è omeomorfo a $ [0,1] $.
    \begin{figure}[htbp]
      \centering
      \begin{tikzpicture}
        \draw[-Latex] (0,0) -- (3,0);
        \draw[-Latex] (0,0) -- (0,3);
        \draw[thick] (0,2) -- (2,0);
        \node[below] () at (2,0) {1};
        \node[left] () at (0,2) {1};
      \end{tikzpicture}
      \caption{1-Simplesso standard}
      \label{fig:lez1:1_standard_simplex}
    \end{figure}
  \item $ \Delta_2 $ è un triangolo
  \item $ \Delta_3 $ è un tetraedro
  \item \dots
  \end{itemize}
\end{osservation}
\begin{figure}[htbp]
  \centering
  \begin{subfigure}{.2\textwidth}
    \centering
    \begin{tikzpicture}
      \draw (0,0) circle (0.05);
    \end{tikzpicture}
    \caption{$ \Delta_0 $}
  \end{subfigure}
  \begin{subfigure}{.2\textwidth}
    \centering
    \begin{tikzpicture}
      \draw (0,0) -- (2,0);
    \end{tikzpicture}
    \caption{$ \Delta_1 $}
  \end{subfigure}
  \begin{subfigure}{.2\textwidth}
    \centering
    \begin{tikzpicture}
      \draw (0,0) -- (2,0) -- (1, 1.7) -- cycle;
    \end{tikzpicture}
    \caption{$ \Delta_2 $}
  \end{subfigure}
  \begin{subfigure}{.33\textwidth}
    \centering
    \def\svgwidth{0.56\textwidth}
    \input{images/Tetrahedron.pdf_tex}
    \caption{$ \Delta_3 $}
  \end{subfigure}%
  \caption{Simplessi standard}
  \label{fig:lez1:standard_simplexes}
\end{figure}

\begin{definition}
  Dato uno spazio topologico $ X $ si definisce il \textbf{$ k $-simplesso singolare}\index{$ k $-simplesso singolare}
  in $ X $ come un'applicazione continua $ \sigma: \Delta_k \to X $.
\end{definition}
Spesso conviene identificare il $ k $-simplesso con la sua immagine in $ X $.
In questo modo uno $ 0 $-simplesso è un punto in $ X $, mentre un $ 1 $-simplesso singolare potrebbe
essere sia un segmento che un punto (se la mappa è costante).
Siccome non c'è relazione tra la dimensione dello spazio di partenza e lo spazio di arrivo
(ad esempio la curva di Peano) il simplesso può deformare, ed è per questo che è detto singolare.

\begin{example}
  Un esempio di $ k $-simplesso singolare in cui è particolarmente evidente la possibilità di fare l'identificazione
  è la mappa identità: $ \Id{} \colon \Delta_k \to \Delta_k $.
\end{example}

Voglio costruire un complesso di gruppi abeliani e definire l'omologia singolare come l'omologia di tale complesso.

$ S_\bullet $ è il compesso ($ S $ sta per singolare), cioè:
\[
  \dots \to S_{k+1}(X) \to S_k(X) \to S_{k-1}(X) \to \dots \to S_0(X)
\]
Dove:
\begin{align*}
  S_k(X) ={}& \{\text{combinazioni lineari finite a coefficienti interi: } \\
         & \sum_g n_g g \;|\; n_g \in \Z, g \; k-\text{simplessi singolari di } X \}
\end{align*}
$ S_k(X) $ è un gruppo abeliano con l'operazione somma definita naturalmente:
\[
  \sum_g n_g g + \sum_h n_h h =   \sum_g n_g g + \sum_g n_g^\star g = \sum_g (n_g + n_g^\star)g
\]
Inoltre $ \forall k < 0 $ si pone $ S_k(X) = 0 $. Un elemento generico di $ S_k(X) $
è una somma formale finita (cioè con un numero finito di coefficienti non nulli)
su tutti i possibili $ k $-simplessi singolari in $ X $.
\begin{example}
  \[
    (n_1 g_1 + n_2 g_2 + 2 n_3 g_3) + (m_1 g_1 + m_4 g_4) = (n_1 + m_1)g_1 + n_2 g_2 + 2 n_3 g_3 + m_4 g_4
  \]
\end{example}
Questa è una somma con tutte le giuste proprietà. Lo zero è la catena con tutti i coefficienti nulli,
mentre l'inverso è la catena con i coefficienti opposti.
Queste catene sono chiamate $ k $-\textbf{catene singolari}\index{$ k $-catene singolari}.
$ S_k(X) $ è generato da tutte le possibili applicazioni continue da $ \Delta_k $ a $ X $, cioè:
\[
  S_k(X) = \langle \set{g | g \text{ $ k $-simplesso singolare in $ X $}} \rangle
\]
Si nota che le catene sono somme formali di mappe e non sono esse stesse mappe.
% Le $ k $-catene singolari in qualche modo generalizzano la caratteristica di Eulero.
% \underline{\emph{Verificare questa affermazione}}
\begin{example}[$ k = 0$]
  Se $ k = 0 $ allora $ S_0(X) $ sono catene di punti ($ g_0 : \Delta_0 \to X $, identifico l'applicazione
  con il punto in $ X $ sapendo che l'immagine di un punto è un punto)
  \[
    S_0(X) = \set { \sum n_i p_i | n_i \in \Z, \; p_i \in X}
  \]
\end{example}
\begin{osservation}
  Quando è possibile faccio un abuso di notazione e identifico la mappa con la sua immagine
  nello spazio topologico.
\end{osservation}

Ora devo introdurre le applicazioni tra i vari $ S_k $, queste applicazioni saranno il bordo.

Definisco $ h: \Delta_1 \to X $ in modo tale che $ h(\Delta) = \alpha $ dove $ \alpha $ è un \textbf{arco}.

\begin{figure}[htbp]
  \centering
  \begin{tikzpicture}
    \draw[-Latex] (0,0) -- (2,0);
    \draw[-Latex] (0,0) -- (0,2);
    \draw[thick] (0,1) -- (1,0);
    \draw plot [smooth cycle] coordinates {(5,2) (6,3) (7,3) (6,0) (4,0)};
    \node () at (7,1) {$ X $};
    \draw plot [smooth, tension = 1] coordinates {(6,2) (5,1) (5.5,0.5)};
    \node[circle] () at (6,2) {\textbullet};
    \node[circle] () at (5.5,0.5) {\textbullet};
    \draw[->] (0.75,0.5) -- (4.75,1);
  \end{tikzpicture}
  \caption{1-Simplesso singolare}
  \label{fig:lez1:1_standard_simplex_with_arc}
\end{figure}

Posso ottenere una $ 0 $-catena prendendo i punti estremi dell'arco, infatti il bordo di un $ 1 $-simplesso
è uno $ 0 $-simplesso. L'idea è quindi ottenere simplessi di ordine più piccolo prendendo il
bordo dei simplessi.

\begin{definition}
  Sia $ \Delta_k $ un $ k $-simplesso standard con $ k \geq 0 $ si definisce l'operatore \textbf{faccia}\index{Operatore faccia}
  come la mappa $ F_i^{\;k}: \Delta_{k-1} \to \Delta_k $ tale che $ F_i^{\;k}(\Delta_{k-1}) $ è una faccia di $ \Delta_k $.
\end{definition}
L'operatore faccia prende un $ k $-simplesso standard e lo immerge in un qualche senso in un
simplesso più grande, ad esempio manda un punto in uno degli estremi di un segmento (nel caso $ k = 0 $),

\begin{example}[$ k = 2 $]
  Per $ k = 2 $ vale che:
  \[
    \Delta_2 = \set{ (x_1,x_2,x_3) \in \RN{3} | x_1 + x_2 + x_3 = 1, \; 0 \leq x_i \leq 1 \; \forall i}
  \]
  Si definisce la base $ e_0 = (1,0,0) \; e_1 = (0,1,0) \; e_2 = (0,0,1) $, voglio vedere il bordo del triangolo
  come facce.

  \begin{figure}[htbp]
    \centering
    \begin{tikzpicture}
      \draw[-Latex] (0,0) -- (4,0);
      \draw[-Latex] (0,0) -- (0,4);
      \draw[-Latex] (0,0) -- (-2,-2);
      \node[below] () at (2,0) {$ e_1 $};
      \node[right] () at (0,2) {$ e_2 $};
      \node[right, below] () at (-1,-1) {$ e_0 $};
      \draw (-1,-1) -- (2,0);
      \draw (0,2) -- (2,0);
      \draw (0,2) -- (-1,-1);
      \draw (5,0) -- (9,0) -- (7,3) -- cycle;
      \node[left] () at (5,0) {$ e_0 $};
      \node[right] () at (9,0) {$ e_2 $};
      \node[above] () at (7,3) {$ e_1 $};
      \node[below] () at (7,0) {$ F_2^{(2)}(\Delta_1) $};
      \node[left] () at (6,2) {$ F_1^{(2)}(\Delta_1) $};
      \node[right] () at (8,2) {$ F_0^{(2)}(\Delta_1) $};
    \end{tikzpicture}
    \caption{Azione dell'operatore faccia}
    \label{fig:lez1:standard_simplex_faces}
  \end{figure}
\end{example}

Il segmento faccia $ i $-esimo è quello che non contiene il vertice $ i $-esimo, cioè
\emph{dimentico} un punto e gli altri punti diventano vertici del simplesso.

In generale se $ \Delta_k $ è un simplesso standard definisco la base canonica (si noti
che la base canonica è ordinata):
\begin{gather*}
  e_0 = (1,0,0,\dots)                            \\
  e_1 = (0,1,0,\dots)                            \\
  e_2 = (0,0,1,\dots)                            \\
  \dots
\end{gather*}
Questi sono i vertici del simplesso, definisco l'azione dell'operatore faccia
come:
\[
  \begin{cases}
    F_i^{\; k}(e_j) = e_{j+1}     & \text{se } j \geq i \\
    F_i^{\; k}(e_j) = e_{j} & \text{se } j < i
  \end{cases}
\]

Se fosse un tetraedro dimenticando punti ottengo triangoli e dimenticando
triangoli ottengo punti, come è giusto.

\begin{exercise}
  Dimostrare che se $ [\cdot, \cdot] $ indica l'inviluppo convesso allora:
  \begin{enumerate}
  \item Per $ j > i $ vale che $ F_i^{\; k+1} \circ F_i^{\; k} = [e_0, \dots, \hat{e}_i, \dots, \hat{e}_j, \dots, e_k ] $.
  \item Per $ j \leq i $ vale che $ F_i^{\; k+1} \circ F_i^{\; k} = [e_0, \dots, \hat{e}_j, \dots, \hat{e}_{i+1}, \dots, e_k ] $.
  \end{enumerate}
  dove i cappucci indicano che quell'elemento è omesso.
\end{exercise}

\begin{definition}
  L'\textbf{inviluppo convesso}\index{Inviluppo convesso} di un insieme $ U $ in $ \RN{n} $ è il più piccolo
  insieme convesso che contiene $ U $.
\end{definition}
\begin{definition}
  Un insieme in $ \RN{n} $ si dice \textbf{convesso}\index{Insieme convesso} se contiene
  il segmento che unisce ogni coppia di punti dell'insieme.
\end{definition}

\begin{definition}
  Dato un $ k $-simplesso singolare $ \sigma: \Delta_k \to X $ si definisce la mappa $ \sigma^{(i)} \colon \Delta_{k-1} \to X $ come la restrizione
  di $ \sigma $ sulla faccia $ i $-esima del simplesso, cioè $ \sigma^{(i)} = \sigma \circ F_i^{\; k} $,
  si definisce quindi il \textbf{bordo}\index{Bordo} come la mappa:
  \begin{align*}
    \partial \colon \Sigma_k(X) & \to \Sigma_{k-1}(X) \\
    \sigma & \mapsto  \sum_{i=0}^{k}(-)^i \sigma^{(i)}
  \end{align*}
  dove $ \Sigma_k(X) $ indica lo spazio dei $ k $-simplessi singolari di $ X $.
  % di $ \sigma $ come $ \partial_k \sigma = \sum_{i=0}^{k}(-)^i \sigma^{(i)} $.
\end{definition}
Il bordo sostanzialmente corrisponde alla somma alterna delle facce.

\begin{figure}[htbp]
  \centering
  \begin{tikzpicture}
   \draw[-Latex] (0,0) -- (3,0);
    \draw[-Latex] (0,0) -- (0,3);
    \draw[-Latex] (0,0) -- (-1.5,-1.5);
    \draw (-1,-1) -- (2,0);
    \draw (0,2) -- (2,0);
    \draw (0,2) -- (-1,-1);
    \draw plot [smooth cycle] coordinates {(5,2) (6,3) (7,3) (6.5,0) (4,0)};
    \node () at (7.5,1) {$ X $};
    \draw plot [smooth, tension = 1] coordinates {(6,2) (5,1) (5.5,0.5)};
    \draw plot [smooth, tension = 1] coordinates {(6,2) (6.2,0.7) (5.5,0.5)};
    \draw[->] (0.75,0.5) -- (4.75,1);
    \node[above] () at (2.5, 0.75) {$ \sigma $};
    \draw[->] (0.35,-0.65) to [out=-30,in=-120]  (5.35,0.45);
    \node () at (2.5, -1.45) {$ \sigma^{(i)} $};
  \end{tikzpicture}
  \caption{Azione di $ \sigma $ e $ \sigma^{(i)} $}
  \label{fig:lez1:sigma}
\end{figure}

\begin{example}[$ k = 1 $]
  Per $ k = 1 $ vale che $ \partial_1 \sigma = p_1 - p_0 $, infatti:
  \begin{align*}
    \sigma^{0} = \sigma \circ F_0^{\; 1} = \sigma(1) = p_1 \\
    \sigma^{1} = \sigma \circ F_1^{\; 1} = \sigma(0) = p_0
  \end{align*}
  Il bordo è la somma con i segni alternati: $ \partial_1 \sigma = p_1 - p_0 $. Tecnicamente il bordo è una mappa
  quindi sarebbe più corretto scrivere $ \partial_1 \sigma = \sigma^{(1)} - \sigma^{(0)} $ dove l'azione di queste due mappe è
  quella di mandare un estremo dell'intervallo $ [0,1] $ in $ p_0 $ o $ p_1 $.
\end{example}

Allora si definisce l'operatore bordo sul complesso delle catene $ \partial_k: S_k(X) \to S_{k-1}(X) $ estendendolo per linearità
$ \partial_k \left( \sum_g n_g g\right) = \sum_g n_g \partial_k g $ (infatti si è definita l'azione sui generatori $ g $).

Devo mostrare che $ \partial_k $ è un omomorfismo e che soddisfa $ \partial_k \circ \partial_{k+1} = 0 $. Comincio con il fatto che è un omomorfismo.

\begin{proof}
  \begin{gather*}
    \partial_k \left( \sum_g n_g g + \sum_g m_g g\right) = \partial_k \left( \sum_g(m_g + n_g)g \right) = \sum_g (m_g + n_g) \partial_k g = \\
    = \sum_g n_g \partial_k g + \sum_g m_g \partial_k g = \partial_k \left( \sum_g n_g g\right) + \partial_k \left( \sum_g m_g g \right)
  \end{gather*}
  Dove si è usato che la mappa di bordo è lineare.
\end{proof}

Quindi il complesso è costituito da:
\[
  \begin{tikzcd}
   \dots \arrow{r}{\partial_{k+1}} & S_k(X) \arrow{r}{\partial_k} & S_{k-1}(X) \arrow{r}{\partial_{k-1}} & \dots
  \end{tikzcd}
\]
Devo verificare che $ \partial_k \circ \partial_{k+1} = 0 $ (spesso come notazione si pone $ \partial^2 = 0 $).
\underline{SISTEMARE}
\begin{proof}
  Se $ \sigma $ è un $ k $-complesso singolare, cioè $ \sigma : \Delta_k \to X $ continua:
  \begin{align*}
    & \partial_k \circ \partial_{k+1} \sigma = \partial_k \left( \sum_{j=0}^{k+1}(-)^j (\sigma \circ F_j^{\; k+1}) \right) =  \sum_{j=0}^{k+1}(-)^j \partial_k (\sigma \circ F_j^{\; k+1}) &= \\
                    & = \sum_{j=0}^{k+1} (-)^j \sum_{i=0}^k (-)^i (\sigma \circ F_j^{\; k+1}) \circ F_i^{\; k} = \sum_{j = 0}^{k+1} \sum_{i = 0}^{k} (-)^{j+i} \sigma \circ F_j^{\; k+1} \circ F_{j}^{\; k} &= \\
    & = \sum_{0 \leq i < j \leq k + 1} (-)^{i+j} \sigma \circ F_j^{\; k+1} \circ F_i^{\; k} + \sum_{0 \leq j < i \leq k} (-)^{i+j} \sigma \circ F_j^{\; k+1} \circ F_i^k &=
  \end{align*}
  Rinominando nella seconda sommatoria \dots
  \begin{align*}
    & = \sum_{0 \leq i < j \leq k + 1} (-)^{i+j} \sigma \circ F_j^{\; k+1} \circ F_i^{\; k} + \sum_{0 \leq j < i \leq k} (-)^{i+j} \sigma \circ F_{i+1}^{\; k+1} \circ F_j^{\; k} & =  \\
    & = 0
  \end{align*}
  Dove nel penultimo passaggio si sono utilizzate le identità lasciate da dimostrare come esercizio,
  e nell'ultimo si è rinominato nel secondo termine $ i + 1 $ con $ i $.
  \begin{exercise}
    Verificare che fa veramente zero.
  \end{exercise}
  Si nota che è di importanza cruciale il fatto che si è definito il bordo con i segni alternati.
\end{proof}

% lezione 2
%  _     _____ ________ ___  _   _ _____   ____
% | |   | ____|__  /_ _/ _ \| \ | | ____| |___ \
% | |   |  _|   / / | | | | |  \| |  _|     __) |
% | |___| |___ / /_ | | |_| | |\  | |___   / __/
% |_____|_____/____|___\___/|_| \_|_____| |_____|


Sia $ X $ uno spazio topologico, voglio definire l'omologia singolare $ H_k(X) $, cioè il $ k $-esimo gruppo di omologia
singolare. Costruisco il complesso $ (S_\bullet(X), \partial) $ con:
\[
  S_k(X) = \set{ \sum_g n_g g | g \text{ simplesso singolare, } n_g \in \Z }
\]
E $ \partial_k : S_k(X) \to S_{k-1}(X) $ applicazione di bordo con $ \partial_k(g) = \sum_{i=0}^k(-)^ig^{(i)} $ con $ g: \Delta_k \to X $, e poi lo estendo per
linearità su tutti gli elementi di $ S $, dove $ g^{(i)} = g \circ F_i^{\; k} $.

Siccome $ \partial_{k-1} \circ \partial_k = 0 $ si ha il complesso
\[
  \begin{tikzcd}
   \dots \arrow{r}{\partial_{k+1}} & S_k(X) \arrow{r}{\partial_k} & S_{k-1}(X) \arrow{r}{\partial_{k-1}} & \dots
  \end{tikzcd}
\]
Inoltre $ \partial_k \circ \partial_{k-1} $ è la mappa nulla dalle catene singolari di $ S_k(X) $
a quelle di $ S_{k-2}(X) $, in questo modo $ (S_\bullet(X), \partial) $ è un complesso di gruppi abeliani.
Posso quindi calcolare l'omologia di $ (S_\bullet(X),\partial) $ come l'avevo definita
in precedenza:
\[
  H_k(S_\bullet(X)) = \quot{\ker{\partial_k}}{\im{\partial_{k+1}}}
\]
Vale che $ \ker{\partial_k} = \set{c \in S_k(X) | \partial_k(c) = 0} $, cioè le $ k $-catene con
bordo nullo, questi sono chiamati $ k $-cicli.

\begin{definition}
  Sia $ (S_\bullet(X),\partial) $ un complesso di moduli, gli elementi di $ \ker{\partial} $ sono detti
  \textbf{$ k $-cicli}\index{$ k $-ciclo}. Un $ k $-ciclo è quindi una $ k $-catena
  con bordo nullo:
  \[
    c \text{ ciclo } \Leftrightarrow \partial c = 0
  \]
  L'insieme dei $ k $-cicli è indicato con $ Z_k(X) $, cioè: $ Z_k(X) = \ker{\partial} $.
\end{definition}

Si pone invece $ B_k(X) $ come l'insieme dei bordi, cioè le $ k $-catene singolari
che sono immagini di $ k+1 $-catene, cioè esplicitamente:
\[
  B_k(X) = \set{\eta \in S_k(X) | \exists b \in S_{k+1}(X), \partial b = \eta}
\]

Per definizione si ha quindi che $ H_k(X) = \quot{Z_k(X)}{B_k(X)} $, cioè il gruppo
di omologia è formato dai cicli modulo i bordi.

Esplicitamente gli elementi di $ H_k(X) $ sono classi di equivalenza tali che se $ [c] \in H_k(X) $
con $ \partial c = 0 $, e $ c_1 \in [c] $ allora
$ c_1 - c \in B_k(X) $ e $ \partial c_1 = 0 $ quindi esiste $ b $ tale che $ c_1 - c = \partial b $.
Cioè due elementi stanno nella stessa classe di equivalenza se differiscono per un bordo:

\newmathsymb{homolog}{\sim_{hom}}{Relazione di omologia}
\begin{definition}
  Due elementi $ a,b $ si dicono \textbf{omologhi} \index{Elementi omologhi} se differiscono per un bordo.
  \[
    a \sim_{hom} b \Leftrightarrow \exists c \; | \; \partial_k c = a - b
  \]
\end{definition}

\begin{osservation}
  Vale che $  H_k(X) = 0 \Leftrightarrow B_k(X) = Z_k(X) $, cioè se ogni ciclo è un bordo, come si è già osservato.
  In generale si ha che $ B_k(X) \subseteq Z_k(X) $ e possono esserci cicli che non sono immagini di bordi.
\end{osservation}

$ \partial_k c $ è il bordo di un $ k $-ciclo, se $ \partial_k c = 0 $ significa che il ciclo non ha bordo, inoltre
se $ c = \partial_{k+1} b $ allora $ c $ è bordo di qualcosa: $ c $ è un bordo che non ha bordo. Questo tipo
di oggetti è di interesse centrale.

Scopo del corso è studiare $ H_k(X) $ e capire se si possono determinare a meno di isomorfismi.
In alcuni casi è possibile calcolare esplicitamente tutti i gruppi di omologia (come nel caso dell'omologia
cellulare).

\subsection{$ H_0(X) $}

\begin{proposition}
  Sia $ X $ uno spazio topologico connesso per archi, allora $ H_0 \cong \Z $, cioè è uno $ \Z $-modulo libero di rango 1.
  In effetti $ H_0(X) $ \emph{conta} le componenti connesse per archi in $ X $ e quindi dà informazioni di natura geometrica.
\end{proposition}

\begin{proof}
  Dalla definizione di gruppo di omologia: $ H_0(X) = \quot{Z_0(X)}{B_0(X)} $.
  Ma $ Z_0(X) = \set{ c \in S_o(X) | \partial_0 c = 0} $ e $ S_0(X) = \set{ \sum n_i p_i | n_i \in \mathbb{N}, p_i \in X} $.
  % Tecnicamente uno $ 0 $-simplesso singolare è una mappa $ \sigma_0 : \Delta_0 \to X $ tale che manda $ \Delta_0 = 1 $ in $ \sigma_0(1) = p_0 $ e per
  % questo è naturale l'identificazione con i punti immagine dello spazio topologico.
  Sia $ c \in S_0(X) $ allora $ c = \sum n_i p_i $, e vale che $ \partial_0(c) = \sum n_i \partial_0 (p_i) = 0 $, infatti
  $ \partial_0 : S_0(X) \to S_{-1}(X) $, ma per $ k < 0 $ $ S_{k} = 0 $ per definizione.
  Quindi per ora ho che:
  \[
    Z_0(X) = \ker{\partial_0} = S_0(X) \; \Rightarrow \; H_0(X) = \quot{S_0(X)}{B_0(X)}
  \]
  Per definizione $ B_0(X) = \set{ x \in S_0(X) | \exists \alpha \in S_1(X), \partial_1(\alpha) = x} $. % $ \alpha $ è una catena.
  Sia $ p_0 \in X $, allora $ q \sim_{hom} p_0 $ se e solo se $ \exists \alpha \in S_1(X) $ tale che $ q - p_0 = \partial_1 \alpha $.
  Per questo motivo i punti sono tutti omologhi, infatti
  essendo $ X $ connesso per archi esiste un arco $ \alpha $ che connette $ q $ e $ p_0 $, ma
  per definizione gli archi sono applicazioni continue da $ \Delta_1 $ a $ X $ che hanno come bordo $ q - p_0 $.
  Esiste quindi un'unica classe di equivalenza che è la classe di equivalenza di un punto.
  Per questo il gruppo è omomorfo a $ \Z $.

  \begin{definition}
    Si definisce inoltre la mappa \textbf{grado} \index{Grado} come l'applicazione che manda una catena in $ S_0(X) $ nella somma
    dei suoi coefficienti:
    \begin{align*}
      \deg \colon S_0(X)    & \to  \Z \\
                     \sum n_i p_i & \mapsto  \sum n_i
    \end{align*}
  \end{definition}

  \begin{theorem}[Teorema fondamentale degli omomorfismi]\index{Teorema fondamentale degli omomorfismi}
    Sia $ f: \mathcal{G}_1 \to \mathcal{G}_2 $ un omomorfismo tra gruppi abeliani, allora vale che:
    \[
      \quot{\mathcal{G}_1}{\ker{f}} \cong \im{f}
    \]
  \end{theorem}

  \begin{proposition}
    La mappa grado gode di alcune proprietà:
    \begin{enumerate}
    \item $ \deg $ è un omomorfismo di gruppi abeliani
    \item $ \deg $ è suriettivo
    \item $ \ker{\deg} \cong B_0(X) $
    \end{enumerate}
    Se dimostro questa proprietà utilizando il primo teorema fondamentale di isomorfismo:
    \[
      \quot{S_0(X)}{B_0(X)} \cong \im{\deg}
    \]
    Ma $ \deg $ è suriettiva, quindi $ \im{\deg} = \Z $, perciò:
    \[
      H_0(X) = \quot{S_0(X)}{B_0(X)} \cong \Z
    \]

    Dimostro quindi questa proposizione.

    \begin{proof}
      \begin{enumerate}
      \item
        Sia $ c_1 = \sum n_i p_i $ e $ c_2 = \sum m_i q_i $, bisogna mostrare che:
        \[
          \deg(c_1 + c_2) = \deg(c_1) + \deg(c_2)
        \]
        ma:
        \[
          c_1 + c_2 = \sum n_i p_i + \sum m_i q_i = \sum (n_i + m_i)r_i
        \]
        dove $ r_i $ è quello comune tra le catene, oppure è zero se
        l'elemento è presente in solo uno delle due catene.
        Quindi:
        \[
          \deg(c_1 + c_2) = \sum (n_i + m_i) = \sum n_i + \sum m_i = \deg(c_1) + \deg(c_2)
        \]
        Alternativamente in modo più semplice si può osservare l'azione di $ \deg $ sui generatori di $ S_0(X) $,
        che è unico e viene mandato dalla mappa grado in $ 1 $, quindi si estende per linearità.
      \item
        La mappa è suriettiva, basta prendere un punto $ p \in X $ e la controimmagine di $ m \in \Z $ è $ \deg^{-1}(m) = mp $
      \item
        \underline{SISTEMARE}
        Mostro che $ \ker{\deg} = B_0(X) $. Sia $ c \in \ker{\deg} $ cioè tale che $ \deg(c) = 0 $, se
        $ c = \sum n_i p_i $ allora $ \sum n_i = 0 $, bisogna mostrare che $ c \in B_0(X) $, cioè che $ \exists b \in S_1(X) $ con $ \partial_1 b = c $.
        Considerato $ p_0 $ e altri punti $ p_i $, ci sono archi
        $ \lambda_i s $ che li uniscono a $ p_0 $. $ b $ si può costruire in questo modo.

        Siano
        $ \lambda_i : [0,1] \to X $ con $ \lambda_i(0) = p_0 $ e $ \lambda_i(1) = p_i $ considero $ c - \partial\left(\sum n_1 \lambda_i \right) =
        c - \sum n_i \partial \lambda_i = c - \sum n_i (p_i - p_0) = c - \sum n_i p_i = \sum n_i p_0 = 0 $. Siccome per ipotesi $ p_0 \in \ker{\deg} $
        e $ c = \sum n_i p_i $ allora $ c = \partial(\sum n_i \lambda_i) $ quindi $ \sum n_i \lambda_i = b $ da cui $ \ker{\deg} \subseteq B_0(X) $.
        Mi rimane da mostrare che $ B_0(X) \subseteq \ker{\deg} $, infatti ora mostro che se $ c \in B_0(X) $ allora
        $ \deg(c) = 0 $. $ c = \partial b $ ma se $ \lambda_i $ sono gli archi $ b = \sum m_i \lambda_i $ quindi  $ \partial b = \sum n_i \partial \lambda_i $
        ma $ \partial \lambda_i = \lambda_i(1) - \lambda_i(0) $ e l'azione dell'opertaore grado è quella di sommare i coefficienti, quindi
        \[
          \deg(c) = \deg(\partial b) = \sum n_i \deg(\partial \lambda_i) = 0
        \]
      \end{enumerate}
    \end{proof}
  \end{proposition}
  Per questo $ H_0 (X) \cong \Z $ generato dalla classe $ [p] \; \forall p \in X $ (con $ X $ connesso per archi).
\end{proof}

Se ci sono più componenti connesse per archi posso ripetere il ragionamento senza connettere componenti
distinte, quindi trovo che:
\[
  H_0(X) \cong \Z^{N_c}
\]
Dove $ N_c $ è il numero di componenti connesse per archi di $ X $ con $ N_c < + \infty $, in pratica
$ H_0(X) $ è generato da un insieme formato da un punto per ogni componente connessa per archi.

\subsection{$ H_1(X) $}
% lezione 3 parte 2

Cosa si può dire invece su $ H_1(X) $?

Sia $ X $ spazio topologico e $ x_0 \in X $, allora alla coppia $ (X, x_0) $ si associa il gruppo fondamentale
$ \pi_1(X,x_0) $. In generale il gruppo fondamentale non è abeliano, allora conviene studiare la versione abelianizzata:
$ \Ab{\pi_1(X,x_0)} = \quot{\pi_1(X,x_0)}{\pi_i(X,x_0)'} $ dove $ ' $ indica il \textbf{gruppo derivato}\index{Gruppo derivato}, cioè il gruppo
generato dai commutatori.
\[
  \pi_1(X,x_0)' = [\pi_1(X,x_0), \pi_1(X,x_0)] = \langle\set{[g,h] | g,h \in \pi_1(X,x_0)}\rangle
\]
Se $ X $ è connesso per archi allora $ \Ab{\pi_1(X,x_0)} \cong H_1(X) $, quindi conoscendo il gruppo fondamentale si può calcolare anche
il primo gruppo di omologia, che quindi è sostanzialmente formato dai lacci (modulo omotopia) che commutano tra loro.

% lezione 3 parte 2

\begin{osservation}
  Sia $ X $ uno spazio topologico connesso per archi e $ \mathcal{G} $ un gruppo abeliano se esiste un omomorfismo
  di gruppi $ \phi: \pi_1(X) \to \mathcal{G} $ allora esiste $ \phi' : \Ab{\pi_1(X)} \to \mathcal{G} $ omomorfismo di gruppi abeliani.
  \[
    \begin{tikzcd}
      \pi_1(X) \arrow{r}{\phi} \arrow{d}{P} & \mathcal{G} \\
      \Ab{\pi_1(X)} \arrow{ur}{\phi'}
    \end{tikzcd}
  \]
  dove $ P $ è la proiezione sul quoziente.
\end{osservation}
\begin{proof}
  Bisogna mostrare che $ \phi' $ è ben definito, cioè che se:
  \[
    \phi'(a) = \phi'(P(c)) = \phi(c) \quad \text{ e } \quad \phi'(a) = \phi'(P(d)) = \phi(d)
  \]
  allora $ \phi(c) = \phi(d) $. Se $ c \sim_H d $ allora $ P(c) = P(d) $,
  e quindi $ c = d[x,y] $, in quanto gli elementi in $ \Ab{\pi_1(X)} $
  differiscono per commutatori. Applicando $ \phi $: $ \phi(c) = \phi(d[x,y]) $,
  siccome $ \phi $ è omomorfismo:
  \[
    \phi(d[x,y]) = \phi(d)\phi([x,y]) = \phi(d) \phi(xyx^{-1}y^{-1}) = \phi(d) \phi(x) \phi(y) \phi(x)^{-1} \phi(y)^{-1} = \phi(d)
  \]
  dove nell'ultimo passaggio ho utilizzato che il gruppo è abeliano.
\end{proof}

Questa osservazione dipende crucialmente dal fatto che il gruppo è abeliano.

\begin{proposition}
  Se $ X $ è uno spazio topologico connesso per archi allora $ \Ab{\pi_1(X)} \cong H_1(X) $,
  cioè si può passare dall'equivalenza omologica a quella omotopica, e
  in questo modo per il teorema di Seifert-van Kampen si possono ottenere tante
  informazioni su $ H_1(X) $.
\end{proposition}

\begin{proof}
  Per dimostrare che $ \Ab{\pi_1(X)} \cong H_1(X) $ trovo un isomorfismo di
  gruppi abeliani.
  Per ora so che $ H_1(X) $ è uno $ \Z $-modulo. Se costruisco $ \phi \colon \pi_1(X) \to H_1(X) $
  omomorfismo di gruppi ottengo
  gratuitamente la mappa da $ \Ab{\pi_1(X)} $ a $ H_1(X) $ per l'osservazione precedente.
  \[
    \begin{tikzcd}
      \pi_1(X) \arrow{r}{\phi} \arrow{d}{P} & H_1(X) \\
      \Ab{\pi_1(X)} \arrow{ur}{\phi'}
    \end{tikzcd}
  \]
  Poi dovrò mostrare che questa mappa è invertibile, cioè $ \exists \psi:H_1(X) \to A_1(X) $ tale che $ \phi' \circ \psi = \Id{H_1(X)} $ e
  $ \psi \circ \phi' = \Id{\Ab{\pi_1(X)}} $.
\end{proof}

Provo a costruire $ \phi $.
\[
  \begin{aligned}[t]
    \phi:  \pi_1(X) & \to H_1(X) \\
    [f]_H & \mapsto [f]_{hom}
  \end{aligned}
\]
Usando il seguente risultato:
\begin{lemma}
  Se $ f \sim_H g $ allora $ f \sim_{hom} g $, cioè se $ f $ e $ g $ sono lacci che definiscono
  lo stesso elemento nel gruppo fondamentale allora differiscono per un bordo.
\end{lemma}
\begin{proof}
  Siccome $ f \sim_{H} g $ allora $ \exists F $ continua tale $ F: I \times I \to X $ tale che $ F(0,x) = f(x) $,
  $ F(1,x) = g(x) $ e $ F(t,0) = F(t, 1) = x_0 $.
  \begin{figure}[htbp]
    \centering
    \begin{tikzpicture}
      \draw (0, 0) rectangle (3,3);
      \draw[-Latex] (-0.5, 1) -- (-0.5, 2);
      \draw[-Latex] (1, -0.5) -- (2, -0.5);
      \node[left] () at (-0.5, 1.5) {$ t $};
      \node[below] () at (1.5, -0.5) {$ x $};
      \node[right] () at (0, 1.5) {$ C_{x_0} $};
      \node[left] () at (3, 1.5) {$ C_{x_0} $};
      \node[above] () at (1.5, 0) {$ f(x) $};
      \node[below] () at (1.5, 3) {$ g(x) $};
      \draw (0,2) -- (3,2);
      \draw[-Latex] (0, 2) -- (1.5, 2);
      \node[below] () at (1.5, 2) {$ F(t,x) $};
    \end{tikzpicture}
    \caption{Omotopia: deforma $ f $ in $ g $ in modo continuo.}
    \label{fig:lez3:homotopy_f_g}
  \end{figure}

  Voglio mostrare che $ f - g $ bordo di un $ 2 $-simplesso.
  Identificando tutti i punti di un uno dei due intervalli con l'equivalenza $ \quot{I \times I}{\set{0} \times I} $
  si ottiene qualcosa che è omeomorfo a $ \Delta_2 $.
  \begin{figure}[htbp]
    \centering
    \begin{tikzpicture}
      \draw (0, 0) rectangle (3,3);
      % \node[right] () at (0, 1.5) {$ C_{x_0} $};
      % \node[left] () at (3, 1.5) {$ C_{x_0} $};
      \node[above] () at (1.5, 0) {$ f(x) $};
      \node[below] () at (1.5, 3) {$ g(x) $};
      \draw[-Latex] (4,1.5) -- (5,1.5);
      \draw (6,0) -- (7.7, 3) -- (9.4,0) -- cycle;
      \node[above] () at (6.4, 1.3) {$ g(x) $};
      \node[below] () at (7.7, 0) {$ f(x) $};
      \node[left] () at (6,0) {$ e_0 $};
      \node[right] () at (9.4,0) {$ e_1 $};
      \node[above] () at (7.7,3) {$ e_2 $};
    \end{tikzpicture}
    \caption{La relazione di equivalenza fa passare da un quadrato a un triangolo
    in quanto fa collassare un intervallo nel punto 0}
    \label{fig:lez3:homotopy_f_g_to_triangle}
  \end{figure}

  \underline{SISTEMARE}
  Siccome rimane costante sul sottospazio su cui su quozienta, cioè su $ x_0 $,
  $ F $ induce $ F' \colon \Delta_2 \to X $ continua:
  \[
    \begin{tikzcd}
      I \times I \arrow{r}{F} \arrow{d}{P} & X \\
      \quot{I \times I}{0 \times I} \simeq \Delta_2 \arrow{ur}{F'}
    \end{tikzcd}
  \]
  Calcolo il bordo: $ \partial F' = F'^{(0)} - F'^{(1)} + F'^{(2)} = K - g + f $
  dove $ K $ è il cammino costante per definizione di omotopia. Se $ K $
  fosse il bordo di qualcosa avrei finito ($ \partial w = f - g $). Prendo il $ 2 $-simplesso
  standard $ K $ costante e uguale a $ x_0 $ (è la stessa costante di $ K $):
  \[
    \partial K = K^{(0)} - K^{(1)} + K^{(2)}
  \]
  ma questi sono uguali perché sono costanti, quindi $ \partial K = K^{(2)} = k $, cioè $ k $
  è un bordo, quindi:
  \[
    \partial F' = \partial K - F'^{(1)} + F'^{(2)} \Rightarrow \partial F' - \partial K = f - g \Rightarrow \partial(F' - K) = f - g
  \]
  $ F' - K $ è $ 2 $-simplesso singolare, lo chiamo $ \sigma $: $ \partial \sigma = f - g $, quindi
  $ f $ e $ g $ sono omologhi e $ \sigma $ è il $ 2 $-simplesso singolare che realizza
  l'omologia.
\end{proof}

Se $ X $ è uno spazio topologico connesso per archi sono in grado di costruire
senza utilizzare l'ipotesi di connessione per archi:
\begin{align*}
  \phi:  \pi_1(X) & \to H_1(X) \\
  [f]_H & \mapsto [f]_{hom}
\end{align*}
Ora voglio costruire $ \phi' \colon \Ab{\pi_1(X)} \to H_1(X) $ e lo faccio ancora
senza l'ipotesi di connessione per archi. Mostro che $ \phi' $ è omomorfismo,
per far ciò basta che mostro che $ \phi $ lo è.

\begin{proof}
  Siano $ [f]_H, [g]_H \in \pi_1(X) $ voglio fare vedere che:
  \[
    \phi ( [f]_H [g]_H) = \phi([f]_H) + \phi([g]_H)
  \]
  Questo è verso se e solo se:
  \[
    \phi([f \star g]_H) = [f]_{hom} + [g]_{hom}
  \]
  Che è vera se e solo se:
  \[
    [f \star g]_{hom} = [f + g]_{hom}
  \]
  Questo è vero se e solo se i due rappresentati sono equivalenti:
  \[
    \exists T: \Delta_2 \to X \text{ $ 2 $-simplesso singolare tale che } \partial T = f + g - f \star g
  \]
  Cioè:
  \[
    \partial T = T^{(0)} - T^{(1)} + T^{(2)} = f + g - f \star g
  \]
  \begin{figure}[htbp]
    \centering
    \begin{tikzpicture}
      \draw (5,0) -- (9,0) -- (7,3) -- cycle;
      \node[left] () at (5,0) {$ e_0 $};
      \node[right] () at (9,0) {$ e_1 $};
      \node[above] () at (7,3) {$ e_2 $};
      \node[below] () at (7,0) {$ g $};
      \node[below] () at (7,-0.5) {$ T^{(2)} $};
      \node[left] () at (5.95,1.7) {$ f \star g $};
      \node[left] () at (5.05,1.7) {$ T^{(1)} $};
      \node[left] () at (6.2,1.5) {};
      \node[right] () at (8,1.7) {$ f $};
      \node[right] () at (8.5, 1.7) {$ T^{(0)} $};
    \end{tikzpicture}
    \caption{Costruzione dell'omomorfismo}
    \label{fig:lez3:proof_homo_1}
  \end{figure}
  Una possibile costruzione parte tracciando la retta che congiunge
  due punti medi di due segmenti, quindi si richiede che $ T $ abbia
  valori costanti sulle rette parallele.
  \begin{figure}[htbp]
    \centering
    \begin{tikzpicture}
      \draw (5,0) -- (9,0) -- (7,3) -- cycle;
      \node[left] () at (5,0) {$ e_0 $};
      \node[right] () at (9,0) {$ e_1 $};
      \node[above] () at (7,3) {$ e_2 $};
      \node[below] () at (7,0) {$ g $};
      \node[left] () at (5.9,1.5) {$ \frac{1}{2} $};
      \node[left] () at (6.2,1.5) {\textbullet};
      \node[] () at (7,-0.025) {\textbullet};
      \node[right] () at (8,2) {$ f $};
      \draw (6.025, 1.525) -- (7,0);
    \end{tikzpicture}
    \caption{Costruzione dell'omomorfismo, deve avere valori costanti su rette parallele}
    \label{fig:lez3:proof_homo}
  \end{figure}

Quindi è un omomorfismo.
\end{proof}

Al momento la situazione è che ho $ \phi: H_1(X,x_0) \to H_1(X) $ omomorfismo di
gruppi ben definito anche con $ X $ non necessariamente connesso per archi,
e dato che $ H_1(X) $ è abeliano ho $ \phi': \Ab{\pi_1(X)} \to H_1(X) $.

\begin{proposition}
  Se $ X $ è connesso per archi allora la mappa $ \phi': \colon \Ab{\pi_1(X)} \to H_1(X) $
  è un isomorfismo.
\end{proposition}
\begin{proof}
  \emph{Sketch of proof, la dimostrazione completa è piuttosto noiosa}.
  Per dimostrare che $ \phi' $ è isomorfismo o dimostro che è iniettiva e suriettiva
  o che ammette un inverso. Procedo con la seconda possibilità: mostro che
  $ \exists \psi \colon H_1(X) \to \Ab{\pi_1(X)} $ tale che $ \psi $ è inverso di $ \phi' $.
  Considero un arco $ f \colon \Delta_1 \to X $ con $ f(0), f(1) \in X $.
  \begin{figure}[htbp]
    \centering
    \begin{tikzpicture}
      \draw (0,-0.75) rectangle (5.25,3);
      \node[right] () at (5.5,1.5) {$ X $};
      \node[above] () at (1,1) {$ x_0 $};
      \node[] () at (1,1) {\textbullet};
      \node[above] () at (2,2) {$ f(0) $};
      \node[] () at (2,2) {\textbullet};
      \node[above, right] () at (4,1) {$ f(1) $};
      \node[] () at (4,1) {\textbullet};
      \draw[-Latex] (1,1) to [out=-30,in=-50] (2,2);
      \node[right] () at (2,1.6) {$ \lambda_{f(0)} $};
      \draw[Latex-] (1,1) to [out=-60,in=-90] (4,1);
      \node[right] () at (3.9,0.5) {$ \bar{\lambda}_{f(1)} $};
      \draw[-Latex] (2,2) to [out=-30,in=90] (4,1);
    \end{tikzpicture}
    \caption{Dimostrazione della proposizione}
    \label{fig:lez3:sketch_of_proof}
  \end{figure}
  Siccome lo spazio è connesso per archi esiste un cammino da $ x_0 $ a $ f(0) $, cioè
  una funzione $ \lambda_{f(0)} \colon I \to X $ tale che $ \lambda_{f(0)} = x_0 $ e $ \lambda_{f(1)} = f(0) $.
  Lo stesso vale per $ x_0 $ e $ f(1) $. Questi archi sono orientati partendo da $ x_0 $, posso
  considerare il cammino con verso opposto $ \bar{\lambda}_{f(1)} $ e quindi costruire il laccio
  di base $ x_0 $: $ \lambda_{f(0)} \star f \star \bar{\lambda}_{f(1)} =: \tilde{f} $. Vale che $
  \psi(f) = \llbracket \tilde{f} \rrbracket $,
  dove $  \llbracket \tilde{f} \rrbracket = P \left([\tilde{f}]_H\right)$.
  Bisogna mostrare che:
  \begin{enumerate}
  \item $ \psi $ è ben definito, cioè se $ f \sim_{hom} g $ allora $ \psi(f) = \psi(g) $ e che $ \psi $
    non dipende dalla scelta del cammino.
  \item $ \psi $ è omomorfismo di gruppi
  \item $ \phi' \circ \psi = \Id{H_1(X)} $
  \item $ \psi \circ \phi' = \Id{\Ab{\pi_1(X)}} $
  \end{enumerate}
  \emph{Lo studente interessato può verificare queste asserzioni.}
  \begin{exercise}
    Verificarli.
  \end{exercise}
  Una volta verificati si trova quindi che $ H_1(X) \cong \Ab{\pi_1(X)} $.
\end{proof}

\begin{example} \hfill
  \begin{itemize}
  \item $ H_1(V_g) \cong \Z^{2g} $ con $ g \geq 0 $
  \item $ H_1(\bigvee_{i=1}^{k}\Sph{1}) \cong \Z^k $ con $ \bigvee_{i=1}^{k}\Sph{1} $ bouquet, cioè $ k $ circonferenze incollate in un punto.
  \item $ H_1(\RN{3} \setminus \Sph{1}) \cong \Z $ (è un toro tappato)
  \item $ H_1(U_1) \cong \Z_2 $ dove $ U_1 $ è il piano proiettivo reale $ \mathbb{P}^2(\RN{}) = \quot{\RN{3} \setminus \set{0}}{\sim} $
    con $ \vec{x} \sim \vec{y} $ se $ \vec{x} = a \vec{y} $ con $ a \in \RN{} $
  \item $ H_1(U_2) \cong \Z \oplus \Z_2 $ dove $ U_2 $ è la bottiglia di Klein.
    Infatti $ \pi_1(U_2) ] \set{a, b | aba^{-1}b^{-1} = 1} $ per ableliannizzarlo bisogna
    porre $ aba^{-1}b = 1 $ e $ aba^{-1}b^{-1} = 1 $ cioè $ b^2 = 1 $ e $ a $ libero:
    $ \Ab{\pi_1(U_2)} = \set{\underset{\Z}{a}, \underset{\Z_2}{b} |  aba^{-1}b = 1 } $
    \begin{figure}[htbp]
      \centering
      \begin{subfigure}{.5\textwidth}
        \centering
        \def\svgwidth{0.26\textwidth}
        \input{images/Klein_bottle.pdf_tex}
        \caption{Bottiglia di Klein}
      \end{subfigure}%
      \begin{subfigure}{.5\textwidth}
        \centering
        \begin{tikzpicture}
          \draw (0,0) rectangle (3,3);
          \draw[-Latex] (0,0) -- (0,1.5);
          \draw[-Latex] (0,3) -- (1.5,3);
          \draw[-Latex] (3,0) -- (3,1.5);
          \draw[-Latex] (3,0) -- (1.5,0);
          \node[left] () at (0,1.5) {$ a $};
          \node[above] () at (1.5,3) {$ b $};
          \node[right] () at (3,1.5) {$ a $};
          \node[below] () at (1.5,0) {$ b $};
        \end{tikzpicture}
        \caption{Bottiglia di Klein, si nota che rispetto al toro di Clifford c'è
          una torsione nella $ a $ di destra}
      \end{subfigure}
      \caption{Bottiglia di Klein}
      \label{fig:lez3:klein_bottle}
    \end{figure}.
  \end{itemize}
\end{example}

% lezione 4
%  _     _____ ________ ___  _   _ _____   _  _
% | |   | ____|__  /_ _/ _ \| \ | | ____| | || |
% | |   |  _|   / / | | | | |  \| |  _|   | || |_
% | |___| |___ / /_ | | |_| | |\  | |___  |__   _|
% |_____|_____/____|___\___/|_| \_|_____|    |_|

\section{Morfismi indotti}

% Se $ X $ è uno spazio topologico connesso per archi allora esiste
% l'isomorfismo:
% \[
%   \phi \colon \quot{\pi_1(X)}{[ \pi_1(X), \pi_1(X)]} \to H_1(X)
% \]
% Il problema è costruire
% \[
%   \psi \colon H_1(X) \to \quot{\pi_1(X)}{[ \pi_1(X), \pi_1(X)]}
% \]
% Tale che: $ \phi \circ \psi = \Id{H_1(X)} $ e $ \psi \circ \phi = \Id{\pi_1(X)} $.
So calcolare $ H_0(X) $ e $ H_1(X) $ se voglio calcolare
gli altri $ H_k(X) $? Prima guardo come si comportano i gruppi sotto
l'azione di applicazioni continue:
Sia $ g \colon X \to Y $ mappa continua tra spazi topologici,
$ g $ induce un'applicazione tra $ H_k(X) $ e $ H_k(Y) $?
Considero $ \sigma \colon \Delta_k \to X $ $ k $-simplesso singolare, posso
considerare la composizione con $ g $:
\[
  \begin{tikzcd}
    \Delta_k \arrow{r}{\sigma} &  X  \arrow{r}{g} & Y
  \end{tikzcd}
\]
Cioè: $ g' \colon \Delta_k \to Y $ con $ g' = g \circ \sigma $. Siccome sia
$ g $ che $ \sigma $ sono continue allora $ g' $ è continua, quindi è un
$ k $-simplesso singolare in $ Y $.
Si definisce $ g_\sharp $:
\begin{align*}
  g_\sharp \colon S_k(X) & \to S_k(Y) \\
  \sum_\sigma n_\sigma \sigma & \mapsto  \sum_\sigma n_\sigma g' =  \sum_\sigma n_\sigma ( g \circ \sigma )
\end{align*}

Questa mappa è ben definita ed è lineare quindi $ g_\sharp $ è un omomorfismo di
gruppi abeliani che manda $ k $-catene in $ S_k(X) $ in $ k $-catene
in $ S_k(Y) $.
Ora voglio ottenere un'applicazione a livello di omologia singolare,
quindi definisco $ g_\star $.
\begin{align*}
  g_\star \colon H_k(X) & \to H_k(Y) \\
  [c]          & \mapsto [g_\sharp (c)]
\end{align*}
Si dice che $ g $ è \textbf{covariante} perché va da $ X $ a $ Y $,
cioè rispetta il verso della applicazione $ g $.
Devo verificare se questa applicazione è ben definita, cioè non
dipende dalla scelta del rappresentate della classe.
Considero $ d \in S_k(X) $ tale che $ \partial d = 0 $, suppongo
che $ d \sim_{hom} c $, questo vale se e solo se $ [d] = [c] $ con $ \partial c = 0 $,
mi chiedo è vero che $ g_\star([d]) = g_\star([c]) $?
Devo cioè mostrare che $ g_\sharp (d) \sim_{hom} g_\sharp (c) $, ma questo è vero
se e solo se $ \exists \tau \in S_{k+1}(Y) $ tale che $ g_\sharp(d) - g_\sharp (c) = \partial \tau $.
Siccome $ g_\sharp $ è omomorfismo allora deve essere $ g_\sharp (d - c) = \partial \tau $,
ma $ d $ e $ c $ sono omologhi per ipotesi, quindi:
\[
  \exists u \in S_{k+1}(X) \; | \; \partial u = d - c
\]
Quindi $ g_\sharp(\partial u) = g_\sharp (d - c) $, e questo implica che $ [g_\sharp (d)] = [g_\sharp(c)] $, infatti
% vorrei che questo sia implicato $ g_\sharp (\partial u) = g_\sharp (d - c) $.
trovo $ \tau $ a partire da $ u $:
\begin{align*}
  g_\sharp (\partial u) & = g_\sharp \left(\sum_{i = 0}^{k+1} (-)^i u^{(i)}\right) = \sum_{i=0}^{k+1}(-)^i g_\sharp (u^{(i)}) =
              \sum_{i=0}^{k+1}(-)^i g \circ u^{(i)} = \\
            & = \sum_{i = 0}^{k+1} (-)^i g \circ \left( u \circ F_i^{\; k+1} \right) =
              \sum_{i = 0}^{k+1} (-)^i  \left( g \circ u \right) \circ F_i^{\; k+1} = \\
            & = \sum_{i = 0}^{k+1}(-)^i \left( g \circ u \right)^{(i)} = \partial \left( g \circ u \right)
\end{align*}
Ma quindi $ g_\sharp(\partial u) = \partial (g_\sharp (u)) $ cioè:
\[
  g_\sharp(d-c) = g_\sharp (\partial u) = \partial (g_\sharp (u)) = \partial \tau \quad \text{con } \tau = g_\sharp(u)
\]
% In conclusione:
% \begin{align*}
%   g_\star \colon H_k(X) & \to H_k(Y) \\
%   [c]_X & \to [g_\sharp (c)]_Y
%   \end{align*}
Quindi $ g_\star $ è ben definita ed è omomorfismo in quanto
è il passaggio a quoziente di omomorfismi.

\begin{example}
  Sia $ j \colon \Sph{1} \to \Sph{2} $ l'immersione di un equatore in una sfera,
  che cosa è $ j_\star \colon H_1(\Sph{1}) \to H_1(\Sph{2}) $?
  $ j_\star $ è una mappa costante in quanto $ \Sph{2} $ ha gruppo fondamentale banale
  quindi $ H_1(\Sph{2}) $ è banale. Si nota che $ j $ era iniettiva,
  ma $ j_\star $ è costante quindi non è più iniettiva.
\end{example}
\begin{example}
  Se considero $ \Sph{1} = \set{z \in \mathbb{C} | |z| = 1 } $
  \begin{align*}
    f \colon \Sph{1} & \to \Sph{1} \\
    z & \to z^4
  \end{align*}
  Come è fatta $ f_\star \colon H_1(\Sph{1}) \to H_1(\Sph{1})$ ?
  Si sa che $ H_1(\Sph{1}) \cong \Z $, quindi, sia
  \begin{align*}
    \sigma \colon \Delta_1 & \to \Sph{1} \\
    t & \to \me^{2 \pi i t}
  \end{align*}
  Cioè in pratica $ [\sigma] \to 1 $, il laccio si avvolge su sè stesso una volta.
  \begin{align*}
    f_\star \colon H_1(\Sph{1}) & \to H_1(\Sph{1}) \\
    [\sigma] & \mapsto [f_\sharp (\sigma)] = [f \circ \sigma]
  \end{align*}
  Si ha:
  \[
    \begin{tikzcd}
      \Delta_1 \arrow{r}{\sigma} & \Sph{1} \arrow{r}{f} & \Sph{1}
    \end{tikzcd}
  \]
  Con:
  \[
    \begin{tikzcd}
      t \arrow{r}{\sigma} & \me^{2 \pi i t} \arrow{r}{f} & \me^{8 \pi i t}
    \end{tikzcd}
  \]
  Quindi:
  \begin{align*}
    f \circ \sigma \colon \Delta_1 & \to \Sph{1} \\
    t & \mapsto \me^{8 \pi i t}
  \end{align*}
  Sostanzialmente $ f \circ \sigma $ è un cammino in $ \Sph{1} $ ed è
  quindi potenza di $ \sigma $, che è l'unico generatore:
  \[
    f \circ \sigma = \sigma^4 = \sigma \star \sigma \star \sigma \star \sigma
  \]
  Cioè avvolgo il laccio quattro volte, quindi:
  \begin{align*}
    f_\star \colon H_1(\Sph{1}) & \to H_1(\Sph{1}) \\
    [\sigma] & \mapsto [\sigma^4]
  \end{align*}
  Cioè:
  \begin{align*}
    f_\star \colon \Z & \to \Z \\
    1 & \mapsto 4
  \end{align*}
  $ f_\star $ è iniettivo ma non suriettivo (non tutti gli interi sono
  multipli di 4)
\end{example}

% Siano $ X, Y $ spazi topologici a partire da $ f \colon X \to Y $
% ho $ f_\star \colon H_k(X) \to H_k(Y) $ $ \forall k $

\begin{osservation}
  Siano $ X $ spazio topologico:
  $ \Id{X} \colon X \to X $ allora:
  \begin{align*}
    \left(\Id{X}\right)_\star \colon H_k(X) & \to H_k(X) \\
    [c] & \mapsto [\left(\Id{X}\right)_\star (c)] = [c]
  \end{align*}
  Quindi $ \left(\Id{X}\right)_\star $ è proprio l'identità
  a livello di gruppi di omologia, cioè:
  \[
     \left(\Id{X}\right)_\star = \Id{H_k(X)}
  \]
\end{osservation}

\begin{osservation}
  Siano $ X, Y, Z $ spazi topologici e $ f \colon X \to Y $,
  $ g \colon Y \to Z $ funzioni continue, allora $ g \circ f \colon X \to Z $
  è continua, si ha quindi:
  \[
    \begin{tikzcd}
      X \arrow{r}{f} &  Y  \arrow{r}{g} & Z
    \end{tikzcd}
  \]
  E:
  \[
    \begin{tikzcd}
      H_k(X) \arrow{r}{f_\star} &  H_k(Y) \arrow{r}{g_\star} & H_k(Z)
    \end{tikzcd}
  \]
  Sono ben definite $ g_\star \circ f_\star \colon H_k(X) \to H_k(Z) $ e
  $ \left(g \circ f\right)_\star \colon H_k(X) \to H_k(Z) $, vale che
  $ g_\star \circ f_\star =  \left(g \circ f\right)_\star $, infatti se $ \sigma $ è
  simplesso singolare (poi basta estendere per linearlità):
  \begin{gather*}
    \left(g \circ f\right)_\star ([\sigma]) = [ (g \circ f)_\sharp (\sigma) ] = [ (g \circ f) \circ \sigma ] =  [ g \circ (f \circ \sigma) ] = \\
    = [ g_\sharp (f \circ \sigma)] = [ g_\sharp \circ f_\sharp (\sigma) ] = (g_\star \circ f_\star) ([\sigma])
  \end{gather*}
\end{osservation}
Quindi sulla categoria degli spazi topologici questo
fornisce un funtore covariante. Si veda più avanti
cosa significa tutto ciò.

\section{Successioni esatte}

Considero due complessi $ (C_\bullet, \partial) $ e $ (C'_\bullet, \partial') $,
considero l'omomorfismo di $ \Z $-moduli $ F \colon (C_\bullet, \partial) \to (C'_\bullet, \partial') $
tale che $ \forall k $ si formi un diagramma commutativo,
cioè valga $ F \circ \partial = \partial' \circ F $
\[
  \begin{tikzcd}
    \dots \arrow{r}{\partial} &  C_{k+1}  \arrow{r}{\partial} \arrow{d}{F} &  C_{k}  \arrow{r}{\partial} \arrow{d}{F} & C_{k-1}  \arrow{r}{\partial} \arrow{d}{F} & \dots \\
   \dots \arrow{r}{\partial'} &  C'_{k+1}  \arrow{r}{\partial'} &  C'_{k}  \arrow{r}{\partial'}  &  C'_{k-1} \arrow{r}{\partial'} & \dots
  \end{tikzcd}
\]
Tutti i quadrati che si formano devono essere
commutativi. Si pone questa richiesta di commutatività
in quanto considerando $ f \colon X \to Y $ e quindi
$ F = f_\sharp \colon (S_\bullet(X), \partial) \to  (S_\bullet(Y), \partial') $ la condizione
di commutatività è $ f_\sharp \circ \partial = \partial' \circ f_\sharp $ che è
proprio quella che ho utilizzato prima per mostrare
che l'applicazione è ben definita a livello
di omologia (avevo usato $ g_\sharp \circ \partial = \partial \circ g_\sharp $).
Una funzione $ F $ fatta in questo modo è detta
\textbf{mappa tra complessi}\index{Mappa tra complessi}.

\begin{definition}
  Si definisce una \textbf{successione esatta corta}\index{Successione esatta corta} di
  gruppi la successione:
  \[
    \begin{tikzcd}
       A \arrow{r}{\alpha} & B \arrow{r}{\beta} & C
    \end{tikzcd}
  \]
  con $ \alpha $ omomorfismo iniettivo, $ \beta $ omomorfismo suriettivo e $ \ker{\beta} = \im{\alpha} $.
  Si nota che richiedere queste condizioni su $ \alpha $ e $ \beta $ è equivalente a scrivere la
  successione esatta come:
  \[
    \begin{tikzcd}
      0 \arrow{r}{} & A \arrow{r}{\alpha} & B\arrow{r}{\beta} & C \arrow{r}{} & 0
    \end{tikzcd}
  \]
  Infatti indicando le mappe sottointese con $ i \colon 0 \to A $ e $ j \colon C \to 0 $
  allora per l'esattezza vale che $ \ker{\alpha} = \im{i} = 0 $ in quanto $ i $ è
  omomorfismo, ma $ \ker{\alpha} = 0 $ signfiica che $ \alpha $ è iniettivo, inoltre
  $ \ker{j} = \im{\beta} = C $, quindi $ \beta $ è suriettiva.
\end{definition}

\begin{definition}
  Si definisce una \textbf{successione esatta corta}\index{Successione esatta corta} di
  complessi la successione:
  \[
    \begin{tikzcd}
      0 \arrow{r}{} & A_\bullet \arrow{r}{\alpha} & B_\bullet \arrow{r}{\beta} & C_\bullet \arrow{r}{} & 0
    \end{tikzcd}
  \]
  con $ (A_\bullet, \partial^A) $, $ (B_\bullet, \partial^B) $ e $ (C_\bullet, \partial^C) $ complessi, e
  $ \alpha $ mappa tra complessi iniettiva, $ \beta $ mappa tra complessi suriettivo
  e deve valere che $ \forall k $ sia $ C_k \cong \quot{B_k}{A_k} $.
\end{definition}

% lezione 4 parte 2

In modo più esteso questo significa:
\[
  \begin{tikzcd}
    {} & 0 \arrow{d}{} & 0 \arrow{d}{} & 0 \arrow{d}{} & {} \\
    \dots \arrow{r}{} & A_{k+1} \arrow{r}{} \arrow{d}{} & A_{k} \arrow{r}{} \arrow{d}{} & A_{k-1} \arrow{r}{} \arrow{d}{} & \dots \\
    \dots \arrow{r}{} & B_{k+1} \arrow{r}{} \arrow{d}{} & B_{k} \arrow{r}{} \arrow{d}{} & B_{k-1} \arrow{r}{} \arrow{d}{} & \dots \\
    \dots \arrow{r}{} & C_{k+1} \arrow{r}{} \arrow{d}{} & C_{k} \arrow{r}{} \arrow{d}{} & C_{k-1} \arrow{r}{} \arrow{d}{} & \dots \\
    {} & 0 & 0 & 0 & {}
  \end{tikzcd}
\]
Le colonne sono successioni esatte corte di $ Z $-moduli, quindi
l'immagine di $ \alpha $ è uguale al nucleo e la mappa è iniettiva
perciò la prima riga è formata da zero (infatti se è
iniettiva il nucleo è zero), similmente siccome
la mappa $ \beta $ è suriettiva quindi l'ultima
riga è formata da zero.
Inoltre tutti i quadrati sono commutativi.
\subsection{Omomorfismo di connessione}
A partire da una successione esatta corta
posso passare all'omologia, se passo brutalmente
all'omologia non ottengo una successione esatta,
ma c'è il modo per indurre una successione esatta lunga:
\begin{theorem}
  Una successione esatta corta di complessi induce una successione
  esatta lunga tale che sia fatta così:
  \[
    \begin{tikzcd}
      \dots \arrow{r}{} & H_p(A_\bullet) \arrow{r}{\alpha_\star} & H_p(B_\bullet) \arrow{r}{\beta_\star} & H_p(C_\bullet) \arrow{r}{\delta}
      & H_{p-1}(A_\bullet) \arrow{r}{\alpha_\star} & \dots
    \end{tikzcd}
  \]
  Quindi $ \forall p $:
  \begin{gather*}
    \im{\alpha_\star} = \ker{\beta_\star} \\
    \im{\beta_\star} = \ker{\delta} \\
    \im{\delta} = \ker{\alpha_\star}
  \end{gather*}
  $ \delta $ è detto \textbf{omomorfismo di connessione}\index{Omomorfismo di connessione}
  in quanto cambia il grado dell'omologia.

  La scrittura estesa della successione è:
\[
  \begin{tikzcd}
    {} & \dots  \arrow{d}{} &  \dots  \arrow{d}{}  &  \dots  \arrow{d}{}  & {} \\
    \dots \arrow{r}{} & H_{p+1}(C_{k+1}) \arrow{r}{} \arrow{d}{} &  H_{p+1}(C_{k}) \arrow{r}{} \arrow{d}{} &  H_{p+1}(C_{k-1}) \arrow{r}{} \arrow{d}{} & \dots \\
    \dots \arrow{r}{} & H_p(A_{k+1}) \arrow{r}{} \arrow{d}{} & H_p(A_{k})  \arrow{r}{} \arrow{d}{} & H_p(A_{k-1})  \arrow{r}{} \arrow{d}{} & \dots \\
    \dots \arrow{r}{} & H_p(B_{k+1}) \arrow{r}{} \arrow{d}{} & H_p(B_{k})  \arrow{r}{} \arrow{d}{} & H_p(B_{k-1})  \arrow{r}{} \arrow{d}{} & \dots \\
    \dots \arrow{r}{} & H_p(C_{k+1}) \arrow{r}{} \arrow{d}{} & H_p(C_{k})  \arrow{r}{} \arrow{d}{} & H_p(C_{k-1})  \arrow{r}{} \arrow{d}{} & \dots \\
    \dots \arrow{r}{} & H_{p-1}(A_{k+1}) \arrow{r}{} \arrow{d}{}  &  H_{p-1}(A_{k}) \arrow{r}{} \arrow{d}{} &  H_{p-1}(A_{k-1}) \arrow{r}{} \arrow{d}{} & \dots \\
    {} & \dots &  \dots &  \dots & {}
  \end{tikzcd}
\]
\end{theorem}
\begin{proof}
  Per dimostrare il teorema bisogna:
  \begin{enumerate}
  \item Dimostrare che $ \alpha_\star $ e $ \beta_\star $ sono ben definite
  \item Costruire l'omomorfismo di connessione e verificare che è effettivamente un omomorfismo
  \item Mostare che la successione è esatta, cioè che
    \begin{gather*}
      \im{\alpha_\star} = \ker{\beta_\star} \\
      \im{\beta_\star} = \ker{\delta} \\
      \im{\delta} = \ker{\alpha_\star}
    \end{gather*}
  \end{enumerate}
  \emph{Sketch of proof, la dimostrazione è lunga e noiosa.}

  Sia $ c \in C_k $ un ciclo, quindi tale che $ \partial c = 0 $,
  siccome $ \beta_k $ è suriettiva $ \exists b \in B_k $ tale che
  $ \beta_k(b) = c $, voglio recuperare un elemento in $ A_{k-a} $.
  \[
    \begin{tikzcd}
      {} & a \in A_{k-1} \arrow{d}{\partial b} \\
      b \in B_k \arrow{r}{\partial} \arrow{d}{\beta_k} & B_{k-1} \arrow{d}{\beta_{k-1}} \\
      c \in B_k \arrow{r}{\partial} & C_{K-1}
    \end{tikzcd}
  \]
  Usando la commutataività:
  \[
    \beta_{k-1}(\partial b) = \partial \beta_k (b) = \partial c = 0
  \]
  Quindi $ \beta_{k-1}(b) = 0 $, e quindi $ \partial b \in \ker{\beta_{k-1}} $, ma
  le colonne sono esatte quindi $ \partial b \in \im{\alpha_{k-1}} = \ker{\beta_{k-1}} $,
  perciò $ \exists a \in A_{k-1} $ tale che $ \alpha_{k-1}(a) = \partial b $, quindi
  a partire da $ c \in C_k $ ho associato un elemento $ a \in A_{k-1} $.
  Per scendere a livello di omologia $ a $ deve essere un ciclo,
  cioè $ \partial a = 0 $, ma per la commutatività:
  \[
    \alpha_{k-2}(\partial a) = \partial \alpha_{k-1}(a) = \partial \partial b = 0
  \]
  Ma $ \alpha_{k-2} $ è iniettiva, quindi $ \partial a = 0 $.
  Sono partito da un $ k $-ciclo in $ C_k $ e
  ho trovato un $ k-1 $-ciclo in $ A_{k-1} $.

  Ci sono un paio di dettagli da verificare:
  \begin{enumerate}
  \item È univoca la scelta dell'elemento $ b $? Se non lo è ci sono
    problemi?
  \item Se prendo un elemento $ c' $ che è omologo
    a $ c $ è sicuro che trovo un $ a' $ che è
    omologo ad $ a $?
  \end{enumerate}
  Se queste due cose non sono verificate l'applicazione non è
  ben definita.
  Verifico che comunque scelga una controimmagine di $ \beta_k $ si ottiene
  qualcosa di omologo ad $ a $: suppongo che $ \beta_k (b') = \beta_k (b) = c $:
  \[
    \beta_k(b' - b) = 0 \iff b' - b \in \ker{\beta_k} = \im{\alpha_k}
  \]
  Quindi esiste $ a_0 \in A_k $ tale che $ \alpha_k(\alpha_0) = b' - b $, prendendo
  il bordo:
  \[
    \partial ( b' - b ) = \partial ( \alpha_k (a_0 )) \Rightarrow \partial b' - \partial b = (\partial \circ \alpha_k)(a_0) = \alpha_{k-1}(\partial a_0)
  \]
  Quindi:
  \[
    \alpha_{k-1}(a') - \alpha_{k-1}(a) = \alpha_{k-1}(\partial a_0) \Rightarrow \alpha_{k-1}(a' - a - \partial a_0) = 0
  \]
  Ma $ \alpha_{k-1} $ è iniettivo quindi $ a' - a - \partial a_0 = 0 $, e perciò  $ a' \sim_{hom} a $,
  in quanto $ a $e $ a' $ differiscono per un bordo.
  Per quanto riguarda la seconda questione considero $ c'' \sim_{hom} c $ in $ C_k $
  allora mostro che $ a'' \sim_{hom} a $ in $ A_{k-1} $, e così facendo
  mostro che l'applicazione è ben definita.
  \[
    c'' \sim_{hom} c \iff \exists c_0 \; | \; c'' - c = \partial c_0
  \]
  Ma per la suriettività $ \exists b, b'' $ tale che $ c = \beta_k(b) $ e
  $ c'' = \beta_k(b'') $, quindi:
  \[
    \beta_k(b'') - \beta_k(b) = \partial c_0 \Rightarrow \beta_k(b'' - b) = \partial c_0  \overset{\text{commutatività}}{\Rightarrow}
    \beta_k(b''-b) = \partial \beta_{k+1}(b_0) = \beta_{k}(\partial b_0)
  \]
  Quindi:
  \[
    \beta_k(b'' - b - \partial b_0) = 0 \Rightarrow b'' - b - \partial b_0 \in \ker{\beta_k} = \im{\alpha_k}
  \]
  Perciò $ \exists \tilde{a} \in A_k$ tale che $ b'' - b - \partial b_0 = \alpha_k (\tilde{a}) $
  Apllicando il bordo: $ \partial b'' - \partial b - \partial \partial \alpha_k(\tilde{a}) $, quindi:
  \[
    \partial b'' - \partial b = \partial \alpha_{k}(\tilde{a}) \Rightarrow \alpha_{k+1}(a'') - \alpha_{k-1}(a) = \alpha_{k-1}(\partial \tilde{a})
  \]
  Ma $ \alpha_{k-1} $ è omomorfismo iniettivo quindi
  $ a'' - a - \partial \tilde{a} = 0 $ cioè $ a'' - a = \partial \tilde{a} $

  Si può quindi definire $ \delta $ su $ [c] \in H_p(C_k) $:
  \[
    \delta([c]) = \llbracket \alpha \circ \partial \circ \beta^{-1}(c) \rrbracket
  \]
  Questa è ben definita.
\end{proof}

\section{Omologia singolare relativa}
Sia $ X $ uno spazio topologico e $ A $ sottospazio generico
di $ X $ (anche improprio), cioè $ A \incl X$.
Vorrei definrie l'omologia
singolare di $ X $ tenendo presente la presenza di $ A $,
cioè $ H_k(X,A) $, il $ k $-esimo gruppo di omologia
singolare dellla coppia $ (X, A) $.
Sia $ S_K(A) $ lo spazio delle $ k $-catene che finiscono
totalmente in $ A $, la mappa di inclusione $ i \colon A \to X $
induce una mappa $ i_\sharp \colon s_k(A) \to S_k(X) $. Questa mappa
è sicuramente iniettiva (basta vedere le catene di $ A $ come
catene di $ X $). A questo punto la successione
\[
  \begin{tikzcd}
    0 \arrow{r}{} & S_k(A) \arrow{r}{i_\sharp} & S_k(X) \arrow{r}{\beta} & \quot{S_k(X)}{S_k(A)} \arrow{r}{} & 0
  \end{tikzcd}
\]
è esatta, e anche la restrizione:
\[
  \begin{tikzcd}
    0 \arrow{r}{} & S_k(A) \arrow{r}{i_\sharp} & S_k(X) \arrow{r}{\beta} & S_k(X,A) \arrow{r}{} & 0
  \end{tikzcd}
\]
Posso fare:
\[
  \begin{tikzcd}
    {} & 0 \arrow{d}{} & 0 \arrow{d}{} & {} \\
    \dots \arrow{r}{} & S_{k+1}(A) \arrow{r}{} \arrow{d}{} & S_k(A) \arrow{r}{} \arrow{d}{} & \dots \\
    \dots \arrow{r}{} & S_{k+1}(X) \arrow{r}{} \arrow{d}{} & S_k(X) \arrow{r}{} \arrow{d}{} & \dots \\
    \dots \arrow{r}{} & S_{k+1}(X,A) \arrow{r}{} \arrow{d}{} & S_k(X,A) \arrow{r}{} \arrow{d}{}& \dots \\
    {} & 0 & 0 & {}
  \end{tikzcd}
\]
I quadrati sono commutativi quindi posso
costruisre una successione esatta lunga.
Si ottiene quindi:
\[
  \begin{tikzcd}
    \dots \arrow{r}{} & H_k(A) \arrow{r}{\alpha_\star} & H_k(B) \arrow{r}{\beta_\star} & H_k(X,A) \arrow{r}{\delta} & H_{k-1}(A) \arrow{r}{} & \dots
  \end{tikzcd}
\]
$ H_k(X,A) $ è l'\textbf{omologia singolare della coppia}\index{Omologia singolare relativa}\index{Omologia singolare della coppia ! \vedi{Omologia singolare relativa}}

Questa è una costruzione geometrica e fornice informazioni su nucleo e conucleo.

Consudero una successione esatta corta di $ \R $-moduli:
\[
  \begin{tikzcd}
    0 \arrow{r}{} & A \arrow{r}{\alpha} & B \arrow{r}{\beta} & C \arrow{r}{} & 0
  \end{tikzcd}
\]
Con $ \alpha $ iniettiva, $ \beta $ suriettiva e $ \quot{B}{A} \cong C $, cioè
$ \im{\alpha} = A = \ker{\beta} $ in quanto $ \beta $ è suriettiva.
\begin{definition}
  Si dice che la \textbf{successione spezza}\index{Successione spezza} se esiste una
  funzione continua $ \phi \colon B \to B $ idempotente, cioè tale che $ \phi^2 = \phi $, e tale che
  $ \ker{\phi} = \im{\alpha} = \ker{\beta} $.
\end{definition}

\begin{example}
  [MANCA ESEMPIO]
\end{example}

% lezione 5


Sia $ B = A \oplus C $ con $ A, C $ $ \Z $-moduli, a questi sono associate la mappa
di inclusione e di passaggio al quoziente:
\begin{align*}
  i \colon A & \to A \oplus C \\
  a & \mapsto (a, 0)
\end{align*}
\begin{align*}
  j \colon A \oplus C & C
\end{align*}
Con $ j $ quoziente per lo $ \Z $-modulo $ i(A) $. Ovviamente si possono
scegliere $ i' $ e $ j' $ in cui si scambia il ruolo di $ A $ e di $ C $.
Si nota che $ i $ è iniettiva e $ j $ è suriettiva.
Ho quindi:
\[
  \begin{tikzcd}
    0 \arrow{r}{} & A \arrow{r}{i} & B \arrow{r}{j} & C \arrow{r}{} & 0
  \end{tikzcd}
\]
Ma esiste anche $ s \colon C \to B $ e quindi ho;
\[
  \begin{tikzcd}
    C \arrow{r}{s} & A \oplus C \arrow{r}{j} & C
  \end{tikzcd}
\]
Con $ s \circ j \colon c \mapsto (0,c) \mapsto c $, mi piacerebbe che $ s \circ j = \Id{C} $.
La mappa $ s $ è detta \textbf{sezione dell'omomorfismo}\index{Sezione dell'omomorfismo}
$ j \colon B \to C $.
Quindi se $ B $ è proprio somma diretta ho automaticamente $ s $ e $ s' $ con $ s' $
quoziente. Questo è il prototipo di successione che spezza.

\begin{definition}
  Siano $ A, B, C $ $ \Z $-moduli con $ \ker{\alpha} = 0 $, $ \im{\beta} = C$ e $ \ker{\beta} = \im{\alpha} $,
  cioe una successione esatta, si dice che la successione
  \[
    \begin{tikzcd}
      0 \arrow{r}{} & A \arrow{r}{\alpha} & B \arrow{r}{\beta} & C \arrow{r}{} & 0
    \end{tikzcd}
  \]
  \textbf{spezza}\index{Successione spezza} se esiste una sezione da $ B $ a $ C $
  o da $ B $ ad $ A $, cioè:
  \begin{gather*}
    \exists s \colon B \to C \text{ tale che } s \circ \beta = \Id{C} \\
    \text{oppure} \\
    \exists s' \colon B \to A \text{ tale che } \alpha \circ s' = \Id{A}
  \end{gather*}
\end{definition}

Questo è equivalente a dire che $ B = A \oplus s(C) $, infatti vale l'osservazione
\begin{osservation}
  Se la successione $ 0 \to A \to B \to C \to 0 $ spezza allora $ B = A \oplus s(C) $ con $ s $ sezione.
  Il viceversa l'ho già dimostrato.
\end{osservation}
\begin{proof}
  Vale che $ A \incl B $ in quanto $ \alpha $ è iniettiva e quindi $ \alpha(A) \cong A $, inoltre
  $ s(C) \incl B $ in quanto $ s \colon C \to B $ è la sezione.
  Sia $ x \in \alpha(A) \cap s(C) $, cioè $ x \in \alpha(A) $ e $ x \in s(C) $ allora
  esiste $ a \in A $ tale che $ x = \alpha(a) $ ed esiste $ k \in C $ tale che
  $ x = s(k) $, vale che $ \alpha(a) = s(k) $. Applicando $ \beta \colon B \to C $ si ottiene
  $ (\beta \circ \alpha) (a) = (\beta \circ s)(k) $, ma $ \beta \circ \alpha = 0$ in quanto $ \ker{\beta} = \im{\alpha} $, quindi
  $ (\beta \circ s)(k) = 0 $. Ma $ s $ è sezione quindi $ \beta \circ s = \Id{C} $, da cui $ k = 0 $.
  $ s $ è omomorfismo quindi $ s(k) = 0 $, perciò $ x = 0 $.

  Ogni elemento di $ B $si scrive come somma di elemento di $ \alpha(A) $ e di un elemento
  di $ s(C) $. Sia $ b \in B $ applicando $ \beta $ si ottiene $ \beta(b) \in C $, ci sono due
  possibilità:
  \begin{enumerate}
  \item $ \beta(b) = 0 $ quindi $ b \in \ker{\beta} = \im{\alpha} $, quindi $ b \in \im{a} $, cioè
    $ \exists \alpha \in A $ tale che $ b = \alpha(a) $ e quindi si scrive come elemento di $ A $ sommato
    a zero.
  \item $ \beta(b) = c \not = 0 $. Vorrei scrivre $ b = x + y $ con $ x \in \alpha(A) $ e $ y \in s(C) $.
    Considero $ s(t) $ con $ s(t) \in B $, allora $ b - s(t) \in B $, allora
    mostro che $ b - s(t) \in \ker{\beta} $ e quindi posso usare lo stesso ragionamento
    di prima.
    \[
      \beta(b - s(t)) = \beta(b) - \beta(s(t)) = t - t = 0 \Rightarrow \beta(b - s(t)) \in \ker{b} = \im{\alpha}
    \]
    Quindi esiste $ a' \in A $ tale che $ \alpha(a') = b - s(t) $ e quindi $ b = s(t) + \alpha(a') $
  \end{enumerate}
\end{proof}

\begin{example}
  Considero la successione:
  \[
    \begin{tikzcd}
      0 \arrow{r}{} & n \Z \arrow{r}{\alpha} & \Z \arrow{r}{\beta} & \quot{\Z}{n\Z} \arrow{r}{} & 0
    \end{tikzcd}
  \]
  Questa successione è esatta ma non spezza, infatti se spezzasse esisterebbe una sezione:
  \[
    s \colon \quot{\Z}{n\Z} \to \Z
  \]
  Ma questa non può esistere per queztioni di immagini.
\end{example}

\begin{proposition}
  Le due definizioni di successione che spezza sono equivalenti, cioè
  se $ \exists s \colon C \to B $ tale che $ s \circ \beta = \Id{C} $ allora $ \exists \phi \colon B \to B $ tale che sia
  idempotente e che $ \ker{\phi} = \ker{\beta} $
\end{proposition}
\begin{proof}
  Una possibile costruzione è $ \phi = s \circ \beta \not = \Id{C} $, infatti:
  \[
    \phi^2 = s \circ \beta \circ s \circ \beta = s \circ \Id{C} \circ \beta = s \circ \beta = \phi
  \]
  Quindi $ \phi $ è idempotente. Inoltre:
  \[
    \ker{\phi} = \ker{s \circ \beta} = \set{ b \in B | (s \circ \beta)(b) = 0}
  \]
  Quindi $ s(\beta(b)) = 0 $ cioè $  \beta \circ s \circ \beta (b) = 0 $ quindi $ \beta(b) = 0 $ che significa
  che $ b \in \ker{\beta} $. Deve essere $ \ker{\phi} = \ker{\beta} $. Siccome
  $ \ker{\beta} \subseteq \ker{s \circ \beta} \subseteq \ker{\beta} $ allora $ \ker{s \circ \beta} = \ker{\beta} $.
  Rimane da mostrare il viceversa.
\end{proof}

\section{Omologia singolare ridotta}

Fin ora ho parlato di omologia singolare $ H_k(X) $, omologia singolare relativa
$ H_k(X,A) $, ora introduco l'omologia singolare ridotta.

\begin{definition}
  Sia $ X $ uno spazio topologico e $ A = \set{x_0 \in X} $, è ben definita l'omologia
  relativa $ H_k(X, A) $, si definisce questa come \textbf{omologia singolare ridotta}\index{Omologia singolare ridotta}
  $ \tilde{H}_k(X) $.
  L'omologia singolare ridotta è l'omologia relativa ad un punto.
\end{definition}

Per costruire l'omologia singolare ridotta servono le $ k $-catene in $ X $ e le
$ k $-catene in $ \set{x_0} $
\[
  \begin{tikzcd}
    0 \arrow{r}{} & S_k(\set{x_0}) \arrow{r}{} & S_k(X) \arrow{r}{} & \quot{S_k(X)}{S_k({\set{x_0}})} = S_k(X, \set{x_0}) \arrow{r}{} & \dots
  \end{tikzcd}
\]
In $ S_k $ $ \sigma \colon \Delta_k \to \set{x_0} $ è simplesso sono le applicazioni costanti
dal $ k $-simplesso standard in $ \set{x_0} $. Quindi $ S_k(\set{x_0}) = \langle\sigma_k\rangle $
$ \sigma_k $ è l'unica mappa che c'è.

\begin{lemma}[Omologia di un punto]
  Sia $ X = \set{x_0} $ con $ x_0 \in X $, allora:
  \[
    H_k(\set{x_0}) \cong
    \begin{cases}
      \Z & \text{se } k = 0 \\
      0 & \text{se } k \geq 1
    \end{cases}
  \]
\end{lemma}

\begin{proof}
  Il generico $ k $-simplesso singolare in $ X $ è una mappa
  continua $ \sigma_k \colon \Delta_k \to \set{x_0} $, quindi fissato $ k $ esiste
  un solo simplesso singolare, che è la mappa costante dal simplesso
  standard a $ x_0 $. Il generico $ S_k(X) $ quindi è il gruppo libero
  generato da questo simplesso singolare, cioè $ S_k(X) = \langle\sigma_k\rangle $.
  A questo punto fissato $ k $ si può computare semplicemente il bordo di $ \sigma_k $:
  % infatti dalla definizione di omologia singolare c'è il complesso:
  % \[
  %   \begin{tikzcd}
  %      \dots \arrow{r}{}  & S_{k+1}(\set{x_0}) \arrow{r}{\partial} &  S_{k}(\set{x_0}) \arrow{r}{\partial} &  S_{k-1}(\set{x_0}) \arrow{r}{} & \dots
  %   \end{tikzcd}
  % \]
  % Che corrisponde alla successione dei generatori:
  % \[
  %   \begin{tikzcd}
  %      \dots \arrow{r}{}  & \langle\sigma_{k+1}\rangle \arrow{r}{\partial} &  \langle\sigma_{k}\rangle \arrow{r}{\partial} & \langle\sigma_{k-1}\rangle \arrow{r}{} & \dots
  %   \end{tikzcd}
  % \]
  \[
    \partial \sigma_k = \sum_{i = 0}^k (-)^i \sigma_k^{(i)}  \text{ con }  \sigma_k^{(i)} \colon \Delta_{k-1} \overset{F_k^{\; i}}{\to} \Delta_k \overset{\sigma_k}{\to} \set{x_0}
    \text{ cioè } \sigma_k^{(i)}  = \sigma_{k-1}
  \]
  Fissato $ k $ nella sommatoria che calcola il bordo tutte le quantita sono uguali,
  quindi la somma a segni alterni è nulla oppure è uguale a $ \sigma_{k-1} $ a seconda
  della parità di $ k $.
  \[
    \partial \sigma_k =
    \begin{cases}
      0       & \text{se $ k $ dispari} \\
      \sigma_{k-1} & \text{se $ k $ pari}
    \end{cases}
  \]
  A questo punto si può calcolare facilmente il nucleo e l'immagine dell'operatore
  bordo:
  \[
    \ker{\partial_k} =
    \begin{cases}
      0       & \text{$ k $ dispari} \\
      S_k(X)  & \text{$ k \geq 2$ pari}
    \end{cases}
  \]
  E:
  \[
    \im{\partial_{k+1}} =
    \begin{cases}
      0       & \text{$ k $ dispari} \\
      S_k(X)  & \text{$ k \geq 2$ pari}
    \end{cases}
  \]
  Infatti, se $ k \geq 2 $ ed è pari:
  \begin{align*}
    \partial_k \colon S_k(\set{x_0}) & \to S_{k-1}(\set{x_0}) \\
    \sigma_k & \mapsto \sigma_{k-1}
  \end{align*}
  quindi solo lo $ 0 $ è mandato in $ 0 $, mentre se è dispari:
  \begin{align*}
    \partial_k \colon S_k(\set{x_0}) & \to S_{k-1}(\set{x_0}) \\
    \sigma_k & \mapsto 0
  \end{align*}
  quindi tutto viene mandato in $ 0 $
  Invece per $ k $ pari:
  \begin{align*}
    \partial_{k+1} \colon S_{k+1}(\set{x_0}) & \to S_{k}(\set{x_0}) \\
    \sigma_{k+1} & \mapsto \sigma_{k}
  \end{align*}
  quindi l'immagine è il generatore, cioè tutto $ S_{k}(X) $, mentre per $ k \geq 2$ pari:
  \begin{align*}
    \partial_k \colon S_k(\set{x_0}) & \to S_{k-1}(\set{x_0}) \\
    \sigma_k & \mapsto 0
  \end{align*}
  Quindi l'immagine è solo $ 0 $.

  A questo punto se $ k \geq 2 $ $ \im{\partial_{k+1}} = \ker{\partial_k} $, quindi:
  \[
    H_k(X) = \quot{\ker{\partial_{k}}}{\im{\partial_{k+1}}} \cong 0
  \]
  Invece se $ k = 0 $ vale che $ \ker{\partial_0} = S_0(X) $, mentre
  $ \im{\partial_1} = 0 $ quindi:
  \[
    \quot{\ker{\partial_{0}}}{\im{\partial_{1}}} \cong S_0(X)
  \]
\end{proof}

\begin{proposition}
  Vale che:
  \[
    \tilde{H}_k(X) \cong
    \begin{cases}
      \quot{H_0(X)}{\Z} & \text{se } k = 0 \\
      H_k(X) & \text{se } k \geq 1
    \end{cases}
  \]
\end{proposition}
\begin{proof}
  Per dimostrarlo uso la succession esatta lunga in omologia relativa:
  \[
    \begin{tikzcd}
      \dots \arrow{r}{} & H_{k+1}(\set{x_0}) \arrow{r}{} & H_{k+1}(X) \arrow{r}{} & \tilde{H}_{k+1}(X) \arrow{r}{} & H_k(\set{x_0}) \arrow{r}{} & \dots
    \end{tikzcd}
  \]

  Nel caso $ k \geq 1 $ sicuramente $ k + 1 > 0 $ e il complesso quindi è:
  \[
    \begin{tikzcd}
      0 \arrow{r}{}  & H_{k+1}(X) \arrow{r}{\psi} &  \tilde{H}_{k+1}(X) \arrow{r}{} &  0
    \end{tikzcd}
  \]
  $ \psi $ è iniettiva quindi $ \ker{\psi} = \im{0} = 0 $, ma è anche surietta, in quanto [MANCA].
  Quindi $ \psi $ è isomorfismo e perciò $ H_m(X) \cong \tilde{H}_m(X) $ per $ m \geq 2 $.
  Mi rimane da calcolare il caso $ k = 1 $ e il caso $ k = 0 $.
  [MANCA MANCA MANCA]


  Quindi $ \ker{i_\star} = 0 $ da cui $ \im{j} = \ker{i_\star} = 0 $ e perciò $ \phi $ è iniettiva ma anche
  suriettiva. Ho quindi la successione:
  \[
    \begin{tikzcd}
       0 \arrow{r}{}  & H_1(X) \arrow{r}{\phi} &  \tilde{H}_1(X) \arrow{r}{} & 0
    \end{tikzcd}
  \]
  $ \phi $ è isomorfismo quindi $ H_1(X) \cong \tilde{H}_1(X) $, ma vale anche che:
  \[
    \begin{tikzcd}
       0 \arrow{r}{}  & H_0(\set{x_0}) \arrow{r}{i_\star} &  H_0(X) \arrow{r}{} & \tilde{H}_0(X) \arrow{r}{} & 0
    \end{tikzcd}
  \]
  Quindi $ \quot{H_0(X)}{H_0(\set{x_0})} \cong \tilde{H}_0(X) $ infatti $ \quot{H_0(X)}{\ker{\tau}} \cong \im{\tau} $
  per il teorema fondamentale dell'isomorfismo. E infine $ \im{\tau} = \tilde{H}_0(X) $ per la suriettività
  e $ \ker{\tau} = \Z $ per l'iniettività.

  Se voglio mostrare che $ H_0(X) \cong \tilde{H}_0(X) \oplus \Z $ basta che mostro che esiste una sezione, e in
  molti casi questo è vero.
\end{proof}

\begin{example}
  Considero ad esempio $ H_k(\Sph{n}) $ con $ n \geq 1 $:
  \[
    H_k(\Sph{n}) \cong
    \begin{cases}
      \Z & \text{se } k \in \set{0,n} \\
      0 & \text{se } k \not \in \set{0,n}
    \end{cases}
  \]
  Fin ora so che:
  \[
    H_1(\Sph{n}) \cong
    \begin{cases}
      \Z & \text{se } n = 1 \\
      0 & \text{se } n \geq 2
    \end{cases}
  \]
  E che $ H_0(\Sph{n}) \cong \Z $ per $ n \geq 1 $, vorrei calcolare gli altri gruppi di omologia,
  ma per farlo mi servono altru strumenti.
\end{example}

\section{Assioni di una teoria omologica}

\begin{definition}[Teoria omologica secondo Eilenberg e Steenrod]
  Una \textbf{teoria omologica}\index{Teoria omologica}\index{Steendord ! \vedi{Teoria omologica}}\index{Eilenberg ! \vedi{Teoria omologica}}
  sulla categoria di tutte le coppie di spazi topologici e mappe continue è
  un funtore che assegna ad ogni coppia di spazi $ (X, A) $ un gruppo
  abeliano $ H_p(X, A) $ e ad ogni applicazione continua $ f \colon (X, A) \to (Y, B) $
  un omomorfismo $ f_\star \colon H_k(X, A) \to H_k(Y, B) $ con una trasformazione
  naturale $ \delta_k \colon H_k(X, A) \to H_{k-1}(A) := H_{k-1}(A, \emptyset) $, detta \textbf{omomorfismo di connessione}\index{Omomorfismo di connessione}
  tale che siano soddisfatti i seguenti assioni:
  \begin{enumerate}
  \item (Omotopia): se $ f \sim_H g $ con $ f, g \colon (X, A) \to (Y, B) $ mappe continue, allora $ f_\star = g_\star$.
    Dove $ f \sim_H g $ se esiste una funzione continua $ F \colon X \times I \to Y $ tale che $ F(x,0) = f(x) $,
    $ F(x, 1) = g(x) $ e $ F(a, t) \subseteq B $ $ \forall a \in A $ e $ \forall t \in I $.
  \item (Esattezza): Per ogni inclusione $ i \colon A \incl X $ e $ j \colon X \incl (X, A) $ la successione:
    \[
      \begin{tikzcd}
        \dots \arrow{r}{}  & H_p(A) \arrow{r}{i_\star} &  H_p(X) \arrow{r}{j_\star} &  H_p(X,A) \arrow{r}{\delta_p} & H_{p-1}(A) \arrow{r}{} & \dots
      \end{tikzcd}
    \]
    è esatta.
  \item (Dimensione): $ H_k (P) = 0 $ $ \forall k \not = 0 $ dove $ P $ è lo spazio formato da un solo punto.
  \item (Additività): Se $ X $ è la somma topologica di spazi $ X_\alpha $ allora $ H_p(X) = MANCA BIG \oplus H_p(X_\alpha) $
  \item (Escissione): Se $ U $ è un aperto in $ X $ tale che $ \bar{U} \subset \mathrm{int}(A) $ allora l'inclusione
    di $ (X \setminus U, A \setminus U) $ in $ (X, A) $ induce un isomorfismo tra $ H_k $, cioe togliendo
    un opportuno insieme da $ (X,A) $ l'omologia non sente della escissione.
  \end{enumerate}
  Per trasformazione naturale si intende che $ \forall f \colon (X, A) \to (Y, B) $ il seguente diagramma è commutativo:
  \[
    \begin{tikzcd}
      H_p(X,A) \arrow{r}{\delta} \arrow{d}{f_\star} & H_{p-1}(A) \arrow{d}{f'_\star} \\
      H_p(Y,B) \arrow{r}{\delta} & H_{p-1}(B)
    \end{tikzcd}
  \]
  dove $ f' = f \lvert_A $.
  Mentre la richiesta che sia funtore significa che se $ f \colon (X, A) \to (Y, B) $ e $ g \colon (Y, B) \to (Z, C) $ sono
  mappe continue allora $ (g \circ f)_\star = g_\star \circ f_\star $.
\end{definition}
L'omologia singolare relativa soddisfa tutti questi assiomi, a meno di quelli non ancora verificati che
sono l'omotopia e l'escissione.

\begin{definition}
  Sia $ \set{X_\alpha} $ una famiglia di spazi topologici, si definisce la \textbf{somma topologica}\index{Somma topologica}
  $ X = \invamalg_\alpha X_\alpha $
  come lo spazio topologico formato dall'unione disgiunta di tutti gli $ X_\alpha $ equipaggiato
  con la \textbf{topologia dell'unione disgiunta}\index{Topologia dell'unione disgiunta},
  ovvero un insieme è aperto se e solo se è aperto rispetto alla topologia di ogni $ X_\alpha $.
\end{definition}

% lezione 6

Ho costruito:
\[
  \begin{tikzcd}
    {} & 0 \arrow{d}{} & 0 \arrow{d}{} & {} \\
    \dots \arrow{r}{} & S_{k+1}(A) \arrow{r}{} \arrow{d}{} & S_k(A) \arrow{r}{} \arrow{d}{} & \dots \\
    \dots \arrow{r}{} & S_{k+1}(X) \arrow{r}{} \arrow{d}{} & S_k(X) \arrow{r}{} \arrow{d}{} & \dots \\
    \dots \arrow{r}{} & S_{k+1}(X,A) \arrow{r}{} \arrow{d}{} & S_k(X,A) \arrow{r}{} \arrow{d}{}& \dots \\
    {} & 0 & 0 & {}
  \end{tikzcd}
\]
So che ho $ \partial' $ perchè la successione è esatta, devo verificare che $ P \circ \partial = \partial' \circ P $.
Sia $ c \in S_{k+1}(X) $ allora $ p(c) = [c' \in S_{k+1}(X) \text{ tale che } c'-c \in S_{k+1}(A)] $,
e $ \partial'([c]_A) := [\partial c]_A $ così la relazione è soddisfatta.

Devo verificare che è ben definita:
Se $ c' \sim_A c $ allora $ c' - c = a \in S_{k+1}(A) $, prendo il bordo $ \partial c' - \partial c = \partial a \in S_{k+1}(A) $,
per definizione $ \partial c \sim_A \partial c $ quindi l'applicazione è ben definita.

L'omologia relativa è l'omologia singolare di $ S_\bullet(X, A) $, cioè:
\[
  H_k(X,A) = H_k(S_\bullet(X,A)) = \quot{\ker{S_k(X,A) \to S_{k-1}}}{\im{S_{k+1}(X,A) \to S_{k}}}
\]
Questo gruppo abeliano (in quanto è quoziente di gruppi abeliani) è detto
gruppo di omologia relativa della coppia  $ (X, A) $.
Ho fatto un'associazione da una coppia a un gruppo, voglio verificare che questa
soddisfa gli assiomi di Eilenberg e Steenrod.

Sia $ f \colon (X,A) \to (Y,B) $ tale che $ f(A) \incl B $, e sia:
\begin{align*}
  f_\star \colon H_k(X,A) & \to H_k(Y,B) \\
  \llbracket c \rrbracket_ A & \mapsto  \llbracket f_\sharp(c) \rrbracket_ B
\end{align*}
Quindi ho;
\[
  \begin{tikzcd}
    \dots \arrow{r}{} & S_{k+1}(X,A) \arrow{r}{} \arrow{d}{f_\sharp} & S_k(X,A) \arrow{r}{}  \arrow{d}{f_\sharp} & S_{k-1}(X,A) \arrow{r}{}  \arrow{d}{f_\sharp} & \dots \\
    \dots \arrow{r}{} & S_{k+1}(Y,B) \arrow{r}{} & S_k(Y,B) \arrow{r}{} & S_{k-1}(Y,B) \arrow{r}{} & \dots
  \end{tikzcd}
\]
$ f_\sharp $ esiste, infatti:
\[
  \begin{tikzcd}
    S_{k}(X) \arrow{r}{f} \arrow{d}{} & S_k(Y)  \arrow{d}{}  \\
    \quot{S_k(X)}{S_k(A)} \arrow{r}{f_\sharp} & \quot{S_k(Y)}{S_k(B)}
  \end{tikzcd}
\]
$ f_\sharp $ esiste perchè $ S_k(A) \to S_k(B) $ per la condizione $ f(A) \incl B $,
il resto l'ho già verificato.

Inoltre so che $ (X,A) \overset{f}{\to} (Y,B) \overset{g}{\to} (Z,C) $, allora
$ (g \circ f)_\star = g_\star \circ f_\star $ e $ (X,A) \overset{\Id{X}}{\to} $
allora $ (\Id{X})_\star = \Id{H_k(X,A)} $.

Poi ho $ \delta $ omomorfismo di connessione $ \delta \colon H_{k+1}(X,A) \to H_k(X) $,
lo che:
\[
  \begin{tikzcd}
    0 \arrow{r}{} & S_\bullet(A) \arrow{r}{} & S_\bullet(X) \arrow{r}{} & S_\bullet(X,A) \arrow{r}{} & \dots
  \end{tikzcd}
\]
Esiste una successione lunga in omologia.
\[
  \begin{tikzcd}
    \dots \arrow{r}{} & H_k(A) \arrow{r}{} & H_k(X) \arrow{r}{} & \arrow{r}{} & \dots
  \end{tikzcd}
\]


[MANCA TUTTA LE LEZIONE 6]

%  _     _____ ________ ___  _   _ _____   _____
% | |   | ____|__  /_ _/ _ \| \ | | ____| |___  |
% | |   |  _|   / / | | | | |  \| |  _|      / /
% | |___| |___ / /_ | | |_| | |\  | |___    / /
% |_____|_____/____|___\___/|_| \_|_____|  /_/

\section{Omologia delle sfere}

Considero $ \Sph{n} $ con $ n \geq 1 $, ho trovato che:
\[
  H_k(\Sph{n}) \cong
  \begin{cases}
    \Z & \text{se } k \in \set{0, n} \\
    0 & \text{se } k \not \in \set{0, n}
  \end{cases}
\]
Questo risultato ha numerose conseguenze, infatti ho trovato uno
strumento più fine del gruppo fondamentale che riesce a distinguere
cose diverse.

\begin{corollary}
  $ \Sph{n} \simeq \Sph{m} $ se e solo se $ n = m $.
\end{corollary}
\begin{proof}
Se $ n = m $ vale che $ \Sph{n} = \Sph{m} $ quindi in particolare
$ \Sph{n} \simeq \Sph{m} $ con la mappa identità. Assumo $ n \not = m $
e senza perdita di generalità pongo $ n > m $.

Per assurdo $ \Sph{n} \simeq \Sph{m} $, quindi esiste un omomorfismo
$ F: \Sph{n} \homoto \Sph{m} $, quindi esiste anche l'omomorfismo
inverso $ G: \Sph{m} \homoto \Sph{n} $.
Quindi esistono anche:
\[
  F_\star: H_k{\Sph{n}} \to H_k(\Sph{m}) \quad \text{e} \quad G_\star: H_k{\Sph{m}} \to H_k(\Sph{n})
\]
Ma $ F \circ G = \Id{\Sph{n}} $ e $ G \circ F = \Id{\Sph{m}} $, ma utilizzando
la funtorialità si trova quindi che:
\[
  F_\star \circ G_\star = \Id{H_k(\Sph{m})} \quad \text{e} \quad G_\star \circ F_\star = \Id{H_k(\Sph{n})}
\]
Da cui si deduce che $ F_\star $ e $ G_\star $ sono continue e sono inverse.
Vale quindi che:
\[
  H_k(\Sph{n}) \cong H_k(\Sph{m}) \; \forall k \geq 0
\]
Se vale per ogni $ k $ in particolare vale per $ k = n $, cioè:
\[
  H_n(\Sph{n}) = H_n(\Sph{m})
\]
Ma $ H_n(\Sph{n}) \cong \Z $ e $ H_n(\Sph{m}) \cong 0 $ da cui $ \Z \cong 0 $, che è assurdo.
\end{proof}

\begin{corollary}[Invarianza topologica della dimensione]
  $ \RN{n} \simeq \RN{m} $ se e solo se $ n = m $.
\end{corollary}
Come si è visto non si riesce a dimostrare questo corollario
utilizzano solo il gruppo fondamentale.
\begin{proof}
  Per assurdo esiste un omomorfismo $ f \colon \RN{n} \homoto \RN{m} $ con $ n > m > 2 $.
  Con i vincolo imposti su $ m $ e $ n $ gli spazi sono contraibili, quindi il gruppo
  fondamentale è in entrambi i casi banale. Togliendo un punto $ p \in \RN{n} $ e
  $ f(p) \in \RN{m} $, e restringendo $ f $ in modo da ottenere l'omomorfismo
  $ f' \colon \RN{n} \setminus \set{p} \homoto \RN{m} \setminus \set{f(p)} $.
  Si sa inoltre che per $ s \geq 2 $ vale che $ \RN{s} \setminus \set{q} \simeq \Sph{s-1} \times \RN{} $,
  infatti è sufficiente mandare a $ 0 $ il punto $ q $ con una traslazione
  (che è certamente un omomorfismo) e quindi si ha:
  \[
    \begin{aligned}[t]
      \RN{k} \setminus \set{q} & \to  \Sph{k-1} \times {\RN{}}^+ & \simeq \Sph{k-1} \times \RN{} \\
      \vec{x} & \mapsto \left( \vec{x}, \frac{\vec{x}}{|| \vec{x} ||} \right) &
    \end{aligned}
  \]
  Quindi:
  \[
    \RN{n} \setminus \set{p} \simeq \RN{m} \setminus \set{f(p)} \iff \Sph{n-1} \times \RN{} \simeq \Sph{m-1} \times \RN{}
  \]
  Si ha la tentazione di eliminare $ \RN{} $ dalla precedente relazione, ma
  questo non si può fare come mostrano alcuni casi molto patologici.
  Tuttavia è possibile passare alla omotopia sapendo che $ \Sph{k} \times \RN{} \sim \Sph{k} $,
  da cui $ \Sph{n-1} \sim \Sph{m-1} $. Ma l'omologia è invariante omotopico, cioè
  $ H_k(\Sph{n-1}) \cong H_k(\Sph{m-1}) $, utilizzando il trucco di prima scelgo $ k = n-1 $
  e quindi:
  \[
    H_{n-1}(\Sph{n-1}) \cong H_{n-1}(\Sph{m-1}) \iff \Z \cong 0
  \]
  Che è assurdo.
\end{proof}

\begin{corollary}
  $ \Sph{n-1} $ non è un retratto di deformazione di $ \Disk{n} $ per $ n \geq 2 $
\end{corollary}
\begin{proof}
  Si ricorda che:
  \[
    \Disk{n} = \set{ \vec{x} \in \RN{n} | || \vec{x} || \leq 1} \quad \Sph{n-1} = \partial \Disk{n} = \set{ \vec{x} \in \RN{n} | || \vec{x} || = 1}
  \]
  Chiaramente esiste $ i \colon \Sph{n-1} \incl \Disk{n} $.
  \begin{definition}
    Uno spazio topologico $ Y $ si dice \textbf{retratto di deformazione}\index{Retratto di deformazione} di un altro
    spazio topologico $ X $ tale che $ Y \incl X $ se esiste una funzione continua $ r\colon X \to Y $ che inverte a meno di omotopia
    la mappa di inclusione $ i\colon Y \to X $, cioè tale che soddisfa:
    \begin{enumerate}
    \item $ r\colon X \to Y $ continua
    \item $ i \circ r \sim \Id{X} $
    \item $ r \circ i = \Id{Y} $
    \end{enumerate}
    Una mappa che soddisfa queste condizioni è detta \textbf{retrazione}\index{Retrazione}.
  \end{definition}
  Suppongo per assurdo che $ \Sph{n-1} $ è un retratto di deformazione di $ \Disk{n} $, cioè che
  esiste una retrazione $ r $. Passando all'omologia:
  \begin{gather*}
    i_\star \colon H_k(\Sph{n-1}) \to H_k(\Disk{n}) \\
    r_\star \colon H_k(\Disk{n}) \to H_k(\Sph{n-1}) \\
    \left( i \circ r \right)_\star = (\Id{\Disk{n}})_\star \text{ e }  \left( r \circ i \right)_\star = (\Id{\Sph{n-1}})_\star
  \end{gather*}
  Quindi:
  \[
    i_\star \circ r_\star = \Id{H_k(\Disk{n})} \text{ e } r_\star \circ i_\star = \Id{H_k(\Sph{n-1})} \; \forall k \in \mathbb{N}
  \]
  In particolare considero $ k = n - 1 $:
  \begin{gather*}
    i_\star \colon H_n-1(\Sph{n-1}) \to H_n-1(\Disk{n}) \\
    r_\star \colon H_n-1(\Disk{n}) \to H_n-1(\Sph{n-1})
  \end{gather*}
  Cioè: $ i_\star \colon \Z \to 0 $. Considero un generatore $ \alpha $ di $ H_{n-1}(\Sph{n-1}) \cong \Z $, cioè tale
  che $ \langle\alpha\rangle = H_{n-1}(\Sph{n-1}) $ allora $ i_\star(\alpha) = 0 $ quindi $ r_\star \circ i_\star = 0 $, ma
  $ \left( r \circ i \right)_\star = \Id{\Sph{n-1}_\star} $ quindi significherebbe $ \Id{\Sph{n-1}_\star}(\alpha) = 0 $,
  cioè che $ \alpha = 0 $, che è assurdo perché $ \Z \not = \langle0\rangle $.
\end{proof}

\begin{theorem}[Teorema del punto fisso di Brouwer\index{Teorema del punto fisso}]
  Ogni funzione continua $ g \colon \Disk{n} \to \Disk{n} $ con $ n \geq 2 $ ammette almeno un punto fisso
  in $ \Disk{n} $, cioè:
  \[
    \exists \vec{x_o} \in \Disk{n} \; | \; g(\vec{x_0}) = \vec{x_0}
  \]
\end{theorem}

\begin{proof}
  Per assurdo $ g $ non ammette punto fisso cioè esisto $ \vec{x} \in \Disk{n} $
  tale che $ g(\vec{x}) \not = \vec{x} $. Sicuramente tuttavia $ g(\vec{x}) \in \Disk{n} $.
  Considero la retta $ l $ passante per $ \vec{x} $ e $ g(\vec{x}) $. Questa retta
  interseca il bordo di $ \Disk{n} $ in due punti $ \set{p_1, p_2} $:
  \[
    l \cap \partial \Disk{n} = l \cap \Sph{n-1} = \set{p_1, p_2}
  \]
  Definisco la mappa $ r \colon \Disk{n} \to \partial \Disk{n} = \Sph{n-1} $ tale che associ
  ad ogni punto del disco il punto di intersezione della retta $ l_{\vec{x}} $ che gli sta più
  vicino (infatti in $ \RN{n} $ è ben definita una nozione di distanza). La retta $ l_{\vec{x}} $
  è ben definita in quanto per due punti distinti (e per ipotesi  $ g(\vec{x}) \not = \vec{x} $)
  passa una e una sola retta.
  \begin{figure}[htbp]
    \centering
    \begin{tikzpicture}
      \draw (0,0) circle (2);
      \draw[-Latex] (-2.5, 0) -- (2.5,0);
      \draw[-Latex] (0, -2.5) -- (0, 2.5);
      \draw (-2, -2.33) -- (2, 3);
      \node[below, right] () at (0.5, 1) {$ g(\vec{x}) $};
      \node[] () at (0.5, 1) {\textbullet};
      \node[above] () at (-1, -1) {$ \vec{x} $};
      \node[] () at (-1, -1) {\textbullet};
      \node[left] () at (2, 3) {$ l $};
      \node[below, left] () at (-1.35, -1.47) {$ p_1 $};
      \node[] () at  (-1.35, -1.47) {\textbullet};
      \node[above, right] () at (1.04, 1.71) {$ p_2 $};
      \node[] () at  (1.04, 1.70) {\textbullet};
    \end{tikzpicture}
    \caption{Schema per $ n = 2 $}
    \label{fig:lez7:brouwer_proof_1}
  \end{figure}
  \begin{exercise}
    Dimostrare che $ r $ è continua.
  \end{exercise}
  Ho una mappa di inclusione naturale:
  \[
    \begin{aligned}[t]
      i \colon &\Sph{n-1} & \to &\Disk{n} \\
      &\vec{x} & \mapsto& \vec{x}
    \end{aligned}
  \]
  Se dimostro che $ r $ è una retrazione trovo un assurdo per il corollario
  precedentemente dimostrato.
  Devo verificare $ r \circ i = \Id{\Sph{n-1}} $ e $ i \circ r \sim \Id{\Disk{n}} $.
  La prima uguaglianza è certamente vera perché se $ \vec{x} \in \partial \Disk{n} $
  allora l'intersezione del bordo del disco che gli sta più vicina corrisponde a
  $ \vec{x} $ stesso.
  Costruisco esplicitamente una relazione di omotopia per mostrare la seconda:
  Siccome $ \Disk{n} $ è convesso è ben definita $ G(t, \vec{x}) = (1-t)\vec{x}
  + t r(\vec{x}) $ con $ t \in [0,1] $. Questa è una buona omotopia in quanto $ \forall t, \vec{x} $:
  \begin{itemize}
    \item $ G $ è continua
    \item $ G(t, \vec{x}) \in \Disk{n} $
    \item $ G(0, \vec{X}) = \vec{x} $
    \item $ G(1, \vec{X}) = r(\vec{x}) $
  \end{itemize}
  Quindi $ r $ è retrazione ma questo è assurdo.
\end{proof}

\subsection{Teoria del grado}

Considero $ H_n(\Sph{m}) $, so che $ H_n(\Sph{m}) \cong \Z $, cioè esiste una mappa
$ f \colon \Z \to H_n(\Sph{m}) $ tale che $ f(1) = \alpha $ con $ \alpha $ $ n $-ciclo
che non è un bordo. In questo modo $ H_n(\Sph{m}) = \langle\alpha\rangle $.
Considero $ \phi \colon \Sph{n} \to \Sph{n} $ continua con $ n \geq 1 $, so che esiste
$ \phi_\star \colon H_n(\Sph{n}) \to H_n(\Sph{n}) $. Per $ n = 0 $ $ \phi_\star $ manda punti
in punti, per $ n \geq 1 $: sia $ c \in H_n(\Sph{n}) $ allora $ c = p \alpha $ con $ p \in \Z $:
\[
  \phi_\star (c) = \phi_\star (p \alpha) = \phi_\star (\underbrace{\alpha + \alpha + \alpha + \dots}_{|p| volte}) =
  \underbrace{\phi_\star (\alpha) + \phi_\star (\alpha) + \dots}_{|p| volte} = p \phi_\star(\alpha)
\]
Ma $ \phi_\star(\alpha) \in H_n(\Sph{n}) $ quindi si deve poter scrivere come multiplo di $ \alpha $:
$ \phi_\star (\alpha) = d \alpha $ da cui: $ \phi_\star (c) = p d \alpha = d c $ con $ d \in \Z $.
Questo numero $ d $ viene fuori dall'immagine di un generatore, ma non dipende dalla
scelta del generatore, infatti:

Sia $ \beta $ un altro generatore, siccome $ \alpha $ è un generatore si può scrivere $ \beta = m \alpha $
con $ m \in \Z $. Pongo come notazione:
\[
  \phi_\star(\beta) = d(\beta) \beta \quad \phi_\star(\alpha) = d(\alpha) \alpha
\]
Allora:
\[
  d(\beta) \beta = \phi_\star (\beta) = m \phi_\star (\alpha) = m d(\alpha) \alpha
\]
Da cui $ d(\beta) \beta = \beta d(\alpha) $ cioè $ \left(d(\beta) - d(\alpha) \right) \beta = 0 $, siccome
questo vale per ogni $ \alpha $ e $ \beta $ allora $ d(\alpha) = d(\beta) $.
\begin{definition}
  Data un'applicazione $ \phi \colon \Sph{n} \to \Sph{n} $ continua è possibile
  associargli in modo univoco un numero intero, questo è il \textbf{grado}\index{Grado di una sfera}:
   \[
    \begin{aligned}[t]
      \phi_\star \colon & H_n(\Sph{n}) & \to & H_n(\Sph{n}) \\
      & \alpha & \mapsto & \deg{(\phi)} \alpha
    \end{aligned}
  \]
  Con $ \alpha $ generatore.
\end{definition}

Ad esempio per $ n = 1 $ e $ p \in \mathbb{N} $:
\[
  \begin{aligned}[t]
    \phi \colon & \Sph{1} & \to & \Sph{1} \\
    & z   & \mapsto & z^p
  \end{aligned}
\]
Vale che $ \deg{(\phi)} = p $, infatti prendo un generatore di $ \Sph{1} $:
[MANCA MANCA MANCA]

Voglio usare la teoria del grado per un'applcazione del teorema della palla pelosa.
\begin{proposition}
  Se $ f \colon \Sph{n} \to \Sph{n} $ è la riflessione rispetto all'iperpiano $ x_0 = 0 $,
  cioè $ f(x_0, x_1, x_2, \dots) = (-x_0, x_1, x_2, \dots) $ allora il grado di $ f $ è $ - 1 $.
\end{proposition}
\begin{proof}
  La dimostrazione è per induzione.
  Per $ n = 1 $
  [MANCA MANCA MANCA MANCA MANCA MANCA]
\end{proof}

% lezione 8

Ho $ \Sph{n} = \set{ (x_1, \dots, x_{n+1}) \in \RN{n+1} | \sum_{i=1}^{n+1} x_i^2 = 1 } \subset \RN{n+1} $
spazio topologico con la topologia indotta.

Ho trovato che:
\[
  H_k(\Sph{n}) \cong
  \begin{cases}
    \Z & \text{se } k \in \set{0, n} \\
    0 & \text{se } k \not \in \set{0, n}
  \end{cases}
\]
Ho che $ f \colon \Sph{n} \to \Sph{n} $ induce $ f_\star \colon H_n(\Sph{n}) \to H_n(\Sph{n}) $ e ho definito il grado
come:

Prendo $ \alpha $ tale che $ \langle\alpha\rangle = H_n(\Sph{n}) $ con:
\begin{align*}
  H_n(\Sph{n}) & \to \Z \\
  \alpha & \mapsto 1
\end{align*}
e $ f_\star (\alpha) = \deg{(f)} \alpha $. So che il grado è un invariante topologico per le sfere.

\begin{proposition}
  Siano $ f,g \colon \Sph{n} \to \Sph{n} $ mappe continue, allora $ \deg{(g \circ f)} = \deg{(f)} \deg{(g)} $.
\end{proposition}
\begin{proof}
  Per la funtorialità $ (g \circ f)_\star = g_\star \circ f_\star $ quindi:
  \[
    (g \circ f)_\star (\alpha) = (g_\star \circ f_\star)(\alpha) \quad \Rightarrow \quad g_\star (f_\star (\alpha)) = g_\star(\deg{(f)}\alpha) = \deg{(f)} g_\star(\alpha) = \deg{(f)}\deg{(g)}\alpha
  \]
  Quindi:
  \[
    \deg{(f)}\deg{(g)}\alpha = (g \circ f)_\star (\alpha) = \deg{(g \circ f)} \alpha
  \]
  Siccome $ \alpha $ è generatore: $ \deg{(g \circ f)} = \deg{(f)} \deg{(g)} $.
\end{proof}

Voglio applicare questa proprietà alla composizione di riflessioni.
Considero:
\begin{align*}
  \rho \colon \Sph{n} & \to \Sph{n} \\
  (x_1, \dots, x_{n+1}) & \mapsto (x_1, \dots, - x_{n-1})
\end{align*}
Questa è la riflessione rispetto al sottospazio $ x_{n+1} = 0 $ in $ \RN{n+1} $, quindi
$ \rho $ fissa su $ \Sph{n} $ l'equatore $ \Sph{n} \cap x_{n+1} = 0 $.
Ho trovato che per $ n = 1 $ $ \deg{(\rho)} = -1 $, ora voglio continuare la
dimostrazione per induzione.

Suppongo che il risultato sia vero per $ \Sph{k} $ e $ \Sph{n} $ con $ k < n $
mostro che è vero anche per $ \Sph{n} $.

In $ \Sph{n} $ ho dei sottoinsiemi naturali:
\begin{gather*}
  \Disk{n}_+ = \set{ (x_1, \dots, x_{n+1}) \in \Sph{n} | x_1 \geq 0 } \\
  \Disk{n}_- = \set{ (x_1, \dots, x_{n+1}) \in \Sph{n} | x_1 \leq 0 }
\end{gather*}
Vale che $ \Disk{n}_+ \cap \Disk{n}_- = \set{ (x_1, \dots, x_{n+1}) \in \Sph{n} | x_1 \leq 0 } = \Sph{n-1} $.
Ho dimostrato che $ \tilde{H}_p(\Sph{n}) \cong H_p(\Disk{n}, \Sph{n-1}) \cong H_p(\Sph{n}, \Disk{n}) $
Lo uso
[FIGURA]

Ora dimsotro che $ H_n{\Disk{n}, \Sph{n-1}} \cong \tilde{H}_{n-1}(\Sph{n-1}) $.
Sia $ i \colon \Sph{n-1} = \partial \Disk{n} \to \Disk{n} $ mappa di inclusione. Ho $ (\Disk{n}, \Sph{n-1}) $ coppia,
faccio la successione esatta lunga in omologia relativa:
[FIGURA]
Considero per $ p = n $ con $ n \geq 2 $ $ H_p(\Disk{n}) = H_p(\Disk{n-1}) = 0 $ perchè contraibili.
Quindi $ H_n(\Disk{n}, \Sph{n-1}) \cong H_{n-1}(\Sph{n-1}) $
[MANCA MANCA]


Quindi il risultato finale è:
\begin{lemma}
  Se $ \rho \colon \Sph{n} \to \Sph{n} $ è la riflessione rispetto a $ \set{x_{n+1} = 0} $ allora
  $ \deg{(\rho)} = - 1 $.
\end{lemma}

Perchè voglio fare la riflessione?
Voglio studiare l'applicazione antipodale che è quella che scambia di segno tutte le componenti:
\begin{align*}
  A \colon \RN{n} & \to \RN{n} \\
  (x_1, \dots, x_{n}) & \mapsto (-x_, \dots, -x_n)
\end{align*}
Questa è continua e vale che $ A^2 = \Id{\RN{n}}$. Definisco per $ n \geq 2 $ la restrizione della trasformazione
antipodale su $ \Sph{n-1} $: $ a = A \lvert_{\Sph{n-1}} $, vale che $ a \colon \Sph{n-1} \to \Sph{n-1} $, infatti
$ \im{a} = \Sph{n-1} $. Quanto vale $ \deg{(a)} $?
Scrivo $ a $ come composizione di riflessioni:
\[
  a = \rho_n \circ \dots \circ \rho_1
\]
Per il lemma appena dimostrato:
\[
  \deg{(a)} = \deg{(\rho_n \circ \dots \circ \rho_1)} = \deg{(\rho_n)}\deg{(\rho_{n-1})}\dots\deg{(\rho_1)} = (-)^n
\]
Quindi $ \deg{(a)} = (-)^n $ e perciò cambia se $ n $ è pari o dispari.

\begin{corollary}
  La mappa antipodale non è omotopicamente equivalente all'identità su $ \Sph{n} $ su $ n $ è pari.
\end{corollary}
\begin{proof}
  Se le due applicazioni fossero omotope varrebbe che $ a_\star = (\Id{\Sph{n}})_\star $ quindi:
  \[
    \deg{(a)} = \deg{(\Id{\Sph{n}})} = (-)^{n+1} = 1
  \]
  Questo è vero solo se $ n + 1 $ è pari, ma se $ n $ è pari $ n + 1 $ non può esserlo.
\end{proof}

Ciò non dimostra che per $ n $ pari invece le due applicazioni sono omotope. Questa è una
dimostrazione avanzata che richiede i gruppi di omotopia superiori con i quali si dimostra
che se due applicazioni definite su $ \Sph{n} $ hanno lo stesso grado allora sono omotope.

\begin{corollary}
  Sia $ f \colon \Sph{n} \to \Sph{n} $ una mappa continua con $ n $ pari, allora esiste almeno
  un punto $ \vec{x_0} \in \Sph{n} $ tale che $ f(\vec{x_0}) = \pm x_0 $.
\end{corollary}
\begin{proof}
  Per assurdo $ f(\vec{x}) \not = \pm \vec{x} \; \forall \vec{x} \in \Sph{n} $. Sia $ F \colon \Sph{n} \times I \to \Sph{n} $
  con:
  \[
    F(\vec{x}, t) = \frac{t f(\vec{x}) + (1-t)\vec{x}}{|| t f(\vec{x}) + (1-t)\vec{x} ||}
  \]
  $ \forall \vec{x}, t $ vale che $ F(\vec{x}, t) \in \Sph{n} $.
  La norma al denominatore non è mai nulla per ipotesi, infatti $ || t f(\vec{x}) + (1-t) \vec{x} || = 0 $
  significa che $ t f(\vec{x}) = (1-t)\vec{x} $, quindi se $ t = 0 $ allora $ 0 = - \vec{x} $ ma $ \vec{x} = 0 \not \in \Sph{n} $,
  se $ t \not = 0 $ allora $ f(\vec{x}) = \left(\frac{t-1}{t}\right)\vec{x} $, ma $ \vec{x}, f(\vec{x}) \in \Sph{n} $
  quindi $ || f(\vec{x}) || = || \vec{x} || = 1 $ e quindi $ 1 = \big \rvert \frac{t-1}{t}\big \lvert $,
  ma $ t \in (0,1] $, quindi non è possibile trovare $ t $.

  Inoltre $ F(\vec{x}, 0) = \vec{x} $ e $ F(\vec{x}, 1) = f(\vec{x}) $ quindi $ F $ è una relazione di omotopia
  tra $ f $ e l'identità.

  Mostro che $ f $ è anche omotopa all'applicazione antipodale, così per la transitività della
  relazione di omotopia trovo l'assurdo.

  Si definisce  $ G\colon \Sph{n} \times I \to \Sph{n} $:
  \[
    G(\vec{x}, t) = \frac{-t \vec{x} + (1-t)f(\vec{x})}{|| -t \vec{x} + (1-t)f(\vec{x}) ||}
  \]
  Con i medesimi ragionamenti si trova che   $ \forall \vec{x}, t $ vale che $ G(\vec{x}, t) \in \Sph{n} $, e inoltre
  $ G(\vec{x}, 0) = f(\vec{x}) $ e $ G(\vec{x}, 1) = - \vec{x} $ quindi $ G $ realizza
  l'omotopia con l'applicazione antipodale.
\end{proof}

% lezione 8 parte 2

\chapter{Omologia cellulare}

\section{Complessi CW}

Considero $ \Disk{n} $, vale che $ \partial\Disk{n} = \Sph{n-1} $, considerato lo spazio quoziente
$ X = \quot{\Disk{n}}{\partial \Disk{n}} $, questo è il quoziente del disco per la relazione
di equivalenza che fa collassare il bordo in un punto $ p $. Si trova che $ X \simeq \Sph{n} $.

Posso fare questa costruzione:
Definisco $ X^{(0)} = P $, lo spazio formato da un solo punto
e $ \phi \colon \Sph{n-1} \to X^{(0)} $, posso definire:
\[
  X^{(1)} = X^{(0)} \cup_\phi \Disk{n}
\]
Dove con $ \cup_\phi $ si intende, con $ X, Y $ spazi topologici:
\newmathsymb{disgun}{sqcup}{Unione disgiunta}
\[
  X^{(0)} \cup_\phi \Disk{n} = \quot{X^{(0)} \sqcup \Disk{n}}{p \sim \phi(q)} \quad \forall q \in \Sph{n-1}
\]
Questo si generalizza. Considero la proiezione al quoziente $ \pi \colon \Disk{n} \to \Sph{n} $
con $ \Sph{n} = X^{(0)} \cup_\phi \Disk{n} $.

\begin{definition}
  Si definisce $ X $ un \textbf{CW-complesso} di tipo finito\index{CW-complesso}, dove C singifica
  \emph{closure finite} e W \emph{weak topology} è il dato dei seguenti oggetti topologici:
  \begin{enumerate}
  \item Un insieme finito $ X^{(0)} = \set{p_1, \dots, p_n} $ detto \textbf{$ 0 $-scheletro}\index{$ 0 $-scheletro}
  \item Il \textbf{$ k $-scheletro}\index{$ k $-scheletro} $ X^{(k)} = X^{(k-1)} \cup_\phi \bigcup_\alpha \Disk{k}_\alpha $
    con $ \Disk{k}_{\alpha_1}, \dots,  \Disk{k}_{\alpha_j} $ numero finito di dischi $ k $-dimensionali.
    Gli elementi $ \Disk{k}_\alpha $ sono detti \textbf{celle}\index{Cella} (o cella chiusa), mentre
    il loro interno è detto cella aperta.
    $ \psi \colon \bigcup_\alpha \partial \Disk{k}_\alpha \to X^{(k-1)} $ continua fornisce le regole di attaccamento per i dischi.
  \item Esiste $ N \in \mathbb{N} $ tale che $ X^{(0)} \subseteq X^{(1)} \subseteq \dots \subseteq X^{(n)} =: X $
  \end{enumerate}
\end{definition}

Si dimostra che in generale la cella chiusa non è omeomorfa all'immagine, mentre la cella
aperta lo è.

In generale uno spazio ha numerose strutture di CW complesso.

La topologia è detta debole perchè la topologia di unione disgiunta per tutti i $ k $-scheletri,
e questo è la topologia più debole di tutte. In questa topologia un insieme è aperto in $ X $
se e solo se è aperto la sua intersezione con tutti gli $ X^{(i)} $ è aperta.

\begin{example}[Sfere]
  $ \Sph{n} $ per $ n \geq 1 $ ammette una struttura di CW complesso: sia $ X^{(0)} = \set{p} $ con
  $ p \in \Sph{n} $ e sia $ \phi \colon \partial \Disk{n} \to \set{p} $ mappa costante, allora
  $ \Sph{n} = X^{(0)} \cup_\phi \Disk{n} = \quot{\Disk{n}}{\partial \Disk{n}} $, cioè una $ 0 $-cella e una
  $ n $-cella.

  Alternativamente una seconda possibile struttura è: considero l'equatore di $ \Sph{n} $ che
  è uno $ \Sph{n-1} $ su questo attacco due dischi che sono calotta superiore e inferiore.
  $ X^{(0)} = \set{p_1, p_2} $ e $ X^{(1)} = \Disk{1}_1 \cup \Disk{1}_2 $, quindi le mappe
  sono:
  \begin{gather*}
    \phi_1 \colon \partial \Disk{1}_1 \to X^{(0)} \quad \text{cioè} \quad \phi_1 \colon \set{-1, +1} \to \set{p_1, p_2} \\
    \phi_1 \colon \partial \Disk{1}_2 \to X^{(0)} \quad \text{cioè} \quad \phi_2 \colon \set{-1, +1} \to \set{p_1, p_2}
  \end{gather*}
  Quindi deve essere:
  \[
    \phi_1(1) = p_1 \quad \phi_1(-1) = p_2 \qquad  \phi_2(1) = p_2 \quad \phi_2(-1) = p_1
  \]
  A questo punto $ X^{(1)} \cup_\phi (\Disk{1}_1 \cup \Disk{1}_2) = \Sph{1} $ e si può aggiungere
  $ \Disk{2} $, cioè $ X^{(2)} = \Disk{2}_1 \cup \Disk{2}_2 $, ora:
  \begin{gather*}
    \psi_1 \colon \partial \Disk{2}_1 \to X^{(2)} \\
    \psi_1 \colon \partial \Disk{2}_2 \to X^{(0)}
  \end{gather*}
  Cioè $ \psi_j \colon \Sph{1} \to X^{(1)} $, quindi $ \psi_j \colon \Sph{1} \to \Sph{1} $ e quindi si può prendere
  l'identità. Si ottiene così una $ 2 $-sfera. A questo punto si può procedere ad libidum.
\end{example}

\begin{example}[Toro]
  Considerato un toro $ T = \Sph{1} \times \Sph{1} $ una possibile costruzione si basa sul prendere
  come punti $ p $ i vertici del quadrato dal quale si fanno le identificazioni per ottenere il
  toro. $ X^{(0)} = \set{p} $, ho quindi due lacci. Quindi:
  \[
    X^{(1)} = (\Disk{1}_1 \cup \Disk{1}_2) \cup_\phi X^{(0)}
  \]
  Le mappe:
  \begin{gather*}
    \phi_1 \colon \set{-1, +1} \to \set{p} \\
    \phi_2 \colon \set{-1, +1} \to \set{p}
  \end{gather*}
  La cella è $ X^{(2)} = (\Disk{2} \cup_\psi X^{(1)}) $ con $ \psi \colon \Sph{1} \to X^{(1)} $
  [MANCA IL SECONDO MODO]
\end{example}

\begin{definition}
  Si definice lo \textbf{spazio proiettivo reale}\index{Spazio proiettivo reale} $ \mathbb{P}^n(\RN{}) = \quot{\RN{n+1} \setminus \set{0}}{\sim} $
  con $ \vec{x} \sim \vec{y} $ se e solo se $ \vec{x} $ e $ \vec{y} $ sono multipli,
  cioè se esiste $ \lambda \in \RN{+} $ tale che $ \vec{x} = \lambda \vec{y} $.
\end{definition}

Si dimostra che $ \mathbb{P}^n(\RN{}) \cong \quot{\Sph{n}}{H} $ con $ H = \set{\Id{\Sph{n}}, a_{\Sph{n}}} $.

Si trova che:
\begin{itemize}
\item $ \mathbb{P}^1(\RN{}) = \Sph{1} = \RN{} \cup {\infty} $
\item $ \mathbb{P}^2(\RN{}) = \mathbb{P}^1(\RN{}) \cup_\phi \Disk{2} $
  Ho $ \quot{\Sph{2}}{H} $, l'emisfero sud della sfera si identifica con quello
  nord per l'applicazione di antipodalità.
  $ \phi = a \big \lvert_{\Sph{1}} $ e $ \Sph{1} = \mathbb{P}^1(\RN{}) $ e $ \Sph{1} = \partial \Disk{2} $,
  quindi:
  \begin{align*}
    \phi \colon \Sph{1} & \to \Sph{1} \\
    (x,y) & \mapsto (-x,-y)
  \end{align*}
\item Se considero $ \mathbb{P}^2(\RN{}) \cup_\phi \Disk{3} $ con:
  \[
    \phi \colon \partial \Disk{3}  \to \mathbb{P}^2(\RN{})
  \]
  cioè il passaggio al quoziente:
  \[
    \phi \colon \Sph{2}  \to \quot{\Sph{1}}{H}
  \]
\end{itemize}

\begin{example}[Spazi proiettivi]
  Se $ X^{(k)} = \mathbb{P}^k(\RN{}) \cup_\phi \Disk{k+1} $ con
  \[
    \phi \colon \partial \Disk{k+1}  \to \mathbb{P}^k(\RN{})
  \]
  Cioè:
  \[
    \phi \colon \Sph{k}  \to \mathbb{P}^k(\RN{})
  \]
  Cioè scelgo $ \phi $ come la proiezione sul quoziente da $ \Sph{k} $ a $ \mathbb{P}^k(\RN{}) = \quot{\Sph{k+1}}{H} $,
  questo è uno spazio compatto.
  $ \mathbb{P}^k(\RN{}) $ è uno spazio di Hausdorf, voglio mostrare che $ X^{(k+1)} \cong \mathbb{P}^{k+1}(\RN{}) $.
  Cerco un'applicazione continua biunivoca e chiusa $ \Phi \colon X^{(k+1)} \to  \mathbb{P}^{k+1}(\RN{}) $,
  cioè un omeomorfismo. Ho il digramma:
  \[
    \begin{tikzcd}
      \mathbb{P}^{k}(\RN{}) \sqcup \Disk{k+1} \arrow{r}{\eta} \arrow{d}{} &  \mathbb{P}^{k+1}(\RN{}) \\
      X^{(k+1)} \arrow{ru}{\Phi}
    \end{tikzcd}
  \]
  \begin{exercise}
    Dimostrare che $ \eta $ è continua e gode di tutte le buone proprietà.
  \end{exercise}
  So che $ i \colon  \mathbb{P}^{k}(\RN{}) \incl \mathbb{P}^{k+1}(\RN{}) $ (è un iperpiano all'infinito),
  quindi posso usare l'inclusione.

  Devo trovare una mappa $ j \colon \Disk{k+1} \to \mathbb{P}^{k+1}(\RN{}) $. $ i $ è ovvia:
  $ i([z_0, \dots, z_k]) = [z_0, \dots, z_k; 0] $, mentre $ j $:
  \[
    j \colon [z_0, \dots, z_k] \mapsto  \left[z_0, \dots, z_{k+1} = \sqrt{1 - \sum_{j=1}^k z_i^2}\right]
  \]
  Siccome $ \sum_{j=1}^k z_i^2 \leq 1 $ l'applicazione è ben definita, quindi $ \eta = (i,j) $.
\end{example}




% lezione 8

\section{Congettura di Poincaré}

Ho calcolato l'omologia di una sfera generica:
\[
  H_k(\Sph{n}) \cong
  \begin{cases}
    \Z & \text{ se } k \in \set{0,n} \\
    0 & \text{ se } k \not \in \set{0,n}
  \end{cases}
\]
In particolare ho $ H_0(\Sph{n}) \cong \Z $ ed è generato dalla classe di
omologia di un punto qualsiasi, mentre $ H_n(\Sph{n}) \cong \Z $ è generatodalla
classe di omologia di un $ n $-simplesso singolare $ \tau_n \colon \Delta_n \to \Sph{n} $.

Per $ n = 2 $ ho $ \Sph{2} $ è una $ 2 $-varietà topologica compatta e connessa
il cui gruppo fondamentale è banale e i gruppi di omologia noti.

\begin{proposition}
  Se $ \M $ è una $ 2 $-varietà topologica compatta e connessa tale che $ \forall k \geq 2 $
  $ H_k(\M) \cong H_k(\Sph{2}) $ allora $ \M \simeq \Sph{2} $.
\end{proposition}
\begin{proof}
  Esiste un teorema di classificazione delle varietà topologiche di dimensione $ 2 $
  compatte e connesse, questo dice che $ \M \simeq V_g $ oppure $ \M \simeq U_n $.
  Dove:
  \[
    V_g =
    \begin{cases}
      \Sph{2} & \text{se } g = 0 \\
      \quot{P_{4g}}{\sim} & \text{se } g \geq 1
    \end{cases}
  \]
  Dove $ \sim $ è l'identificazione $ a_1 b_1 a_1^{-1}b_1^{-1}\dots a_g b_g a_g^{-1}b_g^{-1} $,
  come ad esempio il toro, mentre:
  \[
    U_n =
    \begin{cases}
      \mathbb{P}^2(\RN) & \text{se } n = 0 \\
      \quot{P_{2g}}{\sim} & \text{se } n \geq 1
    \end{cases}
  \]
  Con $ \sim $ è l'identificazione $ a_1 a_1 \dots a_n a_n $, come ad esempio
  la bottiglia di Klein.
  Tutti i $ V_g $ non sono omeomorfi tra loro, e similmente gli $ U_n $, e neppure
  gli $ U_n $ e i $ V_g $ sono vicendevolmente omeomorfi in quanto i primi sono
  non orientabili, mentre i secondi si.

  Ho calcolato:
  \[
    H_1(V_g) \cong
    \begin{cases}
      H_1(\Sph{2}) & \text{se } g = 0 \\
      \Z^{2g} & \text{se } g \geq 1
    \end{cases}
  \]
  $ V_g $ con $ g \geq 1 $ non hanno lo stesso tipo di omologia di $ \Sph{2} $ perchè
  $ H_1(V_g) $ è non banale, mentre il gruppo fondamentale di $ \Sph{2} $ lo è.
  Similmente $ H_1(\mathbb{P}^2(\RN)) \cong \pi_1(\mathbb{P}^2(\RN)) \cong \Z_2 $, che non è
  banale, e $ H_1(U_n) \cong \Ab{\pi_1(U_n)} $, ma per Seifert-van Kampen:
  \[
    \pi_1(U_n) = \langle a_t, \dots, a_n \; | \; a_1^2\dots a_n^2 = 1 \rangle \Rightarrow \Ab{\pi_1(U_n)} = \langle a_t, \dots, a_n, c = a_1 \dots a_n \; | \; a_1^2\dots a_n^2 = 1 \rangle =
    \Z_2 \oplus \Z^{n-1}
  \]
  Dove $ \Z_2 $ viene dal fatto che abelianizzando $ a_1^2\dots a_n^2 = (a_1 \dots a_n)^2 = 1 $ quindi
  $ c = \pm 1 $, mentre $ \Z^{n-1} $ è il gruppo libero generato dai rimanenti.
  Questo non è banale, quindi l'unico spazio possibile è proprio $ \Sph{2} $.
\end{proof}

Cosa si può invece dire su $ \Sph{3} $? Vale la seguente proposizione:
\begin{proposition}
  Se $ \M $ è una $ 3 $-varietà topologica compatta e connessa tale che $ \forall k \geq 3 $
  $ H_k(\M) \cong H_k(\Sph{3}) $ allora non si può concludere che $ \M \simeq \Sph{3} $.
\end{proposition}
\begin{proof}
  Costruisco un controesempio, noto come \textbf{spazio dodecaedrico di Poincaré}\index{Spazio dodecaedrico},
  o anche spazio a omologia razionale\index{Spazio a omologia razionale ! \vedi{Spazio dodecaedrico}}.
  Costruisco $ P $ $ 3 $-varietà topologica compatta e connessa con lo stesso tipo di omologia di
  una $ 3 $-sfera ma non omeomorfa a $ \Sph{3} $ in quanto il gruppo fondamentale è finito
  non abeliano di ordine 120.
  Parto da $ \Sph{3} $, posso scrivere:
  \[
    \Sph{3} \subseteq \mathbb{C}^2 \qquad \Sph{3} = \set{ (z_0, z_1) \in \mathbb{C}^2 | |z_0|^2 + |z_1|^2 = 1}
  \]
  Infatti $ z_0 = x + i y $ e $ z_1 = t + i w $ quindi $ |z_0|^2 = (x + iy)(x - iy) = x^2 + y^2 $
  e $ |z_0|^2 = (t + iw)(t - iw) = t^2 + w^2 $ e quindi ottengo:
  \[
    \Sph{3} = \set{ (x,y,t,w) \in \RN{4} | x^2 + y^2 + t^2 + w^2 = 1}
  \]
  Così come $ \Sph{1} $ ha una struttura di gruppo U(1) è possibile strutturare
  $ \Sph{3} $ come gruppo SU(2):
  \[
    \mathrm{SU(2)} = \set{A \in M_2(\mathbb{C}) | \det{A} = 1, \; AA^\dagger = \Id{2}}
  \]
  Quindi $ \mathrm{SU(2)} \subseteq \mathbb{C}^4 $, si dimostra che $ A \in \mathrm{SU(2)} $ se e solo se
  è della forma:
  \[
    \begin{pmatrix}
      \alpha & - \beta^\star \\
      \beta & \alpha^\star \\
    \end{pmatrix}
    \text{ con } \alpha,\beta \in \mathbb{C} \text{ e } |\alpha|^2 + |\beta|^2 = 1
  \]
  Quello che sto dicendo è che i vettori in $ \mathbb{C}^2 $ $ (\alpha, \beta) $ e $ (-\beta^\star, \alpha^\star) $ sono
  normalizzati e sono tra di loro ortogonali.

  Si costruisce immediatamente la corrispondenza buinivoca tra SU(2) e $ \Sph{3} $:
  \begin{align*}
    \mathrm{SU(2)} & \leftrightarrow \Sph{3} \\
    \begin{pmatrix}
      \alpha & - \beta^\star \\
      \beta & \alpha^\star \\
    \end{pmatrix} & \leftrightarrow (\alpha,\beta)
  \end{align*}
  In questo modo si può definire un prodotto su $ \Sph{3} $ rappresentanto
  $ x,y,t,w $ come numeri complessi e passando alla controparte matriciale, dove
  il prodotto è definito naturalmente come prodotto riga per colonna, quindi
  una volta svolto il prodotto si torna alla notazione a quattro reali. A questo
  punto è triviale trovare l'identità e l'elemento inverso che permettono di dare
  a $ \Sph{3} $ la struttura di gruppo.

  SU(2) può essere visto come spazio topologico con topologia indotta da $ \mathbb{C}^4 $,
  in questo senso SU(2) e $ \Sph{3} $ sono sia isomorfi come gruppi che omeomorfi
  come spazi topologici.

  La costruzione dello spazio dodecaedrico si basa sulle isometrie del dodecaedro $ D_{12} $,
  questo è un solido regolare con 12 facce, 30 spigoli e 20 vertici.
  Il gruppo di isometrie del dodecaedro, cioè:
  \[
    \mathrm{Isom}(D_{12}) = \set{ g \colon \RN{3} \to \RN{3} | g \text{ regolare e } g(D_{12}) = D_{12}}
  \]
  Questo gruppo si può vedere come:
  \[
    \mathrm{Isom}(D_{12}) \cong A_5 \times \Z_2
  \]
  Dove $ A_5 $ è un sottogruppo di $ \mathrm{Isom}(D_{12}) $ ed è il gruppo
  alterno (cioè il gruppo delle permutazioni pari) su 5 elementi e quindi ha ordine 60.
  Le 60 trasformazioni che sono in $ A_5 $ sono l'identità, 24 rotazioni di $ \frac{2}{5} \pi $ attorno
  agli assi per i centri di facce opposti, 20 rotazioni di $ \frac{2}{3} \pi $ attorno
  agli assi per vertici opposti e 15 rotazioni di $ \pi $ attorno agli assi per
  i punti medi di spigoli opposti.
  $ \Z_2 $ invece è dovuto all'applicazione antipodale che è $ (x,y,z) \mapsto (z,y,z) $.
  $ A_5 $ è un sottogruppo finito di SO(3) che sono le rotazioni di $ \RN{3} $ attorno
  a una retta passante per l'origine, cioè:
  \[
    \mathrm{SO(3)} = \set{ R \in M_3(\RN{}) | \det{R} = 1, \;R^T R = \Id{3}}
  \]
  Per passare da SO(3) a $ \Sph{3} $ utilizzo la \textbf{rappresentazione spinoriale}\index{Rappresentazione spinoriale di SO(3)}
  (questo mi permette di passare dal dodecaedro che è tridimensionale alla $ 3 $-sfera).
  Sia $ \rho $ una rappresentazione di SU(2), cioè un omomorfismo:
  \[
    \rho \colon \Sph{3} = \mathrm{SU(2)} \to \mathrm{GL}(V)
  \]
  Dove $ V $ è uno spazio vettoriale di dimensione 3, quindi $ V \cong \RN{3} $, scelgo
  lo spazio delle matrici antihermitiane a traccia nulla:
  \[
    V = \set{H \in M_2(\mathbb{C}) | H + H^\dagger = 0, \; \tr H  = 0}
  \]
  Si trova che $ V $ è generato da:
  \[
    E_1 =
    \begin{pmatrix}
      0 & i \\
      i & 0 \\
    \end{pmatrix}
    \quad
    E_2 =
    \begin{pmatrix}
      0 & 1 \\
      -1 & 0 \\
    \end{pmatrix}
    \quad
    E_3 =
    \begin{pmatrix}
      i & 0 \\
      0 & -i \\
    \end{pmatrix}
  \]
  Perchè $ \rho $ sia una rappresentazione dovrei verificare:
  \begin{enumerate}
  \item $ \rho(T) $ lineare
  \item $ \rho(T)(H) \in V $
  \item $ \rho $ omomorfismo
  \item $ \rho(T) $ inveribile
  \end{enumerate}
  Verifico ad esempoi che $ \rho(T)(H) \in V $:
  \begin{gather*}
    THT^\dagger + TH^\dagger T^\dagger = 0 \iff T(H + H^\dagger)T^\dagger = 0 \iff H \in V \\
    \tr(THT^\dagger) = \tr(THT^{-1}) \overset{\text{ciclicità}}{=} \tr(H) = 0 \iff H \in V
  \end{gather*}
  Ho quindi $ \rho \colon \Sph{3} \to \mathrm{GL}(V) $, vorrei cercare di restringere
  la questione a O($ V $) al posto di $ \mathrm{GL}(V) $.

  Per parlare di isometria bisogna prima definire un prodotto scalare definito
  positivo, e una possibile forma quadratrica naturale è in questo caso il determinante, infatti
  se $ H \in V $ allora:
  \[
    H =
    \begin{pmatrix}
      i a & c + i b \\
      -c + i b & - i a
    \end{pmatrix}
  \]
  Con $ a,b,c \in \RN{} $, infatti $ \det{H} = a^2 + b^2 + c^2 $ che è il consueto
  prodotto scalare in $ \RN{3} $. In questo modo $ V $ diventa uno spazio euclideo
  con prodotto scalare $ q = \det $.

  Mi chiedo $ \rho(T) \colon V \to V $ è isometria? Questo è vero se $ q(\rho(T)(H)) = q(H) $
  cioè se $ \det(THT^\dagger) = \det{H} $, ma per Binet questo equivale a
  $ \det{T}\det{H}\det{T^\dagger} = \det{H} $, utilizzando il fatto che il determinante
  di una matrice è un numero complesso e quindi commuta questo equivale a
  $ \det{T}\det{T^\dagger}\det{H} = \det{H} $, sempre per Binet  $ \det(TT^\dagger)\det{H} = \det{H} $,
  ma per ipotesi $ TT^\dagger = \Id{} $ quindi effettivamente $ \rho(T) $ è isometria, perciò:
  \[
    \rho \colon \Sph{3} \to \mathrm{O}_3(V)
  \]
  \begin{exercise}
    Verificare che $ \rho $ è continuo come applicazione tra spazi topologici
    equipaggiando $ \mathrm{O}_3(V) $ con la topologia indotta da $ \RN{9} $.
  \end{exercise}
  Essendo $ \rho $ continua manda compatti in compatti e connessi in connessi, quindi
  $ \rho(\Sph{3}= \mathrm{SU(2)}) $ è connesso in $ \mathrm{O}_3(V) $.
  Ma $ \mathrm{O}_3(V) $ non è connesso, e anzi è formato da due componenti connesse,
  una è SO(3), l'altra è SO(3) moltiplicata per una qualunque matrice di determinante
  $ - 1 $. Siccome $ \rho $ è omomorfismo $ \rho(\Id{}) = \Id{} $, quindi $ \rho(\Sph{3}) =  \mathrm{SO(3)}) $,
  in questo modo rappresento la $ 3 $-sfera come rotazioni in $ \RN{3} $.
  Si dimostra che $ \rho $ è suriettiva e $ \ker{\rho} = \set{(1,0,0,0), (-1,0,0,0)} $ elementi
  che corrispondono a $ \Id{} $ e $ -\Id{} $.

  Quindi come gruppi:
  \[
    \quot{\Sph{3}}{\ker{\rho}} \cong \mathrm{SO(3)}
  \]
  Ad una rotazione in $ \RN{3} $ corrispondono due punti sulla $ 3 $-sfera che sono
  uno l'antipodale dell'altro.

  Ora ho $ A_5 \subseteq \mathrm{SO(3)} $ definisco $ G = \set{ T \in \Sph{3} | \rho(T) \in A_5} $,
  cioè sono tutti i punti della sfera a cui corrispondono le rotazioni in $ A_5 $.
  $ G $ è un gruppo, infatti se $ T, S \in G $ allora $ \rho(T), \rho(S) \in A_5 $ e
  $ \rho(TS) = \rho(T)\rho(S) \in A_5 $ in quanto $ A_5 $ gruppo. Inoltre $ \Id{} \in G $ in quanto
  $ \rho(\Id{}) \in A_5 $.

  Definisco $ \phi = \rho \big \lvert_G $, per costruzione $ \phi \colon G \to A $ ed è suriettiva.
  Inoltre $ \ker{\phi} = \set{ T \in G | \phi(T) = \Id{}} $, ma $ \phi(T) = \rho(T) $, quindi
  $ T = \pm \Id{} $, cioè $ \ker{\phi} = \set{- \Id{}, + \Id{}} $. Ho perciò la succession
  esatta di gruppi:
  \[
    \begin{tikzcd}
      \Id{} \arrow{r}{} & \ker{\phi} \arrow{r}{} & G \arrow{r}{} & A \arrow{r}{} & \Id{}
    \end{tikzcd}
  \]
  Quindi $ A = \quot{G}{\ker{\phi}} $.
  $ G \subseteq \Sph{3} $ e ha ordine 120, inoltre $ \ker{\phi} $ è normale in $ G $.
  Quello che si trova è $ G \simeq A_5 \times \ker{\phi} $. Questo si verifica formalmente, ma lo si
  intuisce per il fatto che sostanzialmente $ G $ e formato da $ (A, + \Id{}) $ e  $ (A, - \Id{}) $.

  A questo punto posso definire l'azione del gruppo su $ \Sph{3} $:
  \begin{align*}
    G \times \Sph{3} & \to \Sph{3} \\
    (g,x) & \to gx
  \end{align*}
  A questo punto è sensato fare $ \pi \colon \Sph{3} \to \quot{\Sph{3}}{G} $.
\end{proof}



Vale tuttavia il seguente risultato, dimostrato da Perelman nel 2004,
noto come congettura di Poincaré:
\begin{proposition}[Congettura di Poincaré]
  Se $ \M $ è una $ 3 $-varietà topologica compatta, connessa e semplicemente
  connessa tale che $ \forall k \geq 3 $ $ H_k(\M) \cong H_k(\Sph{3}) $ allora $ \M \simeq \Sph{3} $.
\end{proposition}
Questo mostra che il gruppo fondamentale è uno strumento più fine
dei gruppi di omologia.


% TODO:
% 1. Correggere i : con \colon (CAMBIA QUALCOSA? INFORMARSI!)
% NOTA i : servono solo per questioni di relazione, del tipo negli insiemi come "tale che",
% nelle mappe si deve sempre usare \colon (http://tex.stackexchange.com/questions/37789/using-colon-or-in-formulas)
% 2. Correggere i \Leftrightarrow con \iff
% 3. Correggere < e > dei gruppi generati con \langle\rangle
% 4. Aggiungere i figlet delle lezioni
% 5. Aggiungere i simboli usati con \newmathsymb{nome}{simbolo}{Nome che appare}
% 6. Sistemare le mappe (tipo questa sopra), non ho ancora capito come scriverle bene
% 7. Usare al posto di \mathrm{e} \me

\printindex

\end{document}


%%% Local Variables:
%%% mode: latex
%%% TeX-master: t
%%% End:
