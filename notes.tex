\documentclass[10pt, twoside=false, x11names]{scrbook}

\usepackage{amsmath}
\usepackage{amssymb}
\usepackage{amsfonts}
\usepackage{graphicx}
% \usepackage{lmodern}
\usepackage{libertine}
\usepackage{tikz}
\usepackage{epigraph}
\usepackage{lipsum}
\usepackage[utf8]{inputenc}
\usepackage{braket}
\usepackage[italian]{babel}
\usepackage{tikz-cd}
\usepackage{makeidx}
\usepackage{tikz}
\usepackage{mathtools}
\usepackage{supertabular}
\usepackage{array}
\usepackage{textcomp}
\usepackage[T1]{fontenc}

\renewcommand\epigraphflush{flushright}
\renewcommand\epigraphsize{\normalsize}
\setlength\epigraphwidth{0.7\textwidth}

\definecolor{titlepagecolor}{cmyk}{1,.60,0,.40}

\DeclareFixedFont{\titlefont}{T1}{ppl}{b}{it}{0.5in}

\makeatletter
\def\printauthor{%
    {\large \@author}}
\makeatother
\author{%
  Professore: \\
  \emph{Gilberto Bini} \vspace{10pt}\\
  Umile scriba: \\
  \emph{Gabriele Bozzola}
    }

\newcommand\titlepagedecoration{%
\begin{tikzpicture}[remember picture,overlay,shorten >= -10pt]

\coordinate (aux1) at ([yshift=-15pt]current page.north east);
\coordinate (aux2) at ([yshift=-410pt]current page.north east);
\coordinate (aux3) at ([xshift=-4.5cm]current page.north east);
\coordinate (aux4) at ([yshift=-150pt]current page.north east);

\begin{scope}[titlepagecolor!40,line width=12pt,rounded corners=12pt]
\draw
  (aux1) -- coordinate (a)
  ++(225:5) --
  ++(-45:5.1) coordinate (b);
\draw[shorten <= -10pt]
  (aux3) --
  (a) --
  (aux1);
\draw[opacity=0.6,titlepagecolor,shorten <= -10pt]
  (b) --
  ++(225:2.2) --
  ++(-45:2.2);
\end{scope}
\draw[titlepagecolor,line width=8pt,rounded corners=8pt,shorten <= -10pt]
  (aux4) --
  ++(225:0.8) --
  ++(-45:0.8);
\begin{scope}[titlepagecolor!70,line width=6pt,rounded corners=8pt]
\draw[shorten <= -10pt]
  (aux2) --
  ++(225:3) coordinate[pos=0.45] (c) --
  ++(-45:3.1);
\draw
  (aux2) --
  (c) --
  ++(135:2.5) --
  ++(45:2.5) --
  ++(-45:2.5) coordinate[pos=0.3] (d);
\draw
  (d) -- +(45:1);
\end{scope}
\end{tikzpicture}%
}

\newcounter{lecnum}

\newcommand{\lecture}[3]{
   \pagestyle{myheadings}
   \thispagestyle{plain}
   \newpage
   \setcounter{lecnum}{#1}
   \noindent
   \begin{center}
   \framebox{
      \vbox{\vspace{2mm}
    \hbox to \linewidth { {\bf Topologia Algebrica
        \hfill 2016/2017} }
       \vspace{4mm}
       \hbox to \linewidth  { {\Large \hfill Lezione #1: #2  \hfill} }
       \vspace{2mm}
       \hbox { {Agomenti: #3  \hfill} }
      \vspace{2mm}}
   }
   \end{center}
   \markboth{Lezione #1: #2}{Lezione #1: #2}
}
%%% Local Variables:
%%% mode: latex
%%% TeX-master: "notes"
%%% End:


\newtheorem{theorem}{Teorema}[section]
\newtheorem{lemma}[theorem]{Lemma}
\newtheorem{proposition}[theorem]{Proposizione}
\newtheorem{osservation}[theorem]{Osservazione}
\newtheorem{corollary}[theorem]{Corollario}
\newtheorem{definition}[theorem]{Definizione}
\newcounter{exercises}
\newtheorem{exercise}[exercises]{Esercizio}
\newenvironment{proof}{{\textbf{Dimostrazione}:}}{\hfill\rule{2mm}{2mm} \newline}

\newcommand{\R}{\mathcal{R}}
\newcommand{\M}{\mathcal{M}}
\newcommand{\N}{\mathcal{N}}
\newcommand{\Z}{\mathbb{Z}}
\newcommand{\im}[1]{\mathrm{Im}( #1 )}
\renewcommand{\ker}[1]{\mathrm{Ker}( #1)}
\renewcommand{\phi}{\varphi}
\newcommand{\RN}[1][]{\mathbb{R}^#1}
\newcommand{\Id}[1][]{\mathbb{I}_#1}
\newcommand{\Sph}[1][]{\mathcal{S}^#1}
\newcommand{\homoto}{\xrightarrow{\,\smash{\raisebox{-0.65ex}{\ensuremath{\scriptstyle\sim}}}\,}}
\newcommand{\Ab}[1]{\mathrm{Ab}( #1 )}
\newcommand{\incl}{\xhookrightarrow{}}

\newcommand{\vedi}[1]{\emph{vedi} #1}

\newcommand*\quot[2]{{^{\textstyle #1}\big/_{\textstyle #2}}}

\renewcommand\labelitemi{\tiny$\bullet$}

\makeindex

\makeatletter
\usepackage{auxhook}
\AddLineBeginAux{%
  \string\providecommand\string\new@aux@symb[3]{}%
}
\newcommand*{\symb@list}{}
\newlength{\symb@maxwidth}
\newlength{\symb@meaning}
\newcommand*{\new@aux@symb}[3]{%
  \g@addto@macro{\symb@list}{\symb@do{#1}{#2}{#3}}%
  \@newl@bel{SYMB}{#1}{#2}%
  \begingroup
    \settowidth{\dimen@}{#2}%
    \ifdim\dimen@>\symb@maxwidth
      \global\symb@maxwidth=\dimen@
    \fi
  \endgroup
}
\newcommand*{\new@symb}[3]{%
  \@bsphack
  \if@filesw
    \protected@write\@auxout{}{%
      \string\new@aux@symb{#1}{%
        \detokenize\expandafter{\unexpanded{#2}}%
      }{%
        \detokenize\expandafter{\unexpanded{#3}}%
      }%
    }%
  \fi
  \label{symb:#1}%
%  \@ifundefined{SYMB@#1}{%
%    \expandafter\gdef\csname SYMB@#1\endcsname{#2}%
%  }{}%
  \@esphack
}
\newcommand*{\newmathsymb}[3]{%
  \new@symb{#1}{\ensuremath{#2}}{#3}%
}
\newcommand*{\newtextsymb}{%
  \@dblarg\symb@newtext
}
\def\symb@newtext[#1]#2#3{%
  \new@symb{#1}{#2}{#3}%
}
\newcommand*{\symb}[1]{%
  \@ifundefined{SYMB@#1}{%
    \@latex@warning{Symbol `#1' is undefined}%
    \nfss@text{\textbf{??}}%
  }{%
    \csname SYMB@#1\endcsname
  }%
}
% List of symbols
\newcommand*{\symb@head}[1]{\textbf{\large#1}}
\newcommand*{\printsymblist}{%
  \twocolumn[%
    \section*{%
      \centering
      Lista dei simboli e abbreviazioni %
    }%
  ]%
  \thispagestyle{empty}% optional
  \renewcommand*{\arraystretch}{1.1}%
  \settowidth{\dimen@}{\symb@head{Symbol}}%
  \ifdim\dimen@>\symb@maxwidth
    \global\symb@maxwidth\dimen@
  \fi
  \setlength{\symb@meaning}{\linewidth}%
  \addtolength{\symb@meaning}{-\symb@maxwidth}%
  \settowidth{\dimen@}{\symb@head{Page}}%
  \addtolength{\symb@meaning}{-\dimen@}%
  \addtolength{\symb@meaning}{-4\tabcolsep}%
  \tablehead{%
    \symb@head{Simbolo} & \symb@head{Significato} & \symb@head{Pag.}\\[.5ex]%
  }%
  \begin{supertabular}{@{}p{\symb@maxwidth}p{\symb@meaning}c@{}}%
    \symb@list
  \end{supertabular}%
  \clearpage
  \onecolumn
}
\newcommand*{\symb@do}[3]{%
  #2&%
  \sbox0{%
    \renewcommand*{\arraystretch}{1}%
    \begin{tabular}[t]{@{}p{\symb@meaning}@{}}%
      \raggedright
      #3%
    \end{tabular}%
  }%
  \sbox2{%
    \renewcommand*{\arraystretch}{1}%
    \begin{tabular}[b]{@{}b{\symb@meaning}@{}}%
      \raggedright
      #3%
    \end{tabular}%
  }%
  \usebox0 %
  \xdef\symb@raise{\the\dimexpr\ht0-\ht2}%
  &\raisebox{\symb@raise}{\pageref{symb:#1}}\tabularnewline
}
\makeatother


%%% Local Variables:
%%% mode: latex
%%% TeX-master: "notes"
%%% End:


\begin{document}

\begin{titlepage}

\noindent
\titlefont Topologia Algebrica
\epigraph{Zitto e studia.}%
{\textit{Parigi 1905}\\ \textsc{H.\ Poincarè}}
\null\vfill
\vspace*{1cm}
\noindent
\hfill
\begin{minipage}{0.35\linewidth}
    \begin{flushright}
        \printauthor
    \end{flushright}
\end{minipage}
%
\begin{minipage}{0.02\linewidth}
    \rule{1pt}{125pt}
\end{minipage}
\titlepagedecoration
\end{titlepage}

\tableofcontents
\printsymblist

% \lecture{1}{29 Settembre}{General introduction. Homology of a complex. Singular homology.}

\chapter{Omotopia Singolare}

\section{Introduzione}

\subsection{Richiami di algebra}

\newmathsymb{R}{\R}{Anello}
\begin{definition}
  Un \textbf{anello} \index{Anello} è un insieme $ \R $ dotato di due operazioni $ + $ e $ \cdot $ tali che
  $ \R $ sia un gruppo abeliano con l'addizione, sia un monoide con la moltiplicazione
  (ovvero la moltiplicazione è associativa e possiede un elemento neutro\footnote{La richiesta
    di esistenza dell'elemento neutro, cioè dell'unità non è comune a tutti gli autori,
    chi non la richiede chiama anello unitario \index{Anello unitario} la presente
    definizione di anello.}) e goda della proprietà distributiva rispetto all'addizione.
\end{definition}

\begin{definition}
  Un anello si dice \textbf{anello commutativo} \index{Anello commutativo} se l'operazione di moltiplicazione
  è commutativa.
\end{definition}

\begin{definition}
  Un \textbf{campo} \index{Campo} è un anello commutativo in cui ogni elemento non nullo ammette
  un inverso moltiplicativo.
\end{definition}

\begin{definition}
  Sia $ \R $ un anello commutativo si definisce l' \textbf{$ \R $-modulo} \index{$ \R $-modulo}
  un gruppo abeliano $ \M $ equipaggiato con un'operazione di moltiplicazione per uno scalare in $ \R $
  tale che $ \forall v,w \in \M $ e $ \forall a,b \in \R $ vale che:
  \begin{itemize}
  \item $ a(v + w) = av + aw $
  \item $ (a + b)v = av + bv $
  \item $ (ab)v = a(bv) $
  \end{itemize}
\end{definition}

\begin{osservation}
  Se $ \R $ è un campo allora l'$ \R $-modulo è uno spazio vettoriale.
\end{osservation}
Sostanzialmente la nozione di $ \R $-modulo generalizza agli anelli il concetto di spazio vettoriale sui campi.

\begin{osservation}
  Ogni gruppo abeliano $ \mathcal{G} $ è uno $ \Z $-modulo in modo univoco, cioè $ \mathcal{G} $ è un
  gruppo abeliano se e solo e è uno $ \Z $-modulo.
\end{osservation}
\begin{proof}
  Sia $ x \in \mathcal{G} $ si definisce l'applicazione di moltiplicazione per un elemento $ n \in \Z $ come
  \[
    nx =
    \begin{cases}
      \underbrace{ x + x + x + \dots}_{n \text{ volte}} & \text{se } n > 0 \\
      0 & \text{se } n = 0 \\
       \underbrace{ - x - x - x - \dots}_{|n| \text{ volte}} & \text{se } n < 0 \\
    \end{cases}
  \]
  Si verfica banalmente che questa operazione è ben definita e soddisfta
  le giuste proprietà perchè la coppia $ (\mathcal{G}, \Z) $ sia uno $ \Z $-modulo.
  A questo punto non è possibile costruire applicazioni diverse che soddisfino le
  proprietà richieste infatti utilizzando la struttura di anello di $ \Z $:
  $ n x = (1 + 1 + 1 + 1 + \dots) x = x + x + x \dots $, quindi quella definita
  è l'unica possibile.
\end{proof}

\begin{definition}
  Siano $ (X, \cdot) $ e $ (Y, \star) $ due gruppi, un \textbf{omomorfismo} \index{Omomorfismo} è un'applicazione $ f $
  tra $ X $ e $ Y $ che preserva la struttura di gruppo, cioè:
  \[
    \forall u,v \in X \quad f(u \cdot v) = f(u) \star f(v)
  \]
\end{definition}

\begin{osservation}
  Da questa definizione si trova immediatamente che gli omomorfismi si comportano bene nei
  confronti dell'inverso, cioè $ \forall v \in X $ vale che $ f(v^{-1}) = {f(v)}^{-1} $.
\end{osservation}

Voglio studiare gli omomorfismi tra $ \Z $-moduli.

\begin{definition}
  Sia $ \phi: \M \to \N $ un omomorfismo tra gli $ \R $-moduli $ \M $ e $ \N $,
  allora si definisce il \textbf{nucleo} \index{Nucleo} e l'\textbf{immagine} \index{Immagine}:
  \[
    \ker {\phi} := \set{ m \in \M | \phi(m) = 0}  \qquad  \im{\phi} := \set{ m \in \M | \phi(m) = 0}
  \]
\end{definition}

\begin{osservation}
  $ \ker{\phi} $ e $ \im{\phi} $ sono $ \R $-sottomoduli, cioè sono sottoinsiemi di $ \M $ e $ \N $
  che posseggono la struttura di $ \R $-modulo.
\end{osservation}

Se $ M_i $ sono $ \R $-moduli posso fare composizioni di omomorfismi, come:
\[
  \begin{tikzcd}
    \M_1 \arrow{r}{\phi_1} & \M_2 \arrow{r}{\phi_2} & \M_3
  \end{tikzcd}
  \text{o equivalentemente}
  \begin{tikzcd}
    \M_1 \arrow{r}{\phi_2 \circ \phi_1} & \M_3
  \end{tikzcd}
\]
\begin{proposition}
Se vale $ \phi_2 \circ \phi_1 = 0 $ allora $ \im{\phi_1} \subseteq \ker{\phi_2} $.
\end{proposition}
\begin{proof}
  Se $ u \in \im {\phi_2} $ allora $ \exists v \in \M_2 $ tale che $ \phi_1(v) = u $,
  ma $ \phi_2(u) = \phi_2(\phi_1(v)) = (\phi_2 \circ \phi_1)(v) = 0 $ per ipotesi, quindi $ u \in \ker{\phi_2} $.
\end{proof}

Mi interessano questi morfismi perché hanno un preciso significato geometrico che
sarà chiaro successivamente.

\begin{definition}
  Siano $ \M $ un $ \R $-modulo e $ \N $ un suo sottomodulo, allora il \textbf{modulo
  quoziente} \index{Modulo quoziente} di $ \M $ con $ \N $ e definito da:
  \[
    \quot{\M}{\N} := \quot{\M}{\sim} \quad \text{dove } \sim \text{ è definita da: } x \sim y \Leftrightarrow x - y \in \N
  \]
  Dove $ \quot{\M}{\sim} $ è l'insieme delle classi di equivalenza di $ \sim $ equipaggiate
  con operazioni indotte dall'$ \R $-modulo, cioè se $ [u], [w] \in \quot{\M}{\sim} $ e $ a \in \R $:
  \begin{itemize}
  \item $ [u] + [w] = [u + w] $
  \item $ a [u] = [au] $
  \end{itemize}
  In questo caso gli elementi di $ \quot{\M}{\N} $ sono le classi di equivalenza
  $ [m] = \set{ m + n | n \in \N } $.
\end{definition}

Siccome $ \im{\phi} $ è sottomodulo di $ \ker{\phi} $ allora posso prendere
il quoziente:
\[
  \quot{\ker{\phi_2}}{\im{\phi_1}}
\]
Questo è un sottomodulo. Si nota che questo è sensato solo se si impone la condizione
$ \phi_2 \circ \phi_1 = 0 $, altrimenti non c'è l'inclusione e quindi non è possibile fare l'operazione
di quoziente.

A questo punto ci sono due possibilità:
\begin{enumerate}
\item $ {\ker {\phi_2}} \slash {\im{\phi_1}} = 0 $, che significa che $ \ker {\phi_2} = \im{\phi_1} $
  in quanto non ci sono elementi di $ \ker {\phi_2} $ fuori da $ \im{\phi_1} $, dato che l'unica
  classe di equivalenza presente è $ [0] $ significa che $ \forall m \in \ker{\phi_1} \; \exists n \in \im{\phi_2} $
  tale che $ m - n = 0 $, cioè $ m $ e $ n $ coincidono e quindi $ \ker {\phi_2} = \im{\phi_1} $.
\item $ {\ker {\phi_2}} \slash {\im{\phi_1}} \not= 0 $, cioè $ \exists v \in \ker {\phi_2} $
  tale che $ v \not \in \im {\phi_1} $ e quindi $ \im {\phi_1} \subsetneq \ker (\phi_2) $.
\end{enumerate}
Nel primo caso si dice che la successione dei moduli $ \M $ e delle
applicazioni $ \phi $ è \textbf{esatta}\index{Complesso di moduli esatto} in $ \M_2$, nel secondo caso la
successione è detta \textbf{complesso di moduli}\index{Complesso di moduli}.

Sostanzialmente il modulo quoziente quantifica la non esattazza nel punto $ \M_2 $
della successione.

\begin{definition}
  $ H(\M_\bullet) = {\ker {\phi_2}} \slash {\im {\phi_1}} $ è detto \textbf{modulo di omologia} \index{Modulo di omologia}
  del complesso $ M_\bullet = M_1 \longrightarrow M_2 \longrightarrow M_3 $ con le applicazioni $ \phi_1 $ e $ \phi_2 $.
\end{definition}
Per questo  $ H(\M_\bullet) $ quantifica quanto il complesso $ \M_\bullet $ non è esatto.

Questo deriva da un problema topologico concreto.

\begin{definition}
  La coppia $ (X, \mathcal{T}) $ è detta \textbf{spazio topologico}\index{Spazio topologico}
  (generalmente si omette la $ \mathcal{T} $)
  se $ \mathcal{T} $ è una \textbf{topologia}\index{Topologia}, cioè se è una collezione di insiemi di $ X $ tali che:
  \begin{enumerate}
  \item $ \emptyset, X \in \mathcal{T} $
  \item $ \bigcup_{n \in \mathbb{N}} A_n \in \mathcal{T} $ se $ A_n \in \mathcal{T} \; \forall n \in \mathbb{N} $
  \item $ \bigcap_{n \in \set{0,1,\dots,N} } A_n \in \mathcal{T} $ se $ A_n \in \mathcal{T} \; \forall n \in \set{0,1,\dots,N} $
  \end{enumerate}
  Gli elementi di $ \mathcal{T} $ sono detti \textbf{aperti}\index{Insiemi aperti}.
\end{definition}
\begin{osservation}
  Se $ \tau $ è la collezione di tutti i sottoinsiemi di $ X $ allora le proprietà sono automaticamente
  verificate e questa è la \textbf{topologia discreta}\index{Topologia discreta}, invece
  $ \tau = \set{\emptyset, X} $ è una topologia ed è la \textbf{topologia triviale}.
  Infine in $ \RN{n} $ si definisce la \textbf{topologia usuale} che è la topologia in cui gli aperti
  sono iperintervalli aperti del tipo $ (a_1,b_1) \times (a_2, b_2) \times (a_3, b_3) \dots \times (a_n, b_n) $.
  Si dimostra che se si ammettono intersezioni infinite allora la topologia usuale coincide con la topologia
  triviale in $ \RN{n} $.
\end{osservation}

\begin{osservation}
  Uno spazio metrico si può rendere topologico definendo gli insiemi aperti come gli intorni sferici aperti.
\end{osservation}

\begin{osservation}
  Sia $ A \subseteq X $ spazio topologico, si può rendere anche $ A $ uno spazio topologico equipaggiandolo con la
  \textbf{topologia indotta}\index{Topologia indotta} in cui gli aperti sono gli aperti di $ X $ intersecati
  con $ A $.
\end{osservation}

\begin{osservation}
  Uno spazio topologico è \textbf{connesso}\index{Spazio connesso} se si può scrivere come
  unione disgiunta di due suoi aperti.
\end{osservation}

\begin{definition}
  Sia $ X $ uno spazio topologico l'insieme $ \set{A_i | A_i \in X \; \forall i} $ è un \textbf{ricoprimento}\index{Ricoprimento}
  di $ X $ se:
  \[
    \bigcup_{i} A_i = X
  \]
  Se in particolare gli insiemi $ A_i $ sono aperti il ricoprimento è detto \textbf{ricoprimento aperto}.
\end{definition}

\begin{definition}
  Un insieme $ U $ è detto \textbf{compatto}\index{Insieme compatto} se per ogni suo possibile ricoprimento
  aperto ne esiste un sottoinsieme che è un ricoprimento \emph{finito} di $ U $.
\end{definition}

\newmathsymb{homo}{\simeq}{Spazi omeomorfi}
\begin{definition}
  Una mappa tra spazi topologici è detta \textbf{omeomorfismo}\index{Omeomorfismo} se è continua
  e ammette inverso continuo, cioè se è una mappa uno a uno. Se due spazi sono omeomorfi si utilizza
  il simbolo $ \simeq $.
\end{definition}
Siccome gli omeomorfismi sono mappe uno a uno due spazi omeomorfi sono essenzialmente identici. La
relazione di omeomorfismo costituisce una relazione di equivalenza. Molti
degli strumenti sviluppati in questo corso servono a capire se due spazi sono omeomorfi o meno.

\subsection{Omomorfismo tra $ \RN{} $ e $ \RN{N} $}

\begin{definition}
  Un \textbf{arco}\index{Arco} in uno spazio topologico $ X $ tra i punti $ x_0 \in X $ e $ y_0 \in X $
  è una funzione continua da $ I = [0,1] $ a $ X $ tale che $ \alpha(0) = x_0 $ e $ \alpha(1) = y_0 $.
  Si dice che l'arco parte da $ x_0 $ e finisce in $ y_0 $.
\end{definition}

\begin{definition}
  Uno spazio topologico $ X $ è \textbf{connesso per archi}\index{Spazio connesso per archi} se per
  ogni coppia di punto $ x, y \in X $ esiste un arco che parte da $ x $ e termina in $ y $.
\end{definition}

\begin{proposition}
  Se $ f:X \to Y $ è una mappa continua suriettiva tra spazi topologici e se $ X $ è connesso per archi
  allora $ Y $ è connesso per archi. Questo vale in particolare se $ f $ è un omeomorfismo, cioè la
  connessione per archi è una proprietà invariante per omeomorfismi.
\end{proposition}

\begin{proof}
  Siano $ y_0, y_1 $ due punti di $ Y $. La funzione $ f $ è suriettiva, e dunque esistono $ x_0 $ e $ x_1 $ in $ X $
  tali che $ f(x_0)=y_0 $ e $ f(x_1)=y_1 $. Dato che $ X $ è connesso, esiste un cammino $ \alpha:[0,1] \to X $ tale che $ \alpha(0)=x_0 $
  e $ \alpha(1)=x_1 $. Ma la composizione di funzioni continue è continua, e quindi il cammino ottenuto componendo $ \alpha $ con $ f $:
  $ f \circ \alpha : [0,1] \to X \to Y $ è un cammino continuo che parte da $ y_0 $ e arriva a $ y_1 $.
\end{proof}

Si sa inoltre che:
\begin{proposition}
  $ \RN{n} $ è connesso per archi $ \forall n \in \mathbb{N} $.
\end{proposition}

È noto che $ \RN{} \not \simeq \RN{N} $ per $ n \geq 2 $, infatti basta togliere un punto a $ \RN{} $ che diventa sconnesso per archi
mentre $ \RN{N} $ rimane connesso per archi anche togliendogli un punto. In questa dimostrazione ho utilizzato
il seguente risultato fondamentale:
\begin{proposition}
  Se $ f: X \to Y $ è omeomorfismo tra spazi topologici allora $ f \rvert_U : U \to f(U) $ è omeomorfismo per ogni $ U \subseteq X $.
\end{proposition}
Nel caso considerato $ U = {x_0} $, siccome ho trovato un $ U $ per cui la funzione ristretta non è omeomorfismo $ f $
non può essere omeomorfismo. Infatti l'immagine di un punto rimane un punto.

Tuttavia vale anche che $ \RN{2} \not \simeq \RN{N} $ per $ n \geq 3 $, infatti:

\newmathsymb{homoto}{\homoto}{Omeomorfismo}
\begin{proof}
  Per assurdo $ f : \RN{2} \homoto \RN{N} $ è un omeomorfismo con
  $ n \geq 3 $, tolgo un punto da $ \RN{2} $, se $ f $ omeomorfismo anche la restrizione deve essere omeomorfismo, cioè
  $ \forall p \in \RN{2} \quad f:\RN{2} \setminus \set{p} \homoto \RN{N} \setminus \set{f(p)} $.
  Ma $ \RN{2} \setminus \set{p} \simeq \RN{} \times \mathcal{S}^1 $ con la mappa
  $ \underline{x} \mapsto \left( || \underline{x} ||, \frac{\underline{x}}{|| \underline{x} ||} \right) $. In pratica
  sto dicendo che il piano senza un punto è omeomorfo ad un cilindro infinito.
  \emph{Secondo me qui bisogna fare una traslazione e portare p in zero prima di fare questa trasformazione.}
  Analogamente $ \RN{n} \setminus \set{f(p)} \simeq \RN{} \times \Sph{n-1} $. Quindi se esiste un omeomorfismo tra $ \RN{2} $ e
  $ \RN{n} $ significherebbe che $ \RN{} \times \Sph{1} \simeq \RN{} \times \Sph{n-1} $, ma quindi i gruppi fondamentali
  dovrebbero essere isomorfi:
  $ \pi_1 (\RN{} \times \Sph{1}) \simeq \pi_1(\RN{}\times \Sph{n-1}) $ ma
  $ \pi_1 (\RN{} \times \Sph{1}) = \Z $ infatti il gruppo fondamentale di un prodotto è il prodotto dei gruppi
  fondamentali e $ \pi_1(\RN{}) = 1 $, $ \pi_1(\Sph{1}) = \Z $ dato che i lacci omotopicamente distinti
  sono quelli che avvolgono il buco un numero differente di volte. Analogamente $ \pi_1(\RN{}\times \Sph{n-1}) = 1 $
  perché le sfere sono contraibili. Trovo quindi che dovrebbero essere isomorfi $ \pi_1 (\RN{} \times \Sph{1}) = \Z $
  e $ \pi_1(\RN{}\times \Sph{n-1}) = 1 $ che è assurdo.
\end{proof}

Ho quindi dedotto proprietà topologiche a partire da considerazioni algebriche (con il gruppo fondamentale).
Il gruppo fondamentale è un invariante algebrico per problemi topologici.

\newmathsymb{fondgroup}{\pi_1}{Gruppo fondamentale}
\begin{definition}
  Si definisce il \textbf{gruppo fondamentale}\index{Gruppo fondamentale} di uno spazio topologico $ X $
  connesso per archi attorno al punto $ x_0 \in X $
  \[
    \pi_1 (X, x_0) = \quot{\set{ g: \Sph{1} \to X | g \text{ continua}, g(1) = x_0}}{\sim}
  \]
  e $ \sim $ è la relazione di omotopia: $ g_1 \sim g_2 $ se $ \exists G: \mathcal{S}^1 \times I \to X  $ tale che
  $ G(z,0) = g_1(z), G(z,1) = g_2(z), G(1,t) = x_o $ con $ G $ continua. In questo vedo $ \Sph{1} $ come sottospazio
  di $ \RN{2} $ con la topologia indotta (il punto $ 1 $ è un punto della circonferenza vedendola come
  insieme nello spazio complesso $ \Sph{1} = \set{ z \in \mathbb{C} | |z| = 1} $).
\end{definition}
Sostanzialmente il gruppo fondamentale è l'insieme dei lacci quozientato rispetto alla relazione di omotopia.
Infatti $ g $ è un laccio dato che è un arco e il punto di partenza e il punto di arrivo necessariamente
coincidono dato che $ g $ è definito su $ \Sph{1} $.
Questo perché l'insieme dei lacci non è strutturabile come gruppo in quanto il laccio costante non è
l'unità.

Ora voglio mostrare per assurdo che non esiste omomorfismo tra $ \RN{3} $ e $ \RN{N} $.

\begin{proof}
  Come nel caso precedente suppongo esiste $ f $ omeomorfismo tra $ \RN{3} $ a $ \RN{n} $,
  tolgo $ q $ da $ \RN{3} $ e $ f(q) $ da $ \RN{n} $, quindi ottengo
  l'omomorfismo tra $ \RN{} \times \Sph{2} \simeq \RN{} \times \Sph{n-1} $, ma i gruppi fondamentali
  associati sono banali, quindi sono isomorfi, e non è posisbile replicare il ragionamento utilizzato sopra.
\end{proof}

Poincaré introdusse i gruppi di omotopia superiore.

\begin{definition}
  Si definiscono i \textbf{gruppi di omotopia superiore}\index{Gruppi di omotopia superiore} di uno spazio topologico $ X $
  attorno al punto $ x_0 $ per $ k \geq 2 $:
  \[
    \pi_k(X) (X, x_0) = \quot{\set{ g: \Sph{k} \to X | g \text{ continua}, \; g(p_0) = x_0}}{\sim}
  \]
  Con $ p_0 \in \Sph{k} $ e $ \sim $ relazione di omotopia.
\end{definition}
Studiare i gruppi di omotopia superiore è un problema aperto della topologia moderna.
Tuttavia si sa che:
\begin{enumerate}
\item $ \pi_k(\Sph{m}) = 1 \quad \text{per} \quad 1 \leq k < m \quad (m > 2)$
\item $ \pi_m(\Sph{m}) \simeq \Z \quad \text{per} \quad k = m $
\item $ \pi_1(\Sph{2}) = 1 $
\item $ \pi_2(\Sph{2}) \simeq \Z $
\item $ \pi_3(\Sph{2}) \simeq \Z $\footnote{Questo da origine alla fibrazione di Hopf che ha molte applicazioni in fisica.}
\end{enumerate}

\begin{definition}
  Sia $ A \subseteq X $ con $ X $ spazio topologico $ i: A \to X $ si definisce mappa di \textbf{inclusione}\index{Inclusione}
  e si scrive $ i: A \incl X $ se $ \forall a \in A $ vale che $ i(a) = a $.
\end{definition}


Anche se non so calcolare i gruppi di omotopia superiore non vorrei buttarli via\dots
Vorrei degli invarianti algebrici per problemi topologici, come i gruppi di omotopia.


\subsection{Omologia}

Uso la teoria dell'omologia che mi permette di semplificare i problemi. La teoria
dell'omologia serve ad associare agli spazi topologici degli oggetti algebrici
meno complicati dei gruppi di omotopia.
Ci sono varie possibilità:
\begin{itemize}
  \item Omologia singolare
  \item Omologia cellulare
  \item Omologia persistente\footnote{Questa ha numerose applicazioni pratiche, come la ricostruzione di immagini.}
  \item Omologia simpliciale
\end{itemize}
Ma cosa è l'omologia? Assocerò ad ogni spazio topologico (anche patologico) gruppi abeliani e omomorfismi a partire
da applicazioni continue tra due spazi topologici. In tutto questo lavoro sempre con anello di base $ \Z $, che
quindi rimane sottinteso a meno di scriverlo esplicitamente.

\begin{definition}
  In $ \RN{k+1} $ si definisce il \textbf{simplesso standard} $ \Delta_k $ l'insieme:
  \[
    \Delta_k = \set{(x_1,x_2,\dots) \in \RN{k+1} | \forall i \quad 0 \leq x_i \leq 1 \text{ e } \sum_{i=1}^{k+1}x_i = 1}
  \]
\end{definition}

\begin{osservation} Alcuni esempi sono:
  \begin{itemize}
  \item $ \Delta_0 $ è un punto.
  \item $ \Delta_1 $ è un segmento omeomorfo a $ [0,1] $.
    \begin{figure}[htbp]
      \centering
      \begin{tikzpicture}
        \draw[-Latex] (0,0) -- (3,0);
        \draw[-Latex] (0,0) -- (0,3);
        \draw[thick] (0,2) -- (2,0);
        \node[below] () at (2,0) {1};
        \node[left] () at (0,2) {1};
      \end{tikzpicture}
      \caption{1-Simplesso standard}
      \label{fig:lez1:1_standard_simplex}
    \end{figure}
  \end{itemize}
\end{osservation}

\begin{definition}
  Dato uno spazio topologico $ X $ si definisce il \textbf{$ k $-simplesso singolare}
  in $ X $ come un'applicazione continua $ g: \Delta_k \to X $.
\end{definition}
Spesso conviene identificare il $ k $-simplesso con la sua immagine in $ X $.
In quesot modo uno $ 0 $-simplesso è un punto in $ X $, mentre un $ 1 $-simplesso singolare potrebbe essere sia un segmento che un punto (se la mappa è costante). Siccome
il simplesso deforma è detto singolare.

Voglio costruire un complesso di gruppi abeliani e definire l'omologia singolare come l'omologia di tale complesso.

$ S_\cdot $ è il compesso, cioè: $ \dots \to S_{k+1}(X) \to S_k() \to S_{k-1} \to \dots \to S_0(X) $, dove
\begin{align*}
  S_k(X) ={}& \{\text{combinazioni lineari finite a coefficienti interi: } \\
         & \sum_g n_g g \;|\; n_g \in \Z, g \; k-\text{simplessi singolari di } X \}
\end{align*}
$ S_k(X) $ è un gruppo abeliano con l'operazione somma definita naturalmente:
\[
  \sum_g n_g g + \sum_h n_h =   \sum_g n_g g + \sum_g n_g^\star = \sum_g (n_g + n_g^\star)g
\]
Ad esempio:
\[
  (n_1 g_1 + n_2 g_2 + 2 n_3 g_3) + (m_1 g_1 + m_4 g_4) = (n_1 + m_1)g_1 + n_2 g_2 + 2 n_3 g_3 + m_4 g_4
\]
Questa è una somma con tutte le giuste proprietà. Lo zero è la catena con tutti i coefficienti nulli,
mentre l'inverso è la catena con i coefficienti opposti.
Queste catene sono chiamate $ k $-\textbf{catene singolari}.

Ad esempio:
Se $ k = 0 $ $ S_0(X) $ sono catene di punti ($ g_0 : \Delta_0 \to X $)
\[
  S_0(X) = \set { \sum n_i p_i | n_i \in \Z, \; p_i \in X}
\]
Ora devo introdurre le applicazioni tra i vari $ S_k $, queste applicazioni saranno il bordo.

Definisco $ h: \Delta_1 \to X $ in modo tale che $ h(\Delta) = \alpha $ dove $ \alpha $ è un \textbf{arco}.

\begin{definition}
  Uno spazio topologico $ X $ si dice \textbf{connesso per archi} \index{Spazio connesso per archi} se $ \forall x, y \in X $ esiste
  un arco con punto iniziale $ x $ e punto finale $ y $.
\end{definition}

\begin{figure}[htbp]
  \centering
  \begin{tikzpicture}
    \draw[-Latex] (0,0) -- (2,0);
    \draw[-Latex] (0,0) -- (0,2);
    \draw[thick] (0,1) -- (1,0);
    \draw plot [smooth cycle] coordinates {(5,2) (6,3) (7,3) (6,0) (4,0)};
    \node () at (7,1) {$ X $};
    \draw plot [smooth, tension = 1] coordinates {(6,2) (5,1) (5.5,0.5)};
    \node[circle] () at (6,2) {\textbullet};
    \node[circle] () at (5.5,0.5) {\textbullet};
    \draw[->] (0.75,0.5) -- (4.75,1);
  \end{tikzpicture}
  \caption{1-Simplesso singolare}
  \label{fig:lez1:1_standard_simplex_with_arc}
\end{figure}

Posso ottenere una $ 0 $-catena prendendo i punti estremi dell'arco.

\begin{definition}
  Sia $ \Delta_k $ un $ k $-simplesso standard con $ k \geq 0 $ si definisce l'operatore \textbf{faccia} come
  la mappa $ F_i^{\;k} $ da $ \Delta_{k-1} $ a $ \Delta_k $ tale che $ F_i^{\;k}(\Delta_{k-1}) $ è una faccia di $ \Delta_k $.
\end{definition}

Ad esempio per $ k = 2 $ $ \Delta_2 = \set{ (x_1,x_2,x_3) \in \RN{3} | x_1 + x_2 + x_3 = 1, \; 0 \leq x_i \leq 1 \; \forall i} $,
si definisce la base $ e_0 = (1,0,0) \; e_1 = (0,1,0) \; e_2 = (0,0,1) $, voglio vedere il bordo del triangolo
come facce.

\begin{figure}[htbp]
  \centering
  \begin{tikzpicture}
    \draw[-Latex] (0,0) -- (4,0);
    \draw[-Latex] (0,0) -- (0,4);
    \draw[-Latex] (0,0) -- (-2,-2);
    \node[below] () at (2,0) {$ e_1 $};
    \node[right] () at (0,2) {$ e_2 $};
    \node[right, below] () at (-1,-1) {$ e_0 $};
    \draw (-1,-1) -- (2,0);
    \draw (0,2) -- (2,0);
    \draw (0,2) -- (-1,-1);
    \draw (5,0) -- (9,0) -- (7,3) -- cycle;
    \node[left] () at (5,0) {$ e_0 $};
    \node[right] () at (9,0) {$ e_2 $};
    \node[above] () at (7,3) {$ e_1 $};
    \node[below] () at (7,0) {$ F_2^{(2)}(\Delta_1) $};
    \node[left] () at (6,2) {$ F_1^{(2)}(\Delta_1) $};
    \node[right] () at (8,2) {$ F_0^{(2)}(\Delta_1) $};
  \end{tikzpicture}
  \caption{Azione dell'operatore faccia}
  \label{fig:lez1:standard_simplex_faces}
\end{figure}

[FIGURA, CONSIDERAZIONI]

\begin{exercise}
  Dimostrare che se $ [\cdot, \cdot] $ indica l'inviluppo convesso allora:
  \begin{enumerate}
  \item Per $ j > i $ vale che $ F_i^{\; k+1} \circ F_i^{\; k} = [e_0, \dots, e_i, \dots, e_j, \dots, e_k ] $.
  \item Per $ j \leq i $ vale che $ F_i^{\; k+1} \circ F_i^{\; k} = [e_0, \dots, e_j, \dots, e_{i+1}, \dots, e_k ] $.
  \end{enumerate}
\end{exercise}

Dato un $ k $-simplesso singolare $ \sigma: \Delta_k \to X $ si definisce la mappa $ \sigma^{(i)} = \sigma \circ F_i^{\; k} $.

[FIGURA]

\begin{definition}
  Si definisce il \textbf{bordo} di un $ k $-simplesso singolare come $ \partial_k \sigma = \sum_{i=0}^{k}(-)^i \sigma^{(i)} $.
\end{definition}

Per $ k = 1 $ $ \partial_1 \sigma = p_1 - p_0 $ infatti $ \sigma^{0} = \sigma \circ F_0^{\; 1} = \sigma(1) = p_1 $ e $ \sigma^{0} = \sigma \circ F_1^{\; 1} = \sigma(0) = p_0 $.\footnote{Tecnicamente si intende $ p_0 = \partial_1 \sigma^{(0)}(1) $ e $ p_1 = \partial_0 \sigma^{(1)}(1) $.}

Allora definisco $ \partial_k: S_k(X) \to S_{k-1}(X) $ infatti per linearità $ \partial_k \left( \sum_g n_g g\right) = \sum_g n_g \partial_k g $.

Devo mostrare che $ \partial_k $ è un omomorfismo.

\begin{proof}
  \begin{gather*}
  \partial_k \left( \sum_g n_g g + \sum_g m_g g\right) = \partial_k \left( \sum_g(m_g + n_g)g \right) = \sum_g (m_g + n_g) \partial_k g = \\
  = \sum_g n_g \partial_k g + \sum_g m_g \partial_k g = \partial_k \left( \sum_g n_g g\right) + \partial_k \left( \sum_g m_g g \right)
\end{gather*}
\end{proof}

Quindi il complesso è costituito da:
\[
  \begin{tikzcd}
   \dots \arrow{r}{\partial_{k+1}} & S_k(X) \arrow{r}{\partial_k} & S_{k-1}(X) \arrow{r}{\partial_{k-1}} & \dots
  \end{tikzcd}
\]

Devo verificare che $ \partial_k \circ \partial_{k+1} = $. Spesso come notazione si pone $ \partial^2 = 0 $.

\begin{proof}
  Se $ \sigma $ è un $ k $-complesso singolare $ \sigma : \Delta_k \to X $:
  \begin{align*}
    & \partial_k \circ \partial_{k+1} \sigma = \partial_k \left( \sum_{j=0}^{k+1}(-)^j (\sigma \circ F_j^{\; k+1}) \right) =  \sum_{j=0}^{k+1}(-)^j \partial_k (\sigma \circ F_j^{\; k+1}) &= \\
                    & = \sum_{j=0}^{k+1} (-)^j \sum_{i=0}^k (-)^i (\sigma \circ F_j^{\; k+1}) \circ F_i^{\; k} = \sum_{j = 0}^{k+1} \sum_{i = 0}^{k} (-)^{j+i} \sigma \circ F_j^{\; k+1} \circ F_{j}^{\; k} &= \\
                    & = \sum_{0 \leq i < j \leq k + 1} (-)^{i+j} \sigma \circ F_j^{\; k+1} \circ F_i^{\; k} + \sum_{0 \leq j < i \leq k} (-)^{i+j} \sigma \circ F_j^{\; k+1} \circ F_i^k &= \\
    & = \sum_{0 \leq i < j \leq k + 1} (-)^{i+j} \sigma \circ F_j^{\; k+1} \circ F_i^{\; k} + \sum_{0 \leq j < i \leq k} (-)^{i+j} \sigma \circ F_{i+1}^{\; k+1} \circ F_j^{\; k} & =  \\
    & = 0
  \end{align*}
\end{proof}

% \lecture{2}{4 Ottobre}{Banane}

Sia $ X $ uno spazio topologico, voglio definire l'omologia singolare $ H_k(X) $, cioè il $ k $-esimo gruppo di omologia
singolare. Costruisco il complesso $ (S_\bullet(X), \partial) $ con:
\[
  S_k(X) = \set{ \sum_g n_g g | g \text{ continua, } n_g \in \Z }
\]
E $ \partial_k : S_k(X) \to S_{k-1}(X) $ applicazione di bordo con $ \partial_k(g) = \sum_{i=0}^k(-)^ig^{(1)} $ con $ g: \Delta_k \to X $, e poi lo estendo per
linearità. Si trova che $ g^{(1)} = g \circ F_i^{\; k} $.

Il lemma fondamentale è $ \partial_{k-1} \circ \partial_k = 0 $ quindi $ S_k \overset{\partial_k}{\to} S_{k-1} \overset{\partial_{k-1}}{\to} S_{k-2} $ è un complesso
e $ \partial_k \circ \partial_{k-1} $ è la mappa dalle catene di $ S_k $ a quelle di $ S_{k-2} $.

$ (S_\bullet(X), \partial) $ è un complesso di gruppi abeliani o $ \Z $- moduli liberi.

Siccome vale $ \partial^2 = 0 $ posso calcolare l'omologia di $ (S_\bullet(X),\partial_\bullet) $:
\[
  H_k(S_\bullet(X)) = \quot{\ker{\partial_k}}{\im{\partial_{k+1}}}
\]
Vale che $ \ker{\partial_k} = \set{c \in S_K(X) | \partial_k(c) = 0} $, cioè le $ k $-catene con
bordo nullo, questi sono chiamati $ k $-cicli.

\begin{definition}
  Sia $ S_\bullet(X) $ un complesso di moduli, gli elementi di $ \ker{\partial} $ sono detti
  \textbf{$ k $-ciclo}, \index{$ k $-ciclo} i quali sono quindi le $ k $-catene
  con bordo nullo.
\end{definition}

Come notazione si pone $ Z_k(X) $ come il gruppo abeliano dei $ k $-cicli: $ Z_k(X)
= \ker{\partial} $.

Si pone invece $ B_k(X) $ come l'insieme dei bordi, cioè le $ k $-catene singolari
che sono immagini di $ k+1 $-catene, cioè esplicitamente:
\[
  B_k(X) = \set{\eta \in S_k(X) | \exists b \in S_{k+1}(X), \partial b = \eta}
\]

Per definizione si ha quindi che $ H_k(X) = \quot{Z_k(X)}{B_k(X)} $, cioè il gruppo
di omotopia è formato dai cicli modulo i bordi.

Esplicitamente gli elementi di $ H_k(X) $ sono classi di equivalenza con rappresentante:
Sia $ [c] \in H_k(X) $ quindi vale che $ \partial c = 0 $, sia inoltre $ c_1 \in [c] $ allora
$ c_1 - c \in B_k(X) $ e $ \partial c_1 = 0 $ quindi esiste $ b $ tale che $ c_1 - c = \partial b $.

\begin{definition}
  Due elementi $ a,b $ si dicono \textbf{omologhi} \index{Elementi omologhi} se differiscono per un bordo.
  \[
    a \sim_{hom} b \Leftrightarrow \exists c \; | \; \partial_k c
  \]
\end{definition}

\begin{osservation}
  Vale che $  H_k(X) = 1 \Leftrightarrow B_k(X) = Z_k(X) $, cioè se ogni ciclo è un bordo. In generale si ha che $ B_k(X) \subseteq Z_k(X) $
  e possono esserci cicli che non sono immagini di bordi.
\end{osservation}

Scopo del corso è studiare $ H_k(X) $.

\begin{proposition}
  Sia $ X $ uno spazio topologico connesso per archi, allora $ H_0 \cong \Z $, cioè è uno $ \Z $-modulo libero di rango 1.
  In effetti $ H_0(X) $ \emph{conta} le componenti connesse per archi e quindi da informazioni di natura geometrica.
\end{proposition}

\begin{proof}
  Dalla definizione di gruppo di omologia: $ H_0(X) = \quot{Z_0(X)}{B_0(X)} $.
  Ma $ Z_0(X) = \set{ c \in S_o(X) | \partial_0 c = 0} $ e $ S_0(X) = \set{ \sum n_i p_i | n_i \in \mathbb{N}, p_i \in X} $.
  Tecnicamente uno $ 0 $-simplesso è una mappa $ \sigma_0 : \Delta_0 \to X $ tale che manda $ \Delta_0 = 1 $ in $ \sigma_0(1) = p_0 $ e per
  questo è naturale l'identificazione con i punti dello spazio topologico.
  Sia $ c \in S_0(X) $ allora $ c = \sum n_i p_i $, e vale che $ \partial_0(c) = \sum n_1 \partial_0 (p) = 0 $, infatti per definizione
  $ \partial_0 : S_0(X) \to s_{-1}(X) $, ma $ S_{-1}(X) $ in ogni complesso è banale, cioè $ S_{-1}(X) \cong 0 $.
  Quindi per ora ho che:
  \[
    H_0(X) = \quot(S_0(X))(B_0(X))
  \]
  Per definizione $ B_0(X) = \set{ x \in S_0(X) | \exists \alpha \in S_1(X), \partial_1(\alpha) = x} $, $ \alpha $ è una catena. Sia $ p_0 \in X $, allora
  $ q \sim_{hom} p $ se e solo se $ \exists \alpha \in S_1(X) $ tale che $ q - p_0 = \partial_1 \alpha $. Per questo motivo i punti sono tutti omologhi,
  essendo $ X $ connesso per archi esiste un arco che connette $ q $ e $ p_0 $, infatti per definizione gli archi sono applicaizoni
  dall'intervallo a $ X $ che hanno come bordo $ q - p_0 $. Esiste quindi un'unica classe di equivalenza.

  \begin{definition}
    Si definisce inoltre la mappa \textbf{grado} \index{Grado} come l'applicazione che manda una catena in $ S_0(X) $ nella somma
    dei suoi coefficienti:
    \begin{align*}
      \mathrm{deg}: & S_0(X)    & \to & \Z \\
                    & \sum n_i p_i & \mapsto & \sum n_1
    \end{align*}
  \end{definition}

  \begin{theorem}[Teorema fondamentale degli omomorfismi]
    Sia $ f: \mathcal{G}_1 \to \mathcal{G}_2 $ un omomorfismo tra gruppi abeliani, allora vale che:
    \[
      \quot{\mathcal{G}_1}{\ker{f}} = \im{f}
    \]
  \end{theorem}

  \begin{proposition}
    La mappa grado gode di alcune proprietà:
    \begin{enumerate}
    \item deg è un omomorfismo di gruppi abeliani
    \item deg è suriettivo
    \item $ \ker{\mathrm{deg}} \cong B_0(X) $
    \end{enumerate}
    Se dimostro questa proprietà utilizando il primo teorema degli omomorfismi \dots

    Dimostro quindi questa proposizione.
    \begin{proof}
      Sia $ c_1 = \sum n_i p_i $ e $ c_2 = \sum m_i q_i $, devo mostrare che $ \mathrm{deg}(c_1 + c_2) = \mathrm{deg}(c_1) + \mathrm{deg}(c_2) $,
      cioè che $ c_1 + c_2 = \sum n_i p_i + \sum m_i q_i = \sum (n_i + m_i)r_i $ dove $ r_i $ è quello comune tra le catene, oppure è zero se
      l'elemento è presente in solo uno delle due catene.

      La mappa è suriettiva, basta prendere un punto: $ m \in \Z $ e $ \mathrm{deg}^{-1}(m) = mp $

      Mostro che $ \ker{\mathrm{deg}} = B_0(X) $. Prendo $ c $ tale che $ \mathrm{deg}(c) = 0 $, ma
      $ c = \sum n_i p_i $ quindi $ \sum n_i = 0 $, allora $ c \in B_0(X) $?
      Se $ E b \in S_1(X) $ con $ \partial_1 b = c $. Prendo $ p_0 $ e altri punti $ p_1, p_2, p_3, \dots $, ci sono archi
      $ \lambda_1, \lambda_2, \lambda_3, \dots $ che li uniscono a $ p_0 $.Provo a costruire $ b $ in questo modo. Siano
      $ \lambda_i : [0,1] \to X $ con $ \lambda_i(0) = p_i $ e $ \lambda_i(1) = p_i $ considero $ c - \partial\left(\sum n_1 \lambda_i \right) =
      c - \sum n_i \partial \lambda_i = c - \sum n_i (p_i - p_0) = c - \sum n_i p_i = \sum n_i p_0 = 0 $. Siccome per ipotesi $ p_o \in \ker{\mathrm{deg}} $
      e $ c = \sum n_i p_i $ allora $ c = \partial(\sum n_i \lambda_i) $ quindi $ \sum n_i \lambda_i = b $ da cui $ \ker{\mathrm{deg}} \subseteq B_0(X) $.
      Mi rimane da mostrare che $ B_0(X) \subseteq \ker{\mathrm{deg}} $, infatti ora mostro che se $ c \in B_0(X) $ allora
      $ \mathrm{deg}(c) = 0 $.$ c = \partial b $ ma se $ \lambda_i $ sono gli archi $ b = \sum m_i \lambda_i $ quindi  $ \partial b = \sum n_i \partial \lambda_i $
      ma $ \partial \lambda_i = \lambda_i(1) - \lambda_i(0) $ e l'azione dell'opertaore grado è quella di sommare i coefficienti, quindi
      \[
        \mathrm{deg}(c) = \mathrm{deg}(\partial b) = \sum n_i \mathrm{deg}(\partial \lambda_i) = 0
      \]
    \end{proof}
  \end{proposition}
  Per questo $ H_0 (X) \cong \Z $ generato dalla classe $ [p] \; \forall p \in X $ (con $ X $ connesso per archi).
\end{proof}
Se ci sono più componenti connesse per archi posso ripetere il ragionamento senza connettere componenti
distinte, quindi trovo che:
\[
  H_0(X) \cong \Z^{N_c}
\]
Dove $ N_c $ è il numero di componenti connesse per archi di $ X $ con $ N_c < + \infty $.

Cosa si può dire invece su $ H_1(X) $?

Sia $ X $ spazio topologico e $ x_0 \in X $, allora alla coppia $ (X, x_0) $ si associa il gruppo fondamentale
$ \pi_1(X,x_0) $. In generale il gruppo fondamentale non è abeliano, allora conviene studiare la versione abelianizzata:
$ \Ab{\pi_1(X,x_0)} = \quot{\pi_1(X,x_0)}{\pi_i(X,x_0)'} $ dove $ ' $ indica il \textbf{gruppo derivato}, \index{Gruppo derivato} cioè il gruppo
generato dai comutatori.
\[
  \pi_1(X,x_0)' = [\pi_1(X,x_0), \pi_1(X,x_0)]
\]
Se $ X $ è connesso per archi allora $ \Ab{\pi_1(X,x_0)} \cong H_1(X) $, quindi conoscendo il gruppo fondamentale si può calcolare il
primo gruppo di omologia.

% \lecture{3}{6 Ottobre}{Banane}

\subsection{Richiami sul gruppo fondamentale}

\begin{definition}
  Sia $ X $ uno spazio topologico e $ x_0 $ un suo punto, allora un \textbf{laccio}\index{Laccio} è un arco in $ X $
  avente come punto di partenza e punto di arrivo il punto $ x_0 $. Un laccio $ c_{x_0} $ si dice \textbf{costante} se $ \forall t \in I $
  $ c_{x_0}(t) = x_0 $ con $ x_0 \in X $.
\end{definition}

Vorrei strutturare l'insieme dei lacci in uno spazio $ X $ come un gruppo con l'operazione di giunzione
e avente come unità il laccio costante. Questo non si riesce a fare perché il laccio costante non sempre
la giunzione di un laccio con il suo inverso è il laccio costante. Per questo si passa al quoziente
rispetto la relazione di omotopia.

\begin{definition}
  Sia $ X $ uno spazio topologico e $ x_0 \in X $ un suo punto, allora la coppia $ (X, x_0) $ è detta \textbf{spazio topologico puntato}.
  \index{Spazio topologico puntato}
\end{definition}

\begin{definition}
  Sia $ (X, x_0) $ uno spazio topologico puntato e $ f: I \to X $ una mappa continua tale che $ f(0) = f(1) = x_0 \; \forall t \in I $,
  si dice che una funzione continua $ g $ è \textbf{omotopicamente equivalente} a $ f $ ($ g \sim_H f $) se esiste una funzione
  continua $ F: I \times I \to X $ tale che;
  \begin{itemize}
  \item $ F(0,x) = f(x) \; \forall x \in I $
  \item $ F(1,x) = g(x) \; \forall x \in I$
  \item $ F(s,0) = x_0 \; \forall s \in I $
  \item $ F(s,1) = x_0 \; \forall s \in I $
  \end{itemize}
  La relazione $ \sim_H $ è detta \textbf{relazione di omotopia} \index{Relazione di omotopia} \index{Omotopia! \vedi{Relazione di omotopia}}
  e si dimostra essere una relazione di equivalenza.
\end{definition}

\begin{figure}[htbp]
  \centering
  \begin{tikzpicture}
    \draw (0, 0) rectangle (3,3);
    \draw[-Latex] (-0.5, 1) -- (-0.5, 2);
    \draw[-Latex] (1, -0.5) -- (2, -0.5);
    \node[left] () at (-0.5, 1.5) {$ t $};
    \node[below] () at (1.5, -0.5) {$ x $};
    \node[right] () at (0, 1.5) {$ x_0 $};
    \node[left] () at (3, 1.5) {$ x_0 $};
    \node[above] () at (1.5, 0) {$ f(x) $};
    \node[below] () at (1.5, 3) {$ g(x) $};
    \draw (0,2) -- (3,2);
    \draw[-Latex] (0, 2) -- (1.5, 2);
    \node[below] () at (1.5, 2) {$ F(x,s) $};
    \draw (6,1.5) circle [x radius=2, y radius=1];
    \node[above] () at (6, 2.5) {$ f $};
    \draw (6, 1.5) circle (0.5);
    \node[above] () at (6, 1.5) {$ g $};
    \draw[-Latex] (4.5, 1.5) -- (5, 1.5);
    \draw[-Latex] (7.5, 1.5) -- (7, 1.5);
  \end{tikzpicture}
  \caption{Omotopia: deforma $ f $ in $ g $ in modo continuo.}
  \label{fig:lez3:homotopy}
\end{figure}

Si definisce l'insieme;
\[
  \pi_1(X,x_0) = \quot{\set{f:I \to X | f \text{ continua}, f(0) = f(1) = x_0}}{\sim_H}
\]

Questo insieme può essere equipaggiato con un'operazione di somma facendolo diventare un gruppo,
questo è il \textbf{gruppo fondamentale} \index{Gruppo fondamentale}, tale operazione è:
Siano $ [f], [g] \in \pi_i(X,x_0) $ si definisce $ [f][g] = [f \star g] $, dove l'operazione $ \star $ è
il \textbf{cammino composto}, o \textbf{giunzione}, \index{Cammino composto}
\index{Giunzione!\vedi{Cammino composto}} definita da:
\[
  (f \star g)(t) =
  \begin{cases}
    f(2t) & \text{se } 0 \leq t \leq \frac{1}{2} \\
    g(2t -1) & \text{se } \frac{1}{2} \leq t \leq 1
  \end{cases}
\]
L'elemento neutro di questa operazione è il cammino costante $ 1 = [C_{x_0}] $ con
$ C_{x_0}(t) = x_0 \; \forall t $.
L'inverso di un elemento invece è $ [f]^{-1} = [\bar{f}] $ dove $ \bar{f} $ è il
cammino percorso in verso opposto, cioè definito da $ \bar{f}(t) = f(1-t) $, in
questo modo $ \bar{f}(0) = f(1) $ e $ \bar{f}(1) = f(0) $.

Proprietà:
\begin{itemize}
\item $ \pi_1(X,x_0) $ è invariante omotopico, cioè se $ X \sim_H Y $, cioè se
  \[
    \exists f: X \to Y, g: Y \to X \; | f \circ g \sim_H 1_Y \text{ e }  g \circ f \sim_H 1_X
  \]
  allora $ \pi_1(X,x_o) \cong \pi_1(Y,f(x_0)) $. Questo in particolare porta alla seguente
  utile osservazione:
  \begin{osservation}
    Se due spazi topologici puntati hanno gruppi fondamentali non isomorfi allora
    non possono essere omotopicamente equivalenti.
  \end{osservation}
\item Se $ X $ è \textbf{contraibile} \index{Spazio contraibile} (cioè è
  omotopo ad un punto) allora vale che $ \pi_1(X,x_0) \cong 1 $, cioè il gruppo
  fondamentale è banale.
\item Si dimostra che:
  \begin{proposition}
    Se uno spazio tologico $ X $ è connesso per archi allora tutti i gruppi fondamentali
    degli spazi puntati $ (X,x_0) $ sono isomorfi, cioè si può omettere la dipendenza da $ x_0 $.
  \end{proposition}
  Questo intuitivamente è vero perché se gli spazi sono connessi per archi allora esistono cammini
  che collegano qualunque coppia di punti.
\end{itemize}

\begin{definition}
  Uno spazio topologico connesso per archi si dice \textbf{semplicemente connesso} \index{Semplicemente connesso}
  se il suo gruppo fondamentale è banale.
\end{definition}

\begin{osservation}
  Non tutti gli spazi semplicemente connessi sono contraibili, come ad esempio $ \Sph{2} $.
\end{osservation}

\begin{itemize}
\item $ \pi_1(\Sph{1}) \cong \Z $, infatti si può costruire la mappa:
  \[
    \begin{aligned}
      \sigma: &\; I \to \Sph{1} \\
      &\;  t \mapsto \mathrm{e}^{2 \pi i t}
     \end{aligned}
   \]
  Questa è tale che $ \sigma(0) = \sigma(1) = 1 $ quindi $ [\sigma] \in \pi_1(\Sph{1}) $ e $ \pi_1(\Sph{1}) \to \Z $ con $ [\sigma] \mapsto 1 $.
  Ogni elemento è multiplo di $ \sigma $ e il fattore di proporzionalità conta il numero di avvolgimenti
  del cammino.
\item $ \pi_1(X \times Y) \cong \pi_1(X) \times \pi_1(Y) $
\item Il gruppo fondamentale si calcola o partendo da gruppi omotopi oppure utilizzando il \textbf{teorema di Seifert–van Kampen}, \index{Teorema di Seifert–van Kampen}
  il quale fornisce un metodo algoritmico per il calcolo.
\end{itemize}

Ad esempio:
$ V_0 := \Sph{2} \quad V_g := P_{\frac{4g}{N}} \text{ con } g \in \mathbb{N}, g \geq 1 $ e $ P_{\frac{k}{N}} $ poligono con $ k $ lati e con idenfiticazioni.
Nel caso $ g = 1 $ si ottiene un toro piatto.
\begin{figure}[htbp]
  \centering
  \begin{tikzpicture}
    \draw (0,0) rectangle (3,3);
    \draw[-Latex] (0,0) -- (0,1.5);
    \draw[-Latex] (0,3) -- (1.5,3);
    \draw[-Latex] (3,3) -- (3,1.5);
    \draw[-Latex] (3,0) -- (1.5,0);
    \node[left] () at (0,1.5) {$ a $};
    \node[above] () at (1.5,3) {$ b $};
    \node[right] () at (3,1.5) {$ a $};
    \node[below] () at (1.5,0) {$ b $};
  \end{tikzpicture}
  \caption{Toro piatto, o anche toro di Clifford}
  \label{fig:lez3:clifford_torus}
\end{figure}
Si usano simboli combinatori per descrivere l'identificazione: si definisce un verso di percorrenza, si assegnano delle lettere a ciascun
lato e si scrivono in ordine tali lettere, aggiungendo un esponente $ -1 $ quando il verso di percorrenza è opposto. In questo caso quindi
si ha $ aba^{-1}b^{-1} $.

In genrale si ha $ a_1 b_1 a_1^{-1} b_1^{-1} \dots a_g b_g a_g^{-1} b_g^{-1} $.

Si dimostra che queste sono varietà differenziabili, in particolare per $ g = 1 $ si ha un toro, per $ g = 2 $ un bitoro, \dots. $ g $ è
detto \textbf{genere} \index{Genere}.

Si trova con il teorema di Seifert-Van Kampen che:
\[
  \pi_1(V_g) \cong
  \begin{cases}
    1 & \text{se } g = 0 \\
    \Z \oplus \Z & \text{se } g = 1 \\
    < a_1 b_1 \dots \Pi_{i=1}^g [a_i,b_i] = 1 > & \text{se } g > 1
  \end{cases}
\]
Dove $ [,] $ è il commutatore, cioè esattamente $ a_1 b_1 a_1^{-1} b_1^{-1} \dots a_g b_g a_g^{-1} b_g^{-1} $.
Solo per $ g = 0 $ o $ g = 1 $ si ottengono dei gruppi abeliani, ma io vorrei averlo sempre abeliano, quindi lo abelianizzo.
\[
  \Ab{\pi_1(X)} = \quot{\pi_1(X)}{[\pi_1(X), \pi_1(X)]} = \quot{\pi_1(X)}{\pi_1'(X)}
\]
Chiaramente questo gruppo è abeliano e si calcola facilmente che $ \Ab{\pi_1(V_g)} \cong \Z^{2g} $ per $ g \geq 2 $.
Si vedono facilmente anche i generatori, ad esempio per un toro sono riportati in figura

\begin{figure}[htbp]
  \centering
  \input{images/torus_generators.pdf_tex}
  \caption{Generatori di un toro}
  \label{fig:lez3:torus_generators}
\end{figure}

L'abelianizzato è uno $ \Z $-modulo.

\begin{osservation}
  Sia $ X $ uno spazio topologico connesso per archi, $ \mathcal{G} $ un gruppo abeliano. Suppongo esista un omomorfismo
  di gruppi $ \phi: \pi_1(X) \to \mathcal{G} $ allora esiste $ \phi' : \Ab{\pi_1(X)} \to \mathcal{G} $ omomorfismo di gruppi abeliani.
  \[
    \begin{tikzcd}
      \pi_1(X) \arrow{r}{\phi} \arrow{d}{P} & \mathcal{G} \\
      \Ab{\pi_1(X)} \arrow{ur}{\phi'}
    \end{tikzcd}
  \]
  $ P $ è la proiezione sul quoziente. $ \phi' $ esiste perchè in $ \Ab{\pi_1(X)} $ c'è tutto quello che sta nel nucleo.
  $ \phi'(a) = \phi'(P(c)) := \phi(c) $.
  Allora $ \phi'(a) = \phi'(P(d)) = \phi(d) $, devo mostrare che $ \phi(c) = \phi(d) $. Siccome $ \mathcal{G} $ è abeliano $ p(c) \sim p(d) $,
  e quindi $ c = d[x,y] $ per cui: $ \phi(c) = \phi(d[x,y]) $, siccome $ \phi $ è omomorfismo:
  \[
    \phi(d[x,y]) = \phi(d)\phi([x,y]) = \phi(d) \phi(xyx^{-1}y^{-1}) = \phi(d) \phi(x) \phi(y) \phi(x)^{-1} \phi(y)^{-1} = \phi(d)
  \]
  dove nell'ultimo passaggio ho utilizzato che il gruppo è abeliano. Per questo $ \phi' $ è ben definito.
\end{osservation}
Questa osservazione dipende crucialmente dal fatto che il gruppo è abeliano.

Voglio dimostrare che $ \Ab{\pi_1(X)} \cong H_1(X) $, in questo modo per il teorema di Seifert-van Kampen posso ottenere tante
informazioni su $ H_1(X) $. Per ora so che $ H_1(X) $ è uno $ \Z $-modulo. Se costruisco $ \phi: \pi_1(X) \to H_1(X) $ ottengo
gratuitamente la mappa da $ \Ab{\pi_1(X)} $ a $ H_1(X) $.
\[
  \begin{tikzcd}
    \pi_1(X) \arrow{r}{\phi} \arrow{d}{P} & \mathcal{G} \\
    \Ab{\pi_1(X)} \arrow{ur}{\phi'}
  \end{tikzcd}
\]
Poi dovrò mostrare che questa mappa è invertibile, cioè $ \exists \psi:H_!(X) \to A_1(X) $ tale che $ \phi' \circ \psi = \Id{H_1(X)} $ e
$ \psi \circ \phi' = \Id{\Ab{\pi_1(X)}} $. Provo a costruire $ \phi $.
\[
  \begin{aligned}[t]
    \phi:  \pi_1(X) & \to H_!(X) \\
    [f]_H & \mapsto [f]_{hom}
  \end{aligned}
\]
Usando il seguente risultato:
\begin{lemma}
  Se $ f \sim_H g $ allora $ f \sim_{hom} g $.
\end{lemma}
\begin{proof}
  Siccome $ f \sim_{H} g $ allora $ \exists F $ continua tale $ F: I \times I \to X $ tale che $ F(0,x) = f(x) $,
  $ F(1,x) = g(x) $ e $ F(t,0) = x_0 $ in quanto è un laccio.
  [FIGURA]
  Voglio mostrare che è il bordo di un $ 2 $-simplesso. Faccio l'equivalenza $ \quot{I \times I}{0xI} \simeq \Delta_2 $.
  [FIGURA]
  E questo è omeomorfo a un $ 2 $-simplesso standard.
  Siccome rimane costante su $ x_0 $ questa mappa induce $ F' $:
  \[
    \begin{tikzcd}
      I \times I \arrow{r}{F} \arrow{d}{P} & X \\
      \quot{I \times I}{0 \times I} \simeq \Delta_2 \arrow{ur}{F'}
    \end{tikzcd}
  \]
  Calcolo il bordo: $ \partial F' = F'^{(0)} - F'^{(1)} + F'^{(2)} = K - g + f $
  dove $ K $ è il cammino costante per definizione di omotopia. Se $ K $
  fosse il bordo di qualcosa avrei finito ($ \partial w = f - g $). Prendo il $ 2 $-simplesso
  standard $ K $ costante e uguale a $ x_0 $ (è la stessa costante di $ K $):
  \[
    \partial K = K^{(0)} - K^{(1)} + K^{(2)}
  \]
  ma questi sono uguali perché sono costanti.
\end{proof}

\printindex


\end{document}


%%% Local Variables:
%%% mode: latex
%%% TeX-master: t
%%% End:
